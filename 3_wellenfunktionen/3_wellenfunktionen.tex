\renewcommand{\lastmod}{10. September 2024}
\renewcommand{\chapterauthors}{Markus Lippitz}

\chapter{Wellenfunktionen}



\goal{By the end of this chapter, you should be able to draw, calculate and align a ray's path through an optical system.}

Ich kann das Konzept einer Materiewelle benutzen um Experimente mit Teilchen zu erklären, insbesondere deren statistischen Aspekte.


\section{Overview}

39 Wave Functions and Uncertainty 1170

39.1 Waves, Particles, and the Double-Slit Experiment 1171

39.2 Connecting the Wave and Photon Views 1174

39.3 The Wave Function 1176

39.4 Normalization 1178

39.5 Wave Packets 1180

39.6 The Heisenberg Uncertainty Principle 1183

\phet{Quantum_Wave_Interference}

\url{https://www.spektrum.de/magazin/komplementaritaet-und-welle-teilchen-dualismus/822095}



% 3. Probability and Randomness and Wave particle duality
% Sim: Quantum Wave Interference
% • When we ask how students visualize light, ~40% have “Bohmian” view of particle traveling
% alongside EM wave.
% • We use sim to demonstrate how the double slit experiment shows that light must be both a
% wave that goes through both slits and a particle that hits the screen at a single location. This
% lecture led to an unexpected onslaught of deep, fundamental questions that took up nearly an
% entire class period. Many students ask whether the particle is actually inside wave with a
% definition location that we just don’t know. Students get pretty frustrated with this class.


% 9. Double slit and Davisson Germer experiment
% Sims: Quantum Wave Interference, Davisson Germer: Electron Diffraction
% • Students have a much harder time thinking of electrons as waves than photons, because
% electrons have mass.
% • Students often think that the size of the wave packet, rather than the wavelength, should
% determine the spacing of the interference pattern.
% • We have noticed that students often miss the point of the Davisson Germer experiment. They
% remember that electrons were only detected at certain angles, but cannot explain why. They
% view the electrons as particles that happen to bounce off at certain angles for some reason
% they can’t understand, rather than recognizing how the observations can be explained by the
% wave nature of electrons. We have found two things that really help to address this:
% - Start with a review of the double slit experiment, a context where students understand
% interference much better, talking about how you would like to do this to test deBroglie’s
% hypothesis, and then explain why this is really hard to do and then talk about how the
% Davisson Germer experiment is analogous.
% - Use the Davisson Germer sim to illustrate how wave interference leads to peaks in
% intensity at certain angles.

% 10. Wave functions and probability
% • When we first introduce wave functions with arbitrary functions, students often don’t
% recognize these as waves because they think “waves” are sine waves.
% • “Wave number” is usually new and unfamiliar to students, and it’s worth spending 5 minutes
% to discuss why we define this quantity and how it relates to wavelength.

% 11. Wave packets and uncertainty principle
% Sims: Quantum Wave Interference, Quantum Tunneling, Fourier: Making Waves
% • We introduce wave packets early because they are much more intuitive than plane waves and
% easier to relate to “particles.”

% 12. Wave equations and Differential equations
% • It is worth emphasizing that the way we solve differential equations in physics, by just
% “guessing” the solution, is completely different from the way they have been taught to solve
% differential equations in math classes. Many students make this section way too hard by
% attempting to use complicated methods they have learned in math classes.

% 13. Schrodinger equation for free particle
% Sim: Quantum Tunneling
% • You can use the Quantum Tunneling sim to demonstrate free particles by just setting the
% potential to “constant.”
% • Students often have difficulty understanding the meaning of complex wave functions. This
% can perhaps best be illustrated by the observation that students frequently ask, “What is the
% physical meaning of the imaginary part of the wave function?” but never ask about the
% physical meaning of the real part, even though both have the same physical significance


%--------------------
\printbibliography[segment=\therefsegment,heading=subbibliography]
