\renewcommand{\lastmod}{10. September 2024}
\renewcommand{\chapterauthors}{Markus Lippitz}

\chapter{Quantisierung}



\goal{By the end of this chapter, you should be able to draw, calculate and align a ray's path through an optical system.}

Ich kann das Planck’sche Strahlungsgesetz herleiten und seinen Zusammenhang mit quantisierter Strahlung erklären. 

Ich kann wichtige Versuche zur Quantisierung (Photoeffekt, Compton-Effekt, Frank-Herz-Versuch, Millikan-Versuch, Spektrallinien) erklären und darstellen, wie diese eine Quantenhypothese unterstützen. 

Ich kann das Modell aus den Bohr’schen Postulaten herleiten, es in relevanten Fällen anwenden und seine Grenzen erklären.

\section{Overview}

s.a. Demtröder 3, Kap. 3


38.1 The Photoelectric Effect 1138


38.2 Einstein's Explanation 1141

38.3 Photons 1144

38.4 Matter Waves and Energy Quantization 1148

38.5 Bohr's Model of Atomic Quantization 1151

38.6 The Bohr Hydrogen Atom 1155

38.7 The Hydrogen Spectrum 1160

PLUS Planck’sches Strahlungsgesetz

\phet{Photoelectric_Effect}
\phet{Quantum_Wave_Interference}
\phet{Neon_Lights_and_Other_Discharge_Lamps}
\phet{DavissonGermer_Electron_Diffraction}
\phet{Fourier_Making_Waves}

\url{https://www.spektrum.de/magazin/100-jahre-quantentheorie/827483}

\url{https://www.spektrum.de/magazin/bedroht-diequantenverschraenkung-einsteins-theorie/1002937}

2.2 Moden eines Hohlraums 2	

2.3 Planck’sches Strahlungsgesetz [2]3	

3.3 Compton-Effekt [2] 3	

3.4 Strahlungsdruck und Impuls des Photons [2]4	

6.2 Franck-Hertz-Versuch2	hier ??

2.5 Treibhaus


% 2. Photoelectric Effect
% Sim: Photoelectric Effect
% • This is a much harder topic for students than professors think. For details, see:
% www.colorado.edu/physics/EducationIssues/papers/McKagan_etal/photoelectric.pdf
% • Common student difficulties (many can be resolved with sim):
% - think voltage rather than light takes electrons off plate
% - think current increases with speed of electrons
% - can’t explain basic function of experiment
% - can’t explain classical model of light
% - can’t explain why PE experiment leads to photon model of light
% • A general problem that first appears here is that some students have no ability to think
% hypothetically and can’t separate what was expected classically from what really happens.


% 5. Atomic Spectra and Discharge Lamps
% Sim: Discharge Lamps
% • We teach spectra before the Bohr model in order to emphasize how Bohr was able to explain
% the observed spectra with his model.
% • Students often have trouble with the idea that the energy of light corresponds to the difference
% between the levels rather than the values of the levels. They need lots of explicit practice to
% get this distinction straightened out.
% • We get lots of questions about how the electron chooses which level to jump down to, and
% how it decides when to jump down. These questions are useful later for emphasizing why the
% Schrodinger model of the atom is better than the Bohr model.
% • The simulation and associated homework really help students build a clear model of how a
% discharge lamp work. The one place they had trouble was relating this model to what they
% see in a real discharge lamp, even though we did a demo with real discharge lamps and
% diffraction gratings. It’s important to be really explicit in this demo about how the physical
% lamps relate to the model in the sim.
% • When reminding students of Coulomb potential energy, they remember the equation kq1q2/r,
% but often don’t realize that this is the same as –ke²/r.
% • The idea of how fluorescent lights work is harder for students than you might think because
% they have trouble with the idea that red+blue+green light l



% 7. Balmer Series
% • We emphasize the point that Balmer came up with his formula by playing around with
% numbers and didn’t know what it meant. This is probably lost of students who think all of
% physics is like that.


% 8. Bohr and deBroglie Models of the atom
% Sim: Models of The Hydrogen Atom
% • For details about why and how we teach this topic, see:
% www.colorado.edu/physics/EducationIssues/papers/McKagan_etal/BohrModel_McKagan_etal.pdf
% • This is really an opportunity to teach modeling and the significance of Bohr explaining where
% Balmer’s equation came from and deBroglie explaining why there are fixed energy levels.
% This is a really difficult section for students who have trouble thinking hypothetically.
% • In the Bohr model, students often mix up total and potential energy, for example, thinking
% that -13.6eV is the potential energy. This confusion is confounded by the way the total
% energy lines are drawn on top of the potential energy curves.

%--------------------
\printbibliography[segment=\therefsegment,heading=subbibliography]
