\renewcommand{\lastmod}{10. September 2024}
\renewcommand{\chapterauthors}{Markus Lippitz}

\chapter{Licht-Materie-Wechselwirkung}



\goal{By the end of this chapter, you should be able to draw, calculate and align a ray's path through an optical system.}

Ich kann stimulierte Emission und die Funktionsweise eines Lasers erklären.

Ich kann das Röntgenspektrum einer Anode erklären und relevante Größen berechnen.


\section{Overview}

s.a. Demtröder 3, Kap. 7

 
7.1 Einstein-Koeffizienten 1	** (41.8 Stimulated Emission and Lasers 1252)


7.2 Laser 2	** (41.8 Stimulated Emission and Lasers 1252)

7.3* Quantenmechanik optischer Übergänge 3	Wdh (41.6 Excited States and Spectra

7.4 Linienbreite 4	*	4 

7.5 Eigenschaften der Fourier-Transformation 5	*	5 

7.6 Röntgenstrahlung 6	***	

9.5 Moseley’sches Gesetz 5	

\phet{Lasers}


% 6. Lasers
% Sim: Lasers
% • We originally covered Lasers towards the end of the course, but we realized that we didn’t
% actually use anything other than the basics of spectra in our treatment, and the engineers got
% grumpy if we spent too long on fundamentals without any applications, so we moved Lasers
% to so that there was more emphasis on applications early in the course. This worked much
% better.
% • When we ask students why laser beams are so powerful, it’s split 50/50 between more power
% in the beam and more concentrated light.
% • The homework on lasers starts with basic questions about absorption and spontaneous and
% stimulated emission, works through the steps of building a laser and troubleshooting a broken
% laser, and ends with essays on why a population inversion is necessary to build a laser and
% why this requires atoms with three energy levels instead of two. Most students are able to
% give coherent explanations in these es


%--------------------
\printbibliography[segment=\therefsegment,heading=subbibliography]
