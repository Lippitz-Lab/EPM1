\renewcommand{\lastmod}{20. Dezember 2024}
\renewcommand{\chapterauthors}{Markus Lippitz}

\chapter{Rotation in Molekülen}



\section{Überblick}


In diesem und den folgenden beiden  Kapiteln behandeln wir die Spektroskopie von Molekülen mit dem Ziel, die Eigenschaften des Bindungspotentials zu bestimmen. Wir beginnen mit einem allgemeinen, spektral sehr breiten Überblick über die spektroskopischen Eigenschaften von Molekülen, die in ihren dielektrischen Eigenschaften zum Ausdruck kommen. Die dielektrische 'Konstante' kann als eine Summe von Lorentz-Oszillatoren dargestellt werden.  Jeder Oszillator, jede Resonanz entspricht einem Freiheitsgrad des Moleküls. In diesem Kapitel werden Rotationen behandelt, in den darauf folgenden  Kapiteln Schwingung und  elektronische Anregung. Elektronische Anregung ist die Anregung, die wir schon bei den Atomen besprochen haben. Aber nur Moleküle, die aus mehr als einem Atom bestehen, können gegeneinander schwingen oder sich drehen.

Der zweite Abschnitt dieses Kapitels befasst sich also mit den Rotationen der Moleküle und wie sie sich im Spektrum widerspiegeln. Wir werden sehen, dass Rotationen im Modell des starren Rotators zu einer Reihe von äquidistanten Linien mit einer charakteristischen Amplitudenverteilung führen.

Ein schwingendes Molekül zeigt zunächst nur eine Linie im Spektrum. In Kombination mit der Rotation entstehen jedoch weitere Linien, wenn sich bei einem optischen Übergang die Rotationsquantenzahl und die Schwingungsquantenzahl gleichzeitig ändern. Schließlich versuchen wir, die Vielzahl möglicher Schwingungen in mehratomigen Molekülen zu beschreiben.


\section{I Beispiel: Wasser}

\begin{figure}
\inputtikz{\currfiledir fig_water}
\caption{Dielektrische Funktion $\epsilon = \epsilon' - i \, \epsilon''$ von flüssigem Wasser (\cite{Segelstein_water} via 
\href{https://refractiveindex.info/?shelf=main&book=H2O&page=Segelstein}{refractiveindex.info}). Der niederfrequente Bereich unterhalb von $\bar{\nu} = 30$~cm$^{-1}$ ist um den Faktor 20 reduziert dargestellt.
\label{fig:diel_water}}

%D. J. Segelstein. The complex refractive index of water, Master Thesis (1981)
%https://refractiveindex.info/?shelf=main&book=H2O&page=Segelstein
% http://hdl.handle.net/10355/11599 
\end{figure}


Das in Abbildung \ref{fig:diel_water}  gezeigte Spektrum von flüssigem Wasser überdeckt mehr als 6 Größenordnungen im Frequenzbereich und ist mit einem einzigen Gerät nicht  messbar. Die niedrigsten Frequenzen liegen im Bereich von Mikrowellen, benutzt also Techniken aus dem Radio- und Radar-Bereich. Der mittlere Frequenzbereich ist infrarotes Licht, der hohe sichtbares bis ultraviolettes Licht. Wir werden die verschiedenen spektroskopischen Methoden in diesem und dem nächsten  Kapitel besprechen.


Die Wellenzahl $\bar{\nu} = 1 / \lambda =\nu / c =  E / (h c)$ (eigentlich immer angegeben in der Einheit 1/cm) ist eine Maß für die  \emph{Frequenz} oder \emph{Energie}. Dieses Maß in der Spektroskopie sehr weit verbreitet ist, weil es proportional zur Energie ist (und damit das Reziproke der Wellenlänge umgeht), aber gleichzeitig nahe an der praktischen, intuitiven Größe 'Wellenlänge'.



% \begin{questions} 
% \item Wo in  Abbildung  \ref{fig:diel_water} ist der sichtbare Spektralbereich, wo die Frequenz von Radio Mainwelle?
% \end{questions}

Wie kommt es aber nun zu diesem Spektrum? Wenn wir ein Stück dielektrische Materie in einen Plattenkondensator halten, dann bewegt das elektrische Feld $\mathbf{E}$ des Kondensators die Ladungsträger der Ladung $q$ in der Materie um  die Distanz $\Delta  \mathbf{x}$ von der neutralisierenden Gegenladung $-q$ weg. Dadurch entsteht dadurch ein Dipolmoment $\mathbf{p} = q \Delta \mathbf{x}$. Oft werden Dipolmomente in der Einheit Debye angegeben ($1 D = 3{,}33564 \cdot 10^{-30}$ C  m). Ein Elektron im Abstand von 1~\AA\ von einem Proton produziert ein Dipolmoment von etwa 4.8~D. Bei $N$ Molekülen (Ladungsträgerpaaren) pro Volumen ergibt sich eine makroskopische Polarisation $\mathbf{P}$ zu
\begin{equation}
\mathbf{P} = N \, q \, \Delta \mathbf{x} = f(\mathbf{E}) \quad .
\end{equation}
Der Zusammenhang zwischen angelegtem externen Feld $\mathbf{E}$ und resultierende 
Polarisation $\mathbf{P}$ hängt ganz entscheidend vom mikroskopischen Aufbau der Materie ab. Alle Methoden der Spektroskopie vermessen diesen Zusammenhang. Oft wird er als 
\begin{equation}
 \mathbf{P} =  (\epsilon - 1) \, \epsilon_0 \, \mathbf{E} = \epsilon_0 \,\chi \, \mathbf{E} 
 \label{eq:diel_p_lin}
\end{equation}
geschrieben, mit der relativen Permittivität\sidenote{Vorsicht, hier gibt es verschiedene Schreibweisen. Ich benutze die Form $D = \epsilon \epsilon_0 E$, mit einheiten-freiem $\epsilon$. Manchmal findet man $\epsilon_0 \epsilon_r$, manchmal auch nur $\epsilon$.} $\epsilon$ bzw. der elektrischen Suszeptibilität $\chi$. 


Die Aussage von Abbildung \ref{fig:diel_water} ist, dass die  dielektrische 'Konstante' $\epsilon$   nicht  konstant ist, sondern eine dielektrische Funktion der Frequenz, also $\epsilon(\nu)$. Verschiedene Prozesse tragen zur Frequenzabhängigkeit bei:




\paragraph{Orientierungspolarisation} Die Orientierung des Moleküls im Feld ist eine Drehbewegung. Diese Bewegungen werden bei der \emph{Rotationsspektroskopie} unten detaillierter behandelt werden. Hier greifen wir vor. Der Drehimpuls $L$ ist in der Quantenmechanik quantisiert. Sei
\begin{equation}
1 \hbar = L = J \, \omega = m_\text{red} \, R^2 \, \omega
\end{equation}
mit dem Trägheitsmoment $J$, der Kreisfrequenz $\omega$ und der reduzierten Masse 
 $m_\text{red}$ sowie dem Bindungsabstand $R$ in einem angenommenen zwei-atomigen Molekül. Für das Molekül HCl gilt $R = 1.28$~\AA\ und $m_\text{red} \approx m_H = 1$~u. Damit ergibt sich eine Frequenz $\nu = 628$ GHz, also im Mikrowellen-Bereich. Oft wird dies auch geschrieben als Wellenzahl $\bar{\nu} = 1 /\lambda \approx 10$~cm$^{-1}$.
 
\paragraph{Verschiebung der Kerne} Bei der Verschiebungspolarisation können sich zunächst einmal die Kerne bzw. Ionen gegeneinander bewegen. Dies ist Thema der \emph{Schwingungsspektroskopie}, und wieder greifen wir vor. Zwei Atome seien im Gleichgewicht  im Bindungs-Abstand $R_0$. Wir nehmen an, die Rückstellkraft in diesem Gleichgewicht sein allein die Coulomb-Kraft des einen Kerns auf den anderen, also 
\begin{equation}
F = \frac{1}{4 \pi \epsilon_0} \, \frac{e^2}{R^2} \quad .
\end{equation}
Die Federkonstante $k$ ist dann die Ableitung dieser Kraft nach $R$. Für das Molekül HCl mit einem Gleichgewichts-Abstand $R_0 = 1.2$~\AA\ ergibt sich $k = 220$~N/m. Die Eigenfrequenz der Schwingung  ist $\nu = (1/2\pi) \, \sqrt{k/m_\text{red}} = 58$~THz mit der reduzierten Masse von oben. Dies entspricht einer Wellenlänge von $\lambda = 5.12$~\textmu m, also im Infraroten, und einer Wellenzahl $\bar{\nu} = 2000$~cm$^{-1}$.

\paragraph{Verschiebung der Elektronenwolke} Wenn es zu einer Resonanz in der Verschiebung der Elektronenwolke kommt, dann entspricht dies einer \emph{elektronischen Anregung}, also einem quantenmechanischen Übergang zwischen zwei Elektronen-Orbitalen. Dies wird uns  im nächsten Kapitel beschäftigen. Wir schätzen hier die Übergangsenergie analog zu atomaren Übergängen ab
\begin{equation}
  h \nu = hc \, R_H \, \left( \frac{1}{n^2} - \frac{1}{m^2} \right) \quad .
\end{equation}
Für Atome liegt die Frequenz $\nu$ im Bereich von $10^{15}$~Hz $= 1$~PHz. Bei Molekülen liegt sie etwas niedriger, also $\nu \approx 10^{14} \cdots 10^{15}$~Hz, also $100 \cdots 1000$~THz.


\section{Lorentz-Oszillator-Modell}


\begin{marginfigure}
\inputtikz{\currfiledir lorentz_oszillator}

\caption{Frequenzabhängigkeit des Real- und Imaginärteils des Lorentz-Oszillators. Real- und  Imaginärteil des komplexwertigen Brechungsindex $\tilde{n}$ sehen qualitativ gleich aus. \label{fig:diel_lorentz}}
\end{marginfigure}

Alle oben diskutieren Phänomene sind Resonanzen. Das Lorentz-Oszillator-Modell ist ein einfaches Modell, mit dem die Frequenzabhängigkeit der dielektrischen Funktion in der Nähe solcher Resonanzen beschrieben werden kann. In einem gedämpften harmonischen Oszillator (Masse $m$, Dämpfungskonstante $\gamma$, Eigenfrequenz $\omega_0$) wird die Masse durch ein periodisches elektrisches Feld (Amplitude $E_0$, Frequenz $\omega$) um $x$ ausgelenkt, da die Masse eine Ladung $e$ trägt. Alles zusammen
\begin{equation}
 m \ddot{x} +  \gamma \dot{x} + m \omega_0^2  x = e E_0 e^{+ i \omega t} \quad .
\end{equation}
Die stationäre Lösung dieser Differentialgleichung ist
\begin{equation}
 x(t) =  \frac{e \, E_0}{m (\omega_0^2  - \omega^2) + i \gamma \omega} \, e^{+ i \omega t} \quad .
\end{equation}
Die makroskopische Polarisation $P$ ist die Summe über alle mikroskopische Polarisationen, also
\begin{equation}
P(t) = N \, e \,x(t) =  (\epsilon -1 ) \epsilon_0 \, E_0 e^{+ i \omega t}
= \chi \epsilon_0 E(t) \quad .
\end{equation}
Damit ergibt sich die dielektrische Funktion
\begin{equation}
\epsilon(\omega) = 1 + N \alpha = 1 +\frac{N e^2}{\epsilon_0} \frac{1}{m (\omega_0^2  - \omega^2) + i \gamma \omega} = \epsilon' - i \epsilon'' \quad .
\end{equation}
Man beachten das per Konvention negative Vorzeichen des Imaginärteils $\epsilon''$. Explizit sind Real- und Imaginärteil
\begin{align}
 \epsilon' = & 1 + \frac{N e^2}{\epsilon_0} \frac{ m (\omega_0^2  - \omega^2)}{m^2 (\omega_0^2  - \omega^2)^2 +  \gamma^2 \omega^2}  \\
  \epsilon'' = &  \frac{N e^2}{\epsilon_0} \frac{ \gamma \omega }{m^2 (\omega_0^2  - \omega^2)^2 +  \gamma^2 \omega^2}  \quad .
\end{align}
Für den  komplexwertige\sidenote{Oft wird nicht zwischen $n$ und $\tilde{n}$ unterschieden und $n$ selbst ist komplexwertig.} Brechungsindex\sidenote{Die hier benutzte Konvention der Vorzeichen ist die in der Physik übliche, ausgehend von der Zeitabhängigkeit $e^{+ i \omega t}$. In der eher ingenieurwissenschaftlichen Literatur findet sich aber genauso oft auch die Zeitabhängigkeit $e^{- i \omega t}$. Dies führt zu komplex-konjugierten Gleichungen. } $\tilde{n} = n - i k$ gilt
\begin{align}
 \epsilon = & \tilde{n}^2 = (n - i k)^2 \\
  \epsilon' =& n^2 - k^2 \\
 \epsilon'' = & 2 n k \quad .
\end{align}
Dabei beschreibt $k$ die Dämpfung und $n$ die Dispersion, also die Variation der effektiven Wellenlänge in der Nähe einer Resonanz:
\begin{equation}
E(t,z) = E_0 \, e^{i \omega (t - \frac{z}{c}(n - i k))} = 
 E_0 \, e^{ - \frac{\omega}{c} k z}  
 \, e^{i \omega (t - \frac{z}{c/n} )}  \quad .
\end{equation}

Wenn in einem Medium mehrere Resonanzen vorhanden sind, so addieren sich die Suszeptibilitäten:
\begin{equation}
\epsilon(\omega)_\text{ges} = 1 + \chi(\omega)_\text{elec} +  \chi(\omega)_\text{ion}  + \chi(\omega)_\text{orient} \quad .
\end{equation}

\begin{marginfigure}
\inputtikz{\currfiledir multiple_lorentz_oszillator}

\caption{Addition der Suszeptibilitäten. \label{fig:diel_multiple_lorentz}}
\end{marginfigure}

Für Fensterglas liegt die elektronische Resonanz im Ultravioletten. Sichtbares Licht ist also im Frequenzbereich etwas unterhalb dieser Resonanz. Die sogenannte 'normale' Dispersion des Brechungsindex (ansteigend mit fallender Wellenlänge / steigender Frequenz) rührt gerade von dem Anstieg der dielektrischen Funktion zur elektronischen Resonanz hin her.



%%%%%%%%%%%%%%%%%%%%%%%%%%%%%%%%%%%%%%%%%%%%%%%%%


\section{II Rotation von Molekülen}




\begin{figure}
\inputtikz{\currfiledir fig_hcl}
\caption{Mikrowellen-Transmissionspektrum durch  \ch{HCl}-Gas  (\cite{Li_2011_hcl} via \href{https://hitran.org}{hitran.org}).
\label{fig:rot_hcl}}
\end{figure}




Der dargestellte Bereich  von etwa 10 bis 300 cm$^{-1}$ entspricht einer Frequenz von 0.3 bis 9 THz oder einer Wellenlänge von 1000 bis 33 \textmu m. Dieser Spektralbereich ist experimentell nicht einfach zugänglich (\emph{THz gap}). Für niedrigere Frequenzen unterhalb etwa 0.3 THz, im Mikrowellen-Bereich, existieren in der Frequenz durchstimmbare  Mikrowellen-Generatoren (Klystron) und passende Detektoren. Bei höheren Frequenzen ist dies technisch aufwändig. Möglichkeiten sind das Synchrotron oder durch sehr kurze Laserpulse erzeugte THz-Pulse.

Nach Erzeugung der Strahlung durch Klystron oder Synchrotron wird diese durch ein möglichst langes Volumen des zu untersuchenden Gases geleitet, da die Absorption gering ist. Die transmittierte Leistung wird dann als Funktion der Frequenz des Generators gemessen. So erhält man  Spektren ähnlich zu oben stehender Abbildung.

Die Transmission $T$ (oder auch die Absorption) ist eine etwas unpraktische Größe, da sie immer im Bereich zwischen Null und Eins liegt, also beispielsweise nicht linear von der Konzentration des Gases abhängt. Daher betrachtet man eigentlich immer die Absorbanz oder Extinktion.\sidenote{Der Unterschied zwischen Absorbanz und Extinktion ist, dass letztere auch Streuung beinhaltet, was für uns aber keine Rolle spielt.} Die Extinktion $E$ ist
%
\begin{equation}
 E = - \log_{10} T \quad .
\end{equation}
%
Im Folgenden betrachten wir also Absorbanz- oder Extinktions-Spektren, die oft auch einfach Absorptionsspektren genannt werden, auch wenn nicht $1-T$ sondern $\log_{10} ( 1- T)$ dargestellt ist.

\begin{marginfigure}
\inputtikz{\currfiledir fig_hcl_extinction}
\caption{Das HCl-Spektrum aus Abbildung \ref{fig:rot_hcl} als Extinktionsspektrum.}
\end{marginfigure}


% \begin{questions} 
% \item Wo in  Abbildung  \ref{fig:rot_hcl} arbeitet ein Mikrowellenherd?
% \end{questions}


\section{Modell des starren Rotators}

Ein einfaches Modell, um Rotationsspektren zu beschreiben, ist das des starren Rotators. Wir nehmen eine klassische Hantel mit zwei Massen $m_1$ und $m_2$ an, die durch eine starre Achse der Länge $R$ miteinander verbunden sind. Das Trägheitsmoment der Hantel ist
\begin{equation}
 \Theta = \frac{m_1 \, m_2}{m_1 + m_2} \, R^2 = m_\text{red} \, R^2 \quad .
\end{equation}
Damit berechnet sich die Rotationsenergie $E_\text{rot}$ zu
\begin{equation}
 E_\text{rot} = \frac{1}{2} \, \Theta \, \omega^2
\end{equation}
mit der Rotationsfrequenz $\omega$. Die Quantenmechanik kommt durch die Quantisierung des Drehimpulses $\mathbf{L} $ ins Spiel:
\begin{equation}
 | \mathbf{L} | = \Theta \, \omega = \hbar \sqrt{J (J + 1)}
\end{equation}
mit der Drehimpuls-Quantenzahl $J = 0, 1, \dots$. Die Rotationsenergie ist damit
\begin{equation}
 E_\text{rot} = \frac{ | \mathbf{L} |^2}{2 \Theta} = \frac{\hbar^2 \, J (J+1)}{2 \Theta} \quad .
\end{equation}

Dies sind die Energien der \emph{Zustände} des Systems, noch nicht die Lage der Peaks im Spektrum. Bei der Absorption eines Mikrowellen- oder THz-Photons ändert sich der Zustand. Wir suchen also die Energien der \emph{Übergänge} zwischen Zuständen, um die Lage der Peaks im Absorptionsspektrum zu beschreien.


% \begin{questions} 
% \item Um welche Achse dreht man die Hantel in diesem Modell?
% \end{questions}


\section{Auswahlregeln bei Rotationsübergängen}

Zwischen welchen Zuständen können unter welchen Umständen Übergänge durch Absorption (oder Emission) eines Photons stattfinden? Dies beschreiben  die Auswahlregeln.

Zunächst muss die Rotationsbewegung überhaupt an das elektromagnetische Feld koppeln. Dies verlangt  ein statisches, permanentes Dipolmoment des Moleküls. Klassisch hätte man so einen oszillierenden Dipol, und diese Bedingung bleibt auch in der Quantenmechanik erhalten. Damit sind homonukleare Moleküle (z.B. \ch{H2}), symmetrische lineare Moleküle (z.B. \ch{CO2}) und hoch-symmetrische Moleküle (z.B. \ch{CCl4}) ausgeschlossen. Dieser Ausschluss kann, wie wir unten sehen werden, durch eine Schwingung des Moleküls wieder aufgehoben werden.

Wenn optische Rotationsübergänge im Prinzip möglich sind, dann muss noch die Drehimpuls-Erhaltung erfüllt sein. Die Summe des Drehimpulses von Molekül und Photon muss erhalten bleiben. Der Drehimpuls des Photons ist $1 \hbar$. Bei der Absorption eines Photons muss sich also $J$ erhöhen, bei der Emission erniedrigen.\sidenote{Glücklicherweise passt das mit der Änderung der Energie zusammen.} Damit ergibt sich als Auswahlregel
\begin{equation}
\Delta J = \pm 1 \quad \text{und} \quad \Delta M_J = 0, \pm 1 \quad .
\end{equation}
$M_J$ ist die Orientierungs-Quantenzahl zur Drehimpuls-Quantenzahl $J$ des Moleküls, wie immer bei Drehimpuls-artigen Größen.

\section{Modellierung des Spektrums}

\begin{marginfigure}
\inputtikz{\currfiledir fig_states}
\caption{Skizze Zustände und Übergange.}
\end{marginfigure}


Aus der Lage der Zustände $E_\text{rot}(J)$ und der Auswahlregel $\Delta J = \pm 1$ erhalten wir die  Energie (bzw. hier eigentlich Wellenzahl) der erlaubten Übergänge
\begin{align}
 \bar{\nu}_{J \rightarrow J + 1} =& \frac{1}{h c}  \, \left[ E_\text{rot}(J+1) - E_\text{rot}(J) \right]
 \\ 
 =  & \frac{1}{h c}\frac{\hbar^2}{2 \Theta} \, \left[ (J+1)(J+2) - J (J+1) \right] \\
 = & 2 \, \frac{h}{8 \pi^2 c \, \Theta} \, \left( J +1 \right) = 2 \, B \, (J+1) \quad ,
\end{align}
wobei $B = h / (8 \pi^2 c \, \Theta)$ \emph{Rotationskonstante} genannt wird. Die Linien sind im Spektrum also äquidistant, mit dem Abstand $2B$ und auch die erste Linie ist gerade im Abstand $2B$ vom Ursprung. Dies entspricht zumindest qualitativ dem in Abbildung \ref{fig:rot_hcl} gezeigtem Spektrum. Aus dem Abstand der Linien lässt sich der Gleichgewichts-Bindungsabstand $R_0$ bestimmen, wenn die Atom-Massen bekannt sind.



\begin{marginfigure}
\inputtikz{\currfiledir entartung_vs_boltzman}
\caption{Verlauf von $2J +1$ und Boltzmann-Faktor mit $J$.}
\end{marginfigure}


Im Spektrum sieht man weiterhin einen charakteristischen, nicht-monotonen Verlauf der Amplituden der Linien mit der Übergangsfrequenz. Zunächst wächst die Linien-Stärke (oder Amplitude) mit steigende Übergangsfrequenz, um dann wieder abzufallen. Die Ursache dafür sind zwei gegenläufige Effekte. Zum einen steigt der Entartungsgrad mit $J$, da ja $M_J = 0, \pm 1, ... \pm J$. Es gibt also $2J+1$ Zustände mit gleicher Quantenzahl $J$.
Zum anderen fällt die Besetzung des Ausgangszustands mit steigendem $J$. Um überhaupt einen Übergang machen zu können muss ja der Ausgangszustand besetzt sein. Dies geschieht durch thermische Anregung und folgt einer Boltzmann-Verteilung. Die thermische Energie $k_B T$ bei Raumtemperatur entspricht  $\bar{\nu}_{k T} \approx 200$~cm$^{-1}$, liegt also im hier relevanten Energiebereich. Zusammen ergibt sich so für die Besetzung $N_J$ von Zustand $J$
\begin{equation}
 \frac{N_J}{N_0}= (2J+1) \, e^{- E(J) / k_B T} = (2J+1) \, e^{- B hc J (J+1) / k_B T}  \quad .
\end{equation}
Die Besetzung beeinflusst wesentlich die Amplitude der Linien. Um sie wirklich zu berechnen, müsste man noch stimulierte Emission und das nicht konstante Matrixelement des Übergangsdipols berücksichtige. Dies führt hier zu weit, ist aber in \cite{Demtröder_molekuelphysik} dargestellt.






% \begin{questions} 
% \item Bedeutet $\Theta_z = 0$ beim linearen Kreisel, dass das Molekül sich sehr schnell oder gar nicht um die Molekül-Achse dreht? Wenn Sie einen Bleistift um seine drei Achsen drehen, welche dreht sich dann 'einfacher'?
% \end{questions}





%%%%%%%%%%%%%%%%%%%%%%%%%%%%%%%%%%%%%%%%



\section{Zusammenfassung}

\textit{Schreiben Sie hier ihre persönliche Zusammenfassung des Kapitels auf. Konzentrieren Sie sich auf die wichtigsten Aspekte.}

\vspace*{10cm}


%--------------------
\printbibliography[segment=\therefsegment,heading=subbibliography]
