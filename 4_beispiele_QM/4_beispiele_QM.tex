\renewcommand{\lastmod}{10. September 2024}
\renewcommand{\chapterauthors}{Markus Lippitz}

\chapter{Beispiele aus der (1d) Quantenmechanik}



\goal{By the end of this chapter, you should be able to draw, calculate and align a ray's path through an optical system.}



\section{Overview}

s.a. Demtröder 3, Kap. 4


40.1 The Schrödinger Equation 1194

40.2 Solving the Schrödinger Equation 1197

40.3 A Particle in a Rigid Box: Energies and Wave Functions 1199

40.4 A Particle in a Rigid Box: Interpreting the Solution 1202

40.5 The Correspondence Principle 1205


40.6 Finite Potential Wells 1207

40.7 Wave-Function Shapes 1212

40.8 The Quantum Harmonic Oscillator 1214

40.9 More Quantum Models 1217

40.10 Quantum-Mechanical Tunneling 1220

\phet{Quantum_Bound_States}
\phet{Quantum_Tunneling_and_Wave_Packets}

5.3 Harmonischer Oszillator 3	


% 14. Potential Energy
% • We have found that without explicit instruction on how to relate potential energy diagrams to
% physical systems, most students don’t know what a potential energy diagram means or how it
% relates to anything real. When we do give explicit instruction on this, students start asking a
% lot of questions, and it becomes clear what a struggle it is for them to make sense of it.
% • The fact that we often use the symbol V for potential energy and use the words “potential”
% and “potential energy” interchangeably leads to a lot of students thinking that V is actually
% the electric potential. It is worth emphasizing repeatedly that this is NOT what it means.

% 15. Infinite and Finite Square Wells
% Sim: Quantum Bound States
% • We illustrate a finite square well with the physical example of an electron in a short wire, and
% illustrate an infinite square well with the same system with a really big work function. We
% justify why this potential energy represents this system by building it up from a microscopic
% model of the atoms in the wire. Before we did this, we found that students often mixed up
% wells and barriers. Afterwards, this happened much less.
% • The practice of drawing potential energy, total energy, and wave function on the same graph
% leads students to confuse these quantities. We have several clicker questions to elicit and
% address this confusion.

% 16. Quantum Tunneling, Alpha decay and other applications of Tunneling
% Sim: Quantum Tunneling
% • Language such as “potential well,” “step,” and “barrier” often leads students to interpret
% potential energy diagrams as physical objects. It is worth pointing out that these words are
% only analogies, and using examples such as an electron tunneling through an air gap where
% there is clearly no physical barrier.
% • There is a lot of research showing that students often believe that energy is lost in tunneling,
% and we have incorporated a tutorial and homework designed to address this belief. Two main
% reasons that students think this are that they mix up energy and wave function and interpret
% the exponential decay of the wave function as energy loss, or that they think of a classical
% object penetrating a physical barrier, in which case there is always dissipation. To address the
% first reason, we avoid drawing the wave function and the energy on the same graph, and ask
% several clicker questions designed to elicit and address this confusion. To address the second
% reason, we emphasize that there is no dissipation in the Schrodinger equation.
% • While plane waves are mathematically simple, conceptually it is quite difficult to imagine a
% wave that extends forever in space and time, especially when it is tunneling. The language we
% use to describe tunneling is time-dependent. For example, we say that a particle approaches a
% barrier from the left, and then part of it is transmitted and part of it is reflected. This language
% is difficult to reconcile with a picture of a particle that simultaneously incident, transmitted,
% and reflected, for all time. We find that it works much better to start instruction with wave
% packets, using a qualitative description and the Quantum Tunneling sim, and then show how
% plane waves make the math easier.
% • Determining the potential energy function for a physical example such as an STM or α decay
% actually requires understanding many steps and approximations, and is not trivial for students.
% If you simply present students with these potential energy functions, they usually don’t know
% how to relate them to the physical systems they are supposed to represent. We have many
% clicker questions designed to help students build a model for the potential energy functions
% for STMs and α decay.

% 17. Reflection and Transmission
% Sim: Quantum Tunneling
% • There is a lot of math here and it’s very easy to get lost in the math and forget why you’re
% doing it. When asking students to work through it in homework, it’s very useful to ask them
% to stop after each step in the math and explain in words what they just did.
% 18. Superposition, measurement, and expectation values
% Sim: Quantum Bound States
% • Modern Physics textbooks typically do not cover superposition and measurement.
% We do, because it seems to us that if you don’t talk about measurement, you don’t know what
% you’re actually doing in QM or how it relates to the real world.
% • We have chosen to cover expectation values in some semesters and not others. It helps to
% relate it to more familiar examples such as grade distributions and gambling.

%--------------------
\printbibliography[segment=\therefsegment,heading=subbibliography]
