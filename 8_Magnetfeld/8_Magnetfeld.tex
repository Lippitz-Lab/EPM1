\renewcommand{\lastmod}{10. September 2024}
\renewcommand{\chapterauthors}{Markus Lippitz}

\chapter{Atome in externen Feldern}



\goal{By the end of this chapter, you should be able to draw, calculate and align a ray's path through an optical system.}


Ich kann den Einfluss eines externen Magnetfelds auf ein Atom beschreiben und für den Grenzfall schwacher Magnetfelder berechnen.

\section{Overview}


Normaler Zeeman-Effekt

8.1 Landé-g-Faktor 1	*	1 

8.2* Anormaler Zeeman-Effekt 2	***

8.3 Paschen-Back-Effekt 3	*	3 

8.4 Stark-Effekt 4	*	4 

8.5 Kern-Spin-Resonanz (NMR) 5	Anw	5 

Hyperfeinstruktur 


Einstein de Haas: Spin gibt es wirklich

\url{https://phet.colorado.edu/en/simulations/stern-gerlach}




\section{Zusammenfassung}

\textit{Schreiben Sie hier ihre persönliche Zusammenfassung des Kapitels auf. Konzentrieren Sie sich auf die wichtigsten Aspekte.}

\vspace*{10cm}

%--------------------
\printbibliography[segment=\therefsegment,heading=subbibliography]
