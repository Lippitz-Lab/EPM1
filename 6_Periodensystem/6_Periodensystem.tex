\renewcommand{\lastmod}{15. November 2024}
\renewcommand{\chapterauthors}{Markus Lippitz}

\chapter{Die restlichen Atome des Periodensystems}



\section{Überblick}

Nachdem wir uns im letzten Kapitel mit dem Wasserstoffatom beschäftigt haben, wollen wir nun alle anderen Atome des Periodensystems betrachten. Die etwas ungleiche Verteilung der Aufmerksamkeit auf die Elemente rührt daher, dass wir schon bei zwei Elektronen pro Atom mit starken Näherungen beginnen müssen. Die Elektronen stoßen sich gegenseitig ab, was nur mit großem Aufwand modelliert werden kann. Wir betrachten hier also eher allgemeine Prinzipien als konkrete Rechnungen.

Eine zentrale Rolle wird das Pauli-Prinzip spielen: Zwei ununterscheidbare Elektronen können nicht in allen Quantenzahlen übereinstimmen. Es können also nicht alle Elektronen im gleichen Zustand sein. Dies ist ein wesentlicher Beitrag dazu, dass sich die Elemente in ihren chemischen Eigenschaften unterscheiden.

Als Konsequenz des Pauli-Prinzips ergeben sich die Hundschen Regeln, Abkürzungen bei der Suche nach dem Zustand mit der minimalen Energie. Damit ist es möglich, für die meisten Elemente die Quantenzahlen aller Elektronen im Grundzustand des Atoms anzugeben.

Sowohl das Pauli-Prinzip als auch die Hundschen Regeln benutzen die Addition von Drehimpulsen oder drehimpulsähnlichen Größen wie dem Spin. Wir führen einen Gesamtdrehimpuls und einen Bahndrehimpuls ein, wobei die Quantenmechanik manchmal von der rein geometrischen Addition abweicht.

Dieses Kapitel folgt weitgehend dem Kapitel 8 von \cite{Harris_moderne_Physik}. Gut zu lesen ist auch \cite{Demtröder_ep3} und auch \cite{Heintze_WTA}.

\section{Zwei Teilchen in einem Kasten}

Alle Atome mit Ausnahme von Wasserstoff haben mehr als ein Elektron. Wir müssen also unser quantenmechanisches Modell auf zwei und mehr Teilchen erweitern. Dazu kehren wir, \cite{Harris_moderne_Physik} folgend, zunächst zum Teilchen-im-Kasten-Modell zurück, sperren nun aber zwei Teilchen in den Kasten ein. Wie bisher implizit angenommen, brauchen unsere Teilchen keinen Platz, sie schließen sich nicht gegenseitig aus. Jedes Teilchen $i=1,2$ befindet sich an seinem Ort $x_i$ (hier in einer Dimension). Jedes Teilchen besitzt eine kinetische Energie. Und es gibt ein Potential $U(x_1, x_2)$, das den Kasten für jedes Teilchen beschreibt. Später können wir damit auch die Coulomb-Abstoßung von Teilchen modellieren. Aber das brauchen wir jetzt noch nicht. Die Schrödingergleichung lautet
\begin{equation}
    \left(
        - \frac{\hbar^2}{2m} \, \frac{\partial^2}{\partial x_1^2} 
        - \frac{\hbar^2}{2m} \, \frac{\partial^2}{\partial x_2^2} 
     \right)
     \Psi(x_1, x_2)
     \, + \,
     U(x_1, x_2) \, \Psi(x_1, x_2)
     = E \, \Psi(x_1, x_2)
\end{equation}
mit der Wellenfunktion $\Psi(x_1, x_2)$, die nun von zwei Ortskoordinaten abhängt, nämlich den Positionen der beiden Teilchen.

Für einen Topf mit starren Wänden ergibt\sidenote{Rechnung in \cite{Harris_moderne_Physik}} sich
\begin{equation}
    \Psi(x_1, x_2) = \Psi_{n}(x_1) \, \Psi_{n'}(x_2) \quad
    \text{mit} \quad \Psi_n(x) = \sqrt{\frac{2}{L}} \, \sin \frac{n \pi x}{L}
\end{equation}
und ganzzahligen Quantenzahlen $n \ge 1$.

Die alten Zustände, Quantenzahlen, Potentiale und Wellenfunktionen, die wir für ein einzelnes Teilchen gefunden haben, nennen wir \emph{Einteilchen-Zustand} und \emph{Einteilchen-Wellenfunktion}. Die neuen, zusammengesetzten für mehrere Teilchen heißen entsprechend \emph{Mehrteilchen-Wellenfunktion}. $\Psi(x_1, x_2)$ ist eine Mehrteilchen-Wellenfunktion, die als Produkt der Einteilchen-Wellenfunktionen $ \Psi_n(x)$ geschrieben wird.

\subsection{Wahrscheinlichkeitsdichte}

Die Zweiteilchen-Wahrscheinlichkeitsdichte $P(x_a, x_b) = | \Psi(x_a, x_b)|^2$ beschreibt die Wahrscheinlichkeit, Teilchen 1 am Ort $x_a$ und Teilchen 2 am Ort $x_b$ oder in einem Intervall $dx$ um diese Orte herum zu finden.

Betrachten wir ein Beispiel. Die Quantenzahlen seien $n = 4$ und $n' = 3$. Damit ist 
\begin{equation}
    P(x_1, x_2) = |  \Psi(x_1, x_2)|^2 = \frac{4}{L^2} \, \sin^2 \frac{4 \pi x_1}{L} \, \sin^2 \frac{3 \pi x_2}{L} \quad .
\end{equation}
Die Wahrscheinlichkeit, das Teilchen 1 in der Mitte des Kastens bei $x_1 = L/2$ zu finden, ist Null, da $\Psi_4(x)$ dort einen Knoten hat, also 
\begin{equation}
    P\left(\frac{L}{2}, x_2\right) = \frac{4}{L^2} \, \sin^2 \frac{4 \pi (L/2)}{L} \, \sin^2 \frac{3 \pi x_2}{L}  = 0 \quad .
\end{equation}
Diese Wahrscheinlichkeit ist unabhängig von der Position $x_2$ des zweiten Teilchens. Die Wahrscheinlichkeit, das zweite Teilchen in der Mitte des Kastens zu finden, ist jedoch nicht durchgängig Null, da $\Psi_3(x)$ dort einen Bauch hat
\begin{equation}
    P\left(x_1, \frac{L}{2}\right) = \frac{4}{L^2} \, \sin^2 \frac{4 \pi x_1}{L} \, \sin^2 \frac{3 \pi L/2}{L}  \neq 0
\end{equation}
zumindest für manche Werte von $x_1$.


Es kann aber nicht sein, dass Teilchen 1 nicht in der Mitte des Kastens ist, Teilchen 2 aber schon, wenn sie ununterscheidbar sind! Wir können die Quantenteilchen nicht mit Namensschildern versehen. Da der Abstand zwischen den Teilchen beliebig klein sein kann, können wir einem Teilchen auch nicht immer folgen.

Diese Ununterscheidbarkeit der Quantenteilchen ist der zentrale Punkt. Die Wahrscheinlichkeitsdichte und andere beobachtbare Größen dürfen sich nicht ändern, wenn wir die Namen, die Quantenzahlen der ununterscheidbaren Teilchen vertauschen. Man sagt, dass die Wahrscheinlichkeitsdichte unter Vertauschung der Indizes symmetrisch sein muss.

\section{Symmetrische und antisymmetrische Wellenfunktion}

Die Symmetrie der Wahrscheinlichkeitsdichte erreichen wir durch Linearkombination der obigen Zweiteilchen-Wellenfunktion mit ihrer vertauschten Variante:
\begin{align}
    \Psi_S(x_1, x_2) = & \Psi_{n}(x_1) \, \Psi_{n'}(x_2) \, + \, \Psi_{n'}(x_1) \, \Psi_{n}(x_2) & \text{symmetrisch}
    \label{eq:6_sym_WF} \\
    \Psi_A(x_1, x_2) = & \Psi_{n}(x_1) \, \Psi_{n'}(x_2) \, - \, \Psi_{n'}(x_1) \, \Psi_{n}(x_2) & \text{anti-symmetrisch} 
    \label{eq:6_asym_WF} 
\end{align}
Im zweiten Summanden sind die Quantenzahlen gegenüber dem ersten vertauscht und damit auch der Name des Teilchens. Da die Schrödingergleichung linear ist, sind diese Linearkombinationen auch Lösungen der Schrödingergleichung.
Durch Ausmultiplizieren des Betragsquadrats stellt man fest, dass beide Wellenfunktionen zu einer Wahrscheinlichkeitsdichte führen, die sich bei Vertauschung der Indizes nicht ändert, also symmetrisch ist.


\begin{marginfigure}
    \inputtikz{\currfiledir kasten_sym}
    \caption{Wahrscheinlichkeitsdichten zu $ \Psi(x_1, x_2)$,  $ \Psi_S(x_1, x_2)$ und  $ \Psi_A(x_1, x_2)$.}
\end{marginfigure}

Die Abbildung  zeigt die Wahrscheinlichkeitsdichte $ P(x_1, x_2) $ für die erste, noch nicht symmetrisierte Wellenfunktion $ \Psi(x_1, x_2)$ sowie für die symmetrische $ \Psi_S(x_1, x_2)$ und die asymmetrische $ \Psi_A(x_1, x_2)$. Dies ist eine zweidimensionale Darstellung im $x_1$--$x_2$ Raum, aber der Kasten ist eindimensional mit der Achse $x$.


Die Wahrscheinlichkeitsdichten sind bis zu einem gewissen Grad ähnlich. Alle haben 12 Peaks. Bei der nicht symmetrisierten Variante kann man jedoch die Achse $x_1$ von der Achse $x_2$ unterscheiden.
Die symmetrisierten Varianten sind entlang der Linie $x_1 = x_2$ spiegelsymmetrisch. Für $\Psi_S(x_1, x_2)$ liegen die höchsten Peaks entlang dieser $x_1 = x_2$-Linie. Für $\Psi_A(x_1, x_2)$ liegen die höchsten Peaks in den anderen Ecken, d.h. wenn $|x_1 - x_2|$ am größten wird. Zwei Teilchen in der symmetrischen Zweiteilchen-Wellenfunktion neigen dazu, sich am gleichen Ort zu befinden. Zwei Teilchen in der antisymmetrischen Zweiteilchen-Wellenfunktion sind im Mittel sehr weit voneinander entfernt.


\section{Mehr Quantenzahlen und Spin}

Die Wahrscheinlichkeitsdichte muss immer symmetrisch sein, wenn die Teilchen vertauscht werden, unabhängig vom Potential und den zur Beschreibung notwendigen Quantenzahlen. Wir können jedoch die obige Schreibweise beibehalten, indem wir unter dem Index $n$ von $\Psi_n(x)$ die Menge aller benötigten Quantenzahlen verstehen. Im Wasserstoffatom umfasst $n$ also $\{ n, l, m_l, m_s \}$ und $\{ 1, 0, 0, \uparrow \}$ ist der Grundzustand. Insbesondere bezieht sich die Bezeichnung 'symmetrisch' oder 'antisymmetrisch' auf alle Quantenzahlen zusammen. Diese Sorte Symmetrie nennt man Austauschsymmetrie.



\section{Das Pauliprinzip}


Alle Quantenteilchen besitzen die Eigenschaft 'Spin', auch wenn dieser manchmal Null ist. Man unterscheidet zwischen \emph{Fermionen} mit halbzahligem Spin ($1/2$, $3/2$, $5/2$, usw.) und \emph{Bosonen} mit ganzzahligem Spin. Die Tabelle zeigt Beispiele. Bosonen haben eine symmetrische Mehr\-teil\-chen-Wellenfunktion (Gl. \ref{eq:6_sym_WF}) und Fermionen bilden eine antisymmetrische Mehrteilchen-Wellenfunktion (Gl. \ref{eq:6_asym_WF}). Wenn wir Atome mit mehr als einem Elektron beschreiben wollen, müssen wir dafür sorgen, dass die Mehrteilchen-Wellenfunktion anti-symmetrisch ist, denn Elektronen sind Fermionen, da sie den Spin $1/2$ besitzen.

\begin{marginfigure}
    \begin{tabular}{ll}
        Fermionen & \\
        \hline
        Elektron e$^-$ & $1/2$ \\
        Proton p & $1/2$ \\
        Neutron n & $1/2$ \\
        Neutrino $\nu$ & $1/2$ \\
        Omega $\Omega^-$ & $3/2$ \\
    & \\
    Bosonen & \\
    \hline
    Pion $\pi^0$ & $0$ \\
    $\alpha$-Teilchen & $0$ \\
   Photon $\gamma$ & $1$ \\
   Deuteron d & $1$ \\
    Graviton & $2$ \\
    \end{tabular}
    \caption{Beispiele für  Fermionen und Bosonen und deren Spin}
    \label{tab:6_bosonen_fermionen}
\end{marginfigure}

Die weitreichendste Konsequenz ist das nach Wolfgang Pauli benannte Pauli-Prinzip. Wenn zwei Elektronen in allen Quantenzahlen übereinstimmen, also $n = n'$ in unserer obigen Nomenklatur, dann ist die antisymmetrische Zweiteilchen-Wellenfunktion Null:
\begin{equation}
    \Psi_A(x_1, x_2) =  \Psi_{n}(x_1) \, \Psi_{n}(x_2) \, - \, \Psi_{n}(x_1) \, \Psi_{n}(x_2) = 0 \quad ,
\end{equation}
also kann  diesen Fall also nicht geben. Das ist das Ausschlussprinzip von Pauli: Zwei ununterscheidbare Fermionen können nicht im selben Einteilchenzustand sein, d.h. sie können nicht in allen Quantenzahlen übereinstimmen. Dieses Prinzip gilt für jedes Paar von Fermionen, wenn es mehr als zwei gibt.

Das Wort 'ununterscheidbar' ist wichtig. Ein Elektron und ein Proton können in den Quantenzahlen übereinstimmen. Auch zwei Elektronen können in den Quantenzahlen übereinstimmen, wenn sie nur weit genug voneinander entfernt sind, so dass sie nicht unbemerkt die Plätze tauschen können. 

Für Bosonen gilt das Pauli-Prinzip nicht. Das Plus in Gl. \ref{eq:6_sym_WF} stört nicht, wenn $n = n'$.

Eine Folge des Pauli-Prinzips ist, dass sich in einem Atom mit $Z$ Protonen und damit auch $Z$ Elektronen nicht alle Elektronen im Grundzustand $1s$ befinden können. Man füllt die Zustände in aufsteigender Reihenfolge auf, bis alle Elektronen untergebracht sind. Dabei darf kein Zustand doppelt besetzt sein, so dass wir die $Z$ energetisch niedrigsten Zustände besetzen. Die chemischen Eigenschaften eines Elements ergeben sich dann aus den letzten besetzten und den ersten unbesetzten Zuständen des Atoms.


\section{Näherung der unabhängigen Elektronen}

Wie findet man diese energetisch niedrigsten Zustände? Zunächst müssen wir berücksichtigen, dass die Kernladung nun $Z$ ist und nicht mehr eins wie beim Wasserstoffatom. Überall dort, wo im letzten Kapitel ein Term $e^2$ vorkam, müsste man nun $Z \, e^2$ schreiben.\sidenote{Das ist im POtential $U$, den Eigneenergien $E_n$, dem Bohr'schen Radis $a_B$ und der Feinstrukturkonstanten $a$} Damit ist 
\begin{equation}
    E_n \propto Z^2  \quad .
\end{equation}
Das  berücksichtigt aber nur die Ladung im Kern. Die Ladung der Elektronen, die zum Wasserstoff hinzukommen, ist aber auch $Z-1$, also nicht wirklich kleiner.\sidenote{Das ist der Unterschied zu einem Planetensystem. Dort ist   die Masse aller Planeten zusammen viel kleiner als die der Sonne.} Wir können sie nicht vernachlässigen. Eigentlich müssten wir die Schrödingergleichung mit dem Potential
\begin{equation}
    U(r_1, r_2, \dots, r_Z) = -\frac{e^2}{4 \pi \epsilon_0} \, \sum_{i=1}^Z  \left[  \frac{Z}{r_i} - \sum_{j=1}^{i-1}  \frac{1}{|r_i - r_j|} \right]
\end{equation} 
lösen. Das ist näherungsweise möglich, aber nicht einfach und Thema der Vielteilchen-Quantenmechanik. Hier machen wir, wie fast überall, die \emph{Näherung der unabhängigen Elektronen}. Wir nehmen an, dass sich jedes Elektron in einem effektiven Potential bewegt, das aus der Kernladung und einer Wolke der übrigen Elektronen besteht. Diese Elektronenwolke schirmt den Kern etwas ab. Ein sehr weit außen befindliches Elektron erfährt dann quasi nur noch eine einzige positive Restladung, alles andere ist abgeschirmt. Ganz innen sieht das Elektron dann einen Kern der Ladung $Z$. Das abgeschirmte Coulombpotential ist also
\begin{equation}
    U(r) =  -\frac{e^2}{4 \pi \epsilon_0} \, \frac{1}{r} \cdot
    \left\{
 \begin{matrix}
     Z \quad  & \text{für} \quad  r  \rightarrow  0 \\
     1     & \text{für} \quad  r \rightarrow \infty \\
 \end{matrix}
    \right. \quad .
\end{equation}
Alle Abhängigkeiten der Form $|r_i - r_j|$ werden vernachlässigt. Die Abbildung \ref{fig:6_coulomb_schirm}
zeigt ein solches Potential für Lithium. Das 2s-Elektron erfährt außen ein Potential, das dem des Wasserstoffatoms sehr ähnlich ist ($\propto -1/r$). Nach innen geht es jedoch in das von \ch{Li^{2+}} über, also einem dreifach positiv geladenen Kern ($\propto - 3/r$). Dies kann über eine effektive Kernladungszahl $Z_\text{eff}(r)$ modelliert werden.


\begin{marginfigure}
    \inputtikz{\currfiledir coulomb_schirm}
    \caption{Abgeschirmtes Coulombpotential am Beispiel von \ch{Li}. }
    \label{fig:6_coulomb_schirm}
\end{marginfigure}

Wie beim Wasserstoffatom handelt es sich um ein Zentralpotential. Die Winkelabhängigkeit der Lösungen ist daher identisch. Der Radialanteil der Wellenfunktion ändert sich etwas. Die Eigenenergien verschieben sich quadratisch mit steigendem $Z$ zu negativeren Werten, da das Potential viel negativer wird. Gleichzeitig werden aber auch mehr Elektronen eingefüllt, so dass der höchste besetzte Zustand mit steigendem $Z$ weiter vom Grundzustand entfernt ist. Wir werden sehen, dass sich diese beiden Effekte nahezu ausgleichen.


\section{Einfluss des Bahn-Drehimpulses}

Im vorigen Kapitel haben wir gesehen, dass die Wellenfunktionen mit dem maximalen $l$ bei gegebener Hauptquantenzahl $n$, also $l=n-1$, am kreisförmigsten sind und zu einem Maximum in der radialen Aufenthaltswahrscheinlichkeit führen, das gut durch den zugehörigen Bohrschen Bahnradius $n \cdot a_B$ beschrieben wird. Wellenfunktionen mit kleinerem $l$ beschreiben Elektronen mit elliptischeren Bahnen, die sich mit größerer Wahrscheinlichkeit weiter innen befinden. Zustände mit kleinem $l$ sind daher energetisch niedriger als Zustände mit großem $l$. Diese Unterschiede sind größer als die der im letzten Kapitel beschriebenen Spin-Bahn-Kopplung, die ebenfalls die $l$-Entartung im Wasserstoffatom aufhebt.

Berechnet man die Summe der Aufenthaltswahrscheinlichkeiten aller Einteilchen-Elektronenzustände mit gegebenem $n$, so ist diese kugelsymmetrisch und wird von den Elektronen mit maximalem $l$ dominiert, d.h. von denen nahe  der Bohr'schen Bahn. 
Man sagt, dass diese Elektronen mit der gleichen Hauptquantenzahl $n$ eine \emph{Schale} bilden, die manchmal mit den großen Buchstaben K, L, M, N bezeichnet wird.  Aufgrund der gerade beschriebenen Drehimpulsabhängigkeit spaltet sich jede Schale in $n$ Unterschalen auf, die manchmal mit römischen Ziffern bezeichnet werden. In jede Unterschale passen nach dem Pauli-Prinzip $2 (2l +1)$ Elektronen, da es $2l +1$ mögliche Werte der magnetischen Quantenzahl $m_l$ gibt und dann noch je zwei Möglichkeiten für die Orientierung $m_s = \pm 1/2$ des Spins. In jeder Schale befinden sich $2n^2$ Zustände.

Empirisch zeigt sich, dass die energetische Anordnung der Unterschalen und damit die Reihenfolge des Auffüllens der Elektronen mit steigendem $n+l$ erfolgt. Bei Unterschalen mit gleichem $n+l$ werden die mit kleinerem $n$ zuerst aufgefüllt. Dies ist in Abbildung  \ref{fig:6_state_3d}. Man erkennt auch, dass sich die Reihenfolge der Unterschalen mit der Kernladungszahl ändert, unsere einfachen Regeln also nicht alles erfassen.

%XXX mark filled states in fig states-3d ?

\begin{marginfigure}
    \inputtikz{\currfiledir state_3d}
    \caption{Schematische Verschiebung der Zustände mit steigender Kernladungszahl. Zwischen der 4s und 4p-Schale wird bei den Elementen  \ch{Sc} ($Z=21$) bis \ch{Zn} ($Z=30$) die 3d-Schale gefüllt.}
    \label{fig:6_state_3d}
\end{marginfigure}

\section{Das Periodensystem der Elemente}

Die chemischen Eigenschaften der Elemente werden wesentlich durch die Anzahl und die Bindungsenergie der Elektronen in der noch nicht vollständig gefüllten Schale bestimmt. Diese Elektronen werden als Valenzelektronen bezeichnet, da sie die Wertigkeit des Elements bestimmen. Sehr häufig ist dies auch die äußerste, also die energiereichste Schale. Da jede Unterschale nur $2 (2l +1)$ Elektronen aufnehmen kann, sind die chemischen Eigenschaften periodisch in der Kernladungszahl $Z$. Daraus ergibt sich das Periodensystem der Elemente. Abbildung \ref{fig:6:_PSE_states} zeigt die experimentell bestimmte energetische Lage und Besetzung der Zustände der ersten Elemente des Periodensystems. Die Energie besetzter Zustände wird bestimmt, indem ein Elektron durch ein Röntgenphoton aus einem solchen Zustand herausgeschlagen wird (XPS). Die Energie unbesetzter Zustände kann bestimmt werden, indem ein Valenzelektron zu einem solchen Zustand angeregt und die Übergangsenergie bestimmt wird.

\begin{figure}
    \inputtikz{\currfiledir PSE_states}
    \caption{Verlauf der Zustände und deren Besetzung.}
    \label{fig:6_PSE_states}
\end{figure}

\begin{description}
    \item[Wasserstoff (\ch{H})] Das Wasserstoffatom besitzt nur ein Elektron, das sich im 1s-Zustand bei -13,6~eV befindet.
  
  \item[Hellium (\ch{He})] Helium hat zwei Protonen. Die Energie des 1s-Zustandes wäre daher eigentlich um den Faktor $n^2 = 4$ negativer als die des Wasserstoffs, also -54,4~eV. Die beiden Elektronen stoßen sich aber ab, so dass sie im Potential weiter außen liegen, also etwas weniger negativ. Damit liegt der 1s-Zustand bei etwa -24.6 eV. Man schreibt die \emph{Elektronenkonfiguration} als 1s$^2$, wobei die hochgestellte Zahl die Anzahl der Elektronen in der 1s-Schale angibt. Nach dem Pauli-Prinzip haben die beiden Elektronen entgegengesetzten Spin. Helium ist ein Edelgas, da seine voll besetzte 1s-Schale energetisch relativ weit von der Ionisationsgrenze bei $E=0$ entfernt ist.
   
    \item[Lithium (\ch{Li})] Da die 1s-Schale voll besetzt ist, muss das dritte Elektron in die 2s-Schale gehen. Lithium hat also nur ein Valenzelektron, das nur schwach gebunden ist (5.4 eV). Lithium ist daher chemisch sehr reaktionsfreudig.
    
    \item[Kohlenstoff (\ch{C})]   Kohlenstoff hat die Elektronenkonfiguration 1s$^2$2s$^2$2p$^2$ oder abgekürzt [He]2s$^2$2p$^2$. In der 2p-Schale befinden sich also 2 Elektronen. Während bei zwei Elektronen in einer s-Schale der Spin immer entgegengesetzt sein muss, ist dies in einer p-Schale nicht mehr notwendig, es gibt sogar 3 verschiedene Werte für $m_l$. Die \emph{Hund'sche Regel} bestimmt, in welcher Reihenfolge diese insgesamt 6 Zustände der p-Schale besetzt werden: Zunächst haben alle Elektronen die gleiche Spinausrichtung, erst das 4. Elektron (bei Sauerstoff) hat die entgegengesetzte. Wie wir später sehen werden, wird dadurch der Gesamtspin maximiert. Die Hund'sche Regel ist nur eine Abkürzung für eine Energiebetrachtung. Wenn die Elektronen den gleichen Spin haben, sind sie viel weiter voneinander entfernt (siehe Beispiel am Anfang des Kapitels). Dadurch verringert sich die Energie ihrer gegenseitigen Coulomb-Abstoßung.
    
    \item[Fluor (\ch{F})] Beim Fluor befinden sich 7 Elektronen in der 2p-Schale. Für die Edelgaskonfiguration fehlt noch eines. In chemischen Reaktionen nimmt Fluor daher gerne ein weiteres Elektron auf, um die Schale zu vervollständigen.

    \item[Neon (\ch{Ne})]  Bei Neon ist die 2p-Schale vollständig besetzt. Auch dies ist eine der stabilen Edelgaskonfigurationen.
    
    \item[Natrium (\ch{Na})] Bei Natrium wiederholt sich eigentlich alles analog zu Lithium. Auch hier befindet sich nur ein Valenzelektron in einer s-Schale, hier einer 3s-Schale. Hauptgruppe geben bei chemischen Reaktionen gerne Elektronen ab.
\end{description}

Diese Periodizität findet sich in vielen chemischen Eigenschaften wieder. Abbildung \ref{fig:6_E_ionisation}
zeigt sie für die Ionisierungsenergie, d. h. den Abstand des höchsten besetzten Zustandes vom Vakuum. Eine ähnliche Periodizität findet sich auch für das Atomvolumen.

\begin{marginfigure}
    \inputtikz{\currfiledir E_ionisation}
    \caption{Ionisationsenergie der ersten Elemente (Daten aus dem Julia-Paket Mendeleev.jl). Die gefüllten Schalen der Edelgase sind besonders stabil.}
    \label{fig:6_E_ionisation}
\end{marginfigure}


\section{Viele Drehimpulse}


Für das Wasserstoffatom haben wir die Spin-Bahn-Kopplung besprochen, d.h. den Energiebeitrag des Spins des Elektrons im Magnetfeld, der sich aus der Bahn ergibt. Bei Mehrelektronenatomen hat jedes Elektron einen Spin $\bs_i$ und einen Bahndrehimpuls $\bl_l$. Jede Kombination $\bs_i \cdot \bl_j$ liefert einen Energiebeitrag. Das ist zunächst unübersichtlich. Glücklicherweise lassen sich viele Atome sehr gut durch einen der beiden einfachen Grenzfälle beschreiben. Dazu müssen wir zunächst den Gesamtspin und den Gesamtdrehimpuls einführen und auch die Addition von Drehimpulsen in der Quantenmechanik etwas genauer betrachten.


\section{Addition von Drehimpulsen}

Im letzten Kapitel haben wir den Drehimpuls $\bj$ als $\bj = \bl + \bs$ eingeführt und die Eigenschaften von $\bj$ beschrieben. Hilfreich war dabei, dass der Spin eines einzelnen Elektrons nur in zwei Richtungen zeigen kann, nach oben oder nach unten. Der (bald eingeführte) Gesamtspin $\bS$ kann größer sein und damit verschiedenere Richtungen einnehmen. Deshalb müssen wir die Addition von Drehimpulsen etwas genauer betrachten. Ich verwende hier die Variablen $\bJ = \bL + \bS$, also die Großbuchstaben des bald eingeführten Gesamtspins. Das funktioniert aber auch mit zwei beliebigen Drehimpulsvektoren.

Angenommen, wir kennen von zwei Vektoren $\bS$ und $\bL$ die Länge (Quantenzahl $S$ und $L$) und die z-Komponente (Quantenzahl $m_S$ und $m_L$). Was können wir damit über den Vektor $\bJ$ und seine Quantenzahlen $J$ und $m_J$ aussagen? Der Anhang \ref{chap:anhang_drehimpuls}
gibt die Antwort etwas ausführlicher im Formalismus der Quantenmechanik. Hier die Kurzfassung:

Die Längen aller drei Vektoren $\bS$, $\bL$ und $\bJ$ können gleichzeitig ohne Unschärfe gemessen werden, zusammen aber nur die Orientierungsquantenzahl $m_J$ des Summenvektors. Die Orientierungsquantenzahlen $m_S$ und $m_L$ unterliegen nach der Messung von $J$ einer Unschärfe, ähnlich wie im letzten Kapitel die x- und y-Komponente des Bahndrehimpulses $l_x$ und $l_y$.

Die neue Orientierungs-Quantenzahl $m_J$ ist gerade die Summe der Einzeln-Orientierungs-Quantenzahlen
\begin{equation}
 m_J  = m_L + m_S \quad .
\end{equation}
Wenn $m_L$ und $m_S$ bekannt sind, kann $m_J$ berechnet werden. Kennt man aber nur $m_J$, dann ist die Aufteilung in $m_L$ und $m_S$ unbestimmt.  Für die neue Gesamt-Längenquantenzahl $J$ gilt
\begin{equation}
 | L - S | \le J \le  L + S \quad .
\end{equation}
Mehr lässt sich dazu leider nicht sagen. Es ist etwas unbefriedigend, die Summe von zwei Vektoren nicht nennen zu können, obwohl man beide Summanden kennt. Allerdings kennt man die Ausgangs-Vektoren nicht vollständig. Die unbekannte xy-Komponenten sind gerade der Ursprung dieses Spielraums im Wert von $J$.

\subsection{'Gute' Quantenzahlen}

Eine Quantenzahl wird als 'gute' Quantenzahl bezeichnet, wenn sie eine Konstante der Bewegung ist, also erhalten bleibt. Im Fall der Drehimpulsaddition sind dies $L$, $S$, $J$ und $m_J$, aber nicht mehr $m_S$ und $m_L$. Diese waren früher gute Quantenzahlen, bevor wir einen Energiebeitrag proportional zu $\bS \cdot \bL$ erlaubten. Wenn wir das zulassen, dann ist die Aufteilung zwischen $m_S$ und $m_L$§ nicht mehr zeitlich konstant, und diese Quantenzahlen sind nicht mehr 'gut', sie helfen nicht mehr, das System zu beschreiben. Die Gesamtzahl der guten Quantenzahlen bleibt jedoch erhalten. Vor der Kopplung waren es noch $m_S$ und $m_L$, nachher $J$ und $m_J$, jeweils zusätzlich zu $L$ udn $S$. Ebenso muss durch die Kopplung die Anzahl der Zustände, d.h. die Anzahl der möglichen Kombinationen von Quantenzahlen, erhalten bleiben.



\subsection{Geometrische Interpretation}

Bei der Kopplung von Spin und Bahndrehimpuls gibt es einen Energiebeitrag des Spins im Magnetfeld der Bahnbewegung. Klassisch würde dieser vom Winkel zwischen den beiden abhängen. Dieser Winkel ist aber nicht die Quantenzahl, sondern das sich aus $\bS$, $\bL$ und $\bJ$ bildende Dreieck wird vollständig durch die Längen der Seiten bestimmt. Das beinhaltet den Winkel zwischen $\bS$ und $\bL$, aber auch deren Amplitude. Gleichzeitig ist nur $m_J$ eine gute Quantenzahl. Bei einer Messung wie im Stern-Gerlach-Experiment spielt also nur die Orientierung von $\bJ$ eine Rolle. Die Spitze von $\bJ$ kann wieder auf einem Kreis in der xy-Ebene liegen, solange die Länge von $\bJ$ erhalten bleibt. Bei $\bS$ und $\bL$ ist nun aber \emph{nur} die Länge eine gute Quantenzahl, die z-Komponenten nicht mehr. Die Spitze von $\bS$ kann damit auf einem Kreis liegen, dessen Symmetrieachse durch $\bJ$ gegeben ist. Alles andere ist unbekannt, kann nicht gleichzeitig gemessen werden. %Insbesondere ist die Aufteilung zwischen $m_S$ und $m_L$ nicht fix, nur die Summe, also $m_J$.

%XXX skizze Kegel L+S




\section{Gesamtspin}

Nun soll endlich der Gesamtspin $\bS$ eingeführt werden. Er ist die Summe der Spins aller Elektronen eines Atoms
\begin{equation}
    \bS = \sum_{i=1}^Z \, \bs_i \quad . \label{eq:6_S_sum}
\end{equation}
Analog gilt das auch für den Gesamt-Bahndrehimpuls $\bL$. Betrachten wir hier zunächst einmal nur zwei Elektronen $i =1, 2$. Es ist 
\begin{equation}
s_{1,2} = \frac{1}{2} \quad \text{und} \quad m_{s,1,2} = \pm \frac{1}{2}  \quad .
\end{equation}
Welche Werte können nun die Quantenzahlen  $S$ und $m_S$ der Summe annehmen? Die magnetische Quantenzahl $m_S  = m_{s,1} + m_{s,2}$ ist einfach und in nebenstehender Tabelle skizziert.
%
\begin{marginfigure}
\begin{tabular}{r|rr}
                           & $-\frac{1}{2} $  & $+\frac{1}{2} $ \\
                           \hline
 $+\frac{1}{2} $    &     $0$              & $1$ \\
 $-\frac{1}{2} $    &     $-1$              & $0$ 
\end{tabular}
\vspace*{2mm}
\caption{Die möglichen Kombinationen von $m_{s,1}$ und $m_{s,2}$ zu $m_S  = m_{s,1} + m_{s,2}$.}
\end{marginfigure}

Falls $m_{s,1} = m_{s,2}$, also $|m_S| = 1$, dann muss auch $S = 1$ sein, da $S$ nie kleiner als $m_S$ sein kann. Dies sind die Zustände  $(S, m_S) = (1,1) = \uparrow\uparrow$ und  $(S, m_S) = (1,-1) = \downarrow\downarrow$. Die Pfeile zeigen in dieser Darstellung die  $m_{s,i}$-Quantenzahlen als up oder down an.

Damit verbleiben noch die beiden Fälle $m_{s,1} = - m_{s,2}$, also die Diagonale in der Tabelle. Daraus müssen sich die verbleibenden Kombinationen von $S$ und $m_S$ ergeben, nämlich $(S, m_S) = (0,0)$ und $(1,0)$. Die Gesamtzahl der Zustände passt schon einmal. Wie oft in der Quantenmechanik, wenn die Zuordnung nicht einfach entschieden werden kann, werden hier wieder die symmetrische und antisymmetrische Superposition der Ausgangszustände, also der Einträge in der Matrix, gebildet. Welche davon wird  $(S, m_S) = (1,0)$? Die schon gefundenen Zustände $(S, m_S) = (1,\pm1) $ sind symmetrisch bei Vertauschen $1 \leftrightarrow 2$, also wird auch$(S, m_S) = (1,0)$ symmetrisch sein, also 
\begin{equation}
   \frac{1}{\sqrt{2}} \left(  \uparrow \downarrow  +  \downarrow \uparrow \right) \quad .
\end{equation}
Damit gibt es einen anti-symmetrischen Zustand mit $S = 0$, 
\begin{align}
    (S, m_S) = (0, 0)  = & \frac{1}{\sqrt{2}} \left(  \uparrow \downarrow -  \downarrow \uparrow \right) 
\end{align}
und drei symmetrische mit $S=1$
\begin{align}
    (S, m_S) =  (1, +1) =& \uparrow \uparrow  \\
  (1, 0) = & \frac{1}{\sqrt{2}} \left( \uparrow \downarrow +  \downarrow \uparrow \right) \\
  (1, -1) = &\downarrow \downarrow   \quad .
\end{align}
Da es nur einen anti-symmetrischen Zustand gibt, wird dieser als \emph{Singulett} und die anderen als \emph{Triplett} bezeichnet. Generell gibt es, wie immer bei Drehimpulsen, bei einer Quantenzahl $S$ insgesamt $2S+1$ mögliche Werte von $m_S$, also mögliche Zustände.

\subsection{Geometrische Interpretation}

Wie kann man sich vorstellen, dass die Addition von zwei Vektoren gleicher Länge aber unterschiedlicher Orientierungs-Quantenzahl $m_i$ einmal zu einem Vektor der Länge Null und einmal zu einem Vektor der beinahe doppelten Länge führt? Ein Teil der Wahrheit sind die nicht gleichzeitig messbaren anderen Vektor-Komponenten.\sidenote{Ein anderer Teil ist 'so ist die QM eben'.} Die Spitze beider Vektoren liegt auf eine Kreis. Wenn die Position 'in Phase' ist, dann addieren sie sich zu einem Vektor mit verschwindender z-Komponente und der Länge $\hbar \sqrt{2}$, was in diesem Bild dem Zustand $(S, m_S) = (1,0)$ entspricht. Wenn die beiden Ausgangs-Vektoren 'außer Phase' sind, dann addieren sie sich zu Null, ergeben also  $(S, m_S) = (0,0)$. Bei bekannten, aber unterschiedlichen $m_i$, also beispielsweise $\uparrow \downarrow$ ist also nicht eindeutig, welcher Summenvektor sich ergibt. Die Eigenfunktionen des Summen-Operators $\hat{S}$ sind nur Linearkombinationen aus $\uparrow \downarrow$ und $\downarrow \uparrow$.

\begin{marginfigure}
\inputtikz{\currfiledir vector3d_summe}
\vspace*{2mm}

\caption{Die Addition von zwei Vektoren $s=1/2, m_s = 1/2$ und  $s=1/2, m_s = -1/2$ kann sowohl einen Vektor   $S=1, m_S = 0$ ergeben (links) als auch $S=0, m_S = 0$ (rechts).}
\end{marginfigure}

\subsection{Volle Schalen}

In der Gleichung \ref{eq:6_S_sum} hatte ich für den Gesamtspin die Summe über alle Elektronen des Atoms gebildet. Das ist zwar richtig, aber unnötig. Zum Glück ist der Gesamtspin bei vollständig gefüllten Schalen gleich Null. Der Gesamtdrehimpuls ist ebenfalls Null. Das liegt daran, dass die möglichen Werte von $m_S$ und $m_L$ immer symmetrisch um Null sind. Wenn alle beitragen, heben sich die beiden Vorzeichen gegenseitig auf und die Summe ist Null. Daher müssen zur Berechnung des Gesamtspins $\bS$ und des Gesamtbahndrehimpulses $\bL$ nur die Valenzelektonen, also nur die teilweise gefüllten Schalen, betrachtet werden. Die Edelgase haben also im Grundzustand $S=L=0$.


\section{LS-Kopplung und Hund'sche Regeln}

Nun haben wir alle Bausteine beisammen und können diskutieren, wie die vielen Spin- und Bahndrehimpulse eines Mehrelementkerns miteinander wechselwirken. Bei 'leichten' Atomkernen folgt dies meist der LS-Kopplung. Die Hierarchie der Wechselwirkungsenergien ist dabei so, dass zunächst alle Valenzelektronen einen Gesamtspin $\bS$ und einen Gesamtbahndrehimpuls $\bL$ bilden. Die Wechselwirkung zwischen den Spins ist am stärksten. Das Atom versucht, den Gesamtspin zu maximieren, d.h. möglichst viele Elektronen in den gleichen Spinzustand zu bringen, z.B. $\uparrow$. Erst wenn es keine weiteren Zustände mit $m_s = +1/2$ mehr gibt, werden die Zustände mit $m_s = -1/2$ besetzt, wodurch $S$ wieder kleiner wird. Dies ist die \emph{erste Hundsche Regel}, aber eigentlich eine Folge der Columb-Abstoßung der Elektronen. Wenn die Spins gleich sind, ist die Wellenfunktion so, dass die Elektronen sich an Orten befinden, die weiter voneinander entfernt sind.

Der nächste Schritt in der Energiehierarchie ist die Wechselwirkung der Bahndrehimpulse. Die zweite Hundsche Regel fordert ein maximales $L$. Dies ist wiederum auf die Coulomb-Abstoßung der Elektronen zurückzuführen. Wenn sich die $m_l$ der Wellenfunktionen unterscheiden, dann tendieren die Elektronen dazu, weiter voneinander entfernt zu sein, wie z.B. bei den 3D-Isoflächen im letzten Kapitel.

Erst danach folgt die Spin-Bahn-Kopplung mit dem kleinsten Energiebeitrag. Hier orientiert sich der Gesamtspin $\bS$ relativ zum Gesamtbahndrehimpuls $\bL$. Deshalb wird die LS-Kopplung mit Großbuchstaben geschrieben. Die Spin-Bahn-Kopplung bevorzugt ein kleines $J$, wenn die Schale weniger als halb gefüllt ist. Wenn die Schale mehr als halb gefüllt ist, wird ein möglichst großes $J$ bevorzugt. Dies ist die dritte Hund'sche Regel und stammt aus der Energie  proportional zu $\bS \cdot \bL$.

\section{jj-Kopplung}

Der Vollständigkeit halber sei hier noch die jj-Kopplung erwähnt. Sie tritt bei 'schweren' Atomen auf. Mit zunehmender Kernladungszahl $Z$ nimmt das Magnetfeld am Ort des Elektrons zu, das durch die scheinbare Kernbewegung verursacht wird. Damit gewinnt die Spin-Bahn-Kopplung in der Energiehierarchie an Bedeutung. In diesem Fall ist die Kopplung zwischen dem Spin $\bs$ eines einzelnen Elektrons und seinem Bahndrehimpuls $\bl$ der wichtigste Energiebeitrag. Diese beiden koppeln für jedes Elektron zu einem Einzel-Elektron-$\bj$. Erst danach koppeln die vielen $j$ der Elektronen zu einem Gesamt-$J$, um die Abstoßung der Elektronen zu verringern. Deshalb schreibt man jj-Kopplung klein.


\section{Beispiel: Helium}

Helium hat zwei Elektronen, die sich im Grundzustand beide im Ein-Teilchen-Zustand 1s befinden. Der Gesamtdrehimpuls ist $L=0$, da beide $l=0$ sind.

Wie groß ist hier der Gesamtspin $\bS$? Ein Argument ist, dass volle Schalen nichts beitragen, also ist $S=0$. Das andere ist, dass die Gesamtwellenfunktion wegen des Pauli-Prinzips anti-symmetrisch sein muss. Da der Ortsanteil für beide Elektronen gleich ist, nämlich die 1s-Wellenfunktion, muss der Spinanteil zur Antisymmetrie beitragen. Der antisymmetrische Spinzustand ist also der Singulettzustand mit $S=0$.

Das Gesamt-$J$ kann dann nur noch $J  = L + S = 0$ sein.  Analog zu den Ein-Elektron-Zuständen, die als '1s' usw. geschrieben werden, schreibt man die Mehrelektronenzustände als Termsymbol
\begin{equation}
    n^{2S + 1}L_{J}
\end{equation}
wobei für $L$ die gleichen Buchstaben wie für $L$ verwendet werden, jedoch in Großbuchstaben. Man gibt nicht $S$ an, sondern die Multiplizität $2S+1$. Daher auch die Bezeichnungen Singulett (1) und Triplett (3). Der Grundzustand von Helium ist\sidenote{sprich 1-Singulett-S-0} also $1^1S_0$. Einen Triplett-Grundzustand $1^3S_0$ kann es nicht geben.

In den energetisch niedrigsten angeregten Zuständen ändert nur ein Elektron seine Quantenzahl, was auf die sehr große Energielücke von 1s bis 2s zurückzuführen ist. Um zwei Elektronen nach 2s anzuregen, muss wesentlich mehr Energie aufgewendet werden, als um ein Elektron zu ionisieren.

Betrachten wir den Zustand, in dem ein Elektron in 1s bleibt und das zweite nach 2p angeregt ist. Dann muss $L=1$ sein. Da sich die Elektronen nun in zwei verschiedenen Ortswellenfunktionen befinden, können die Spins sowohl symmetrisch (Triplett) als auch antisymmetrisch (Singulett) angeordnet sein, also $S=0$ oder $S=1$. Die Spin-Bahn-Wechselwirkung führt dann zu einem $J$, das in Einerschritten zwischen $|L - S|$ und $L+S$ liegen muss. Die Tabelle zeugt die sich daraus ergebenden Termsymbole.

\begin{marginfigure}
   \begin{tabular}{llll}
    $L$ & $S$ & $J$ & Symbol \\
    1   & 0  &  1 & $2^1P_1$ \\
    1   & 1 &  0 & $2^3P_0$ \\
    1   & 1  &  1 & $2^3P_1$ \\
    1   & 1  &  2 & $2^3P_2$ \\
\end{tabular} 
\vspace*{2mm}

\caption{Mögliche Zustände ausgehend von Einteilchenzustand 1s2p.}
\end{marginfigure}

Wie wir später sehen werden, können optische Übergänge nur zwischen zwei Zuständen gleicher Multiplizität stattfinden. Der Gesamtspin $S$ darf sich dabei nicht ändern. Da der Grundzustand ein Singulett-Zustand ist, können die Zustände $2^3P_J$ noch in den tieferen $2^3S_1$ übergehen, aber nicht weiter. Dieser untere Triplettzustand $2^3S_1$ ist also metastabil, d.h. über einen längeren, aber nicht unendlich langen Zeitraum stabil.


\section{Weitere Beispiele}

Betrachten wir einige weitere Beispiele für die Anwendung der Hundschen Regel und der zugehörigen Termsymbole. Wie wir im Kapitel über die Wechselwirkung mit Licht sehen werden, sind die Symbole so gewählt, dass sie die spektroskopisch relevanten Informationen enthalten, also gut anwendbar sind.

\begin{description}\setlength{\itemsep}{0pt}

    \item[Lithium] Die Elektronenkonfiguration ist [He]2s$^1$. Es gibt nur ein Valenzelektron, also $L=0$ und $S=J=1/2$. Das Termsymbol des Grundzustandes ist daher $2^2S_{1/2}$.
    
    \item[Beryllium] Das hinzukommende Elektron muss den entgegengesetzten Spin haben, um noch in die 2s-Unterschale zu passen. Insgesamt muss dies ein Singulett-Zustand werden, um die Antisymmetrie zu erhalten. Also $L=0$ und $S=J=0$ und somit $2^1S_0$.
    
    \item[Bor] Mit Bor beginnt die 2p-Unterschale. Dieses neue Elektron hat $S=1/2$ und $L=1$. Da die Schale weniger als halb voll ist, wird das niedrigere $J$ bevorzugt, also $J=L-S = 1/2$. Insgesamt ist das Termsymbol $2^2P_{1/2}$.
    
    \item[Kohlenstoff] Kohlenstoff hat zwei Elektronen in 2p. Beide Elektronen haben die gleiche Spinausrichtung, um $S$ auf $S=1$ zu maximieren. Beide Elektronen haben $l=1$, aber die $m_l$ sind unterschiedlich (sonst würde das Pauli-Prinzip verletzt). Die $m_l$ sind so gewählt, dass $L$ maximal wird, d.h. ein Elektron mit $m_l=1$ und eines mit $m_l=0$. Damit ist der Gesamtbahndrehimpuls $L=1$. (Für $L=2$ müssten sich beide Elektronen im Zustand $m_l = 1$ befinden). $J$ wird minimal mit $J=L-S =0$. Das Termsymbol lautet daher $2^3P_0$.
    
    \item[Stickstoff] Jetzt sind 3 Elektronen in 2p. Alle haben die gleiche Spinausrichtung, um $S$ zu maximieren, also $S=3/2$. Die $m_l$ decken den ganzen Bereich $m_l = -1, 0, 1$ ab, so dass die Summe $L= 0$ ist. Damit ist auch $J = S \pm L = 3/2$ und das Termsymbol $2^4S_{3/2}$.
    
    \item[Sauerstoff] Das vierte Elektron muss einen anderen Spin haben, sonst passt es nicht mehr in die 2p-Unterschale. Damit heben sich zwei Spins auf und es bleibt nur $S=1$. Das vierte Elektron liefert aber einen Beitrag zum Bahndrehimpuls, der dann $L=1$ ist. Die Schale ist nun mehr als halb voll, und $J$ will mit $J=L+S=2$ maximal werden. Das Termsymbol ist $2^3P_2$.
    
    \item[Fluor] Ein weiteres Elektron in 2p. Nur ein Spin bleibt ungepaart, $S=1/2$. Das letzte Elektron hat $m_l = 0$, also bleibt $L=1$. Das maximale $J$ ist $J=L+S = 3/2$, zusammen also $2^2P_{3/2}$.
    
    \item[Neon] Das Edelgas Neon ist in dieser Hinsicht langweilig. Die 2p-Schale ist voll besetzt und $S = L = J = 0$ bzw. $2^1S_0$.   
\end{description}

\newpage


\section{Zusammenfassung}

\textit{Schreiben Sie hier ihre persönliche Zusammenfassung des Kapitels auf. Konzentrieren Sie sich auf die wichtigsten Aspekte.}

\vspace*{10cm}


%--------------------
\printbibliography[segment=\therefsegment,heading=subbibliography]
