\renewcommand{\lastmod}{10. September 2024}
\renewcommand{\chapterauthors}{Markus Lippitz}

\chapter{Die restlichen Atome des Periodensystems}



\goal{By the end of this chapter, you should be able to draw, calculate and align a ray's path through an optical system.}

Ich kann Eigenschaften der Elemente und deren Anordnung im Periodensystem mit dem Schalenaufbau der Elektronen erklären. 

Ich kann die Hund‘schen Regeln aus dem Pauli-Prinzip motivieren und anwenden.

Ich kann die Eigenwerte und Eigenfunktionen eines Drehimpuls-Operators bestimmen, der die Summe aus zwei Drehimpuls-Operatoren mit (teilweise) bekannten Eigenwerten bildet.

Ich kann die Wechselwirkung zwischen den verschiedenen magnetischen Momenten eines Atoms (ohne externes Magnetfeld) beschreiben und die sich daraus ergebende Verschiebung der Energien berechnen.



\section{Overview}

s.a. Demtröder 3, Kap. 6


6.1 Das Periodensystem der Elemente 1	* (41.4 Multielectron Atoms; 41.5 The Periodic Table of the Elements )

6.2 Atomare Term-Symbole2	*	2 

6.3* Hund’sche Regeln3	***

6.4 Kopplungsschemata von Drehimpulsen: LS und jj 4	**

6.5 Vollständiges Termschema 5	*	5 

8.4 Feinstruktur-Aufspaltung [2] 4	

8.5 Natrium D Linien [2] 5

Schalenmodell

Rubidium Experiment

% 20. Multielectron atoms
% • We cover this topic very qualitatively and briefly, mentioning spin and the Pauli exclusion
% principle and talking about how quantum mechanics explains the patterns in the periodic
% table, but not doing any calculations. Students have seen this before in chemistry.

\section{Zusammenfassung}

\textit{Schreiben Sie hier ihre persönliche Zusammenfassung des Kapitels auf. Konzentrieren Sie sich auf die wichtigsten Aspekte.}

\vspace*{10cm}


%--------------------
\printbibliography[segment=\therefsegment,heading=subbibliography]
