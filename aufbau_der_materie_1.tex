\documentclass[notoc,nofonts,a4paper,twoside,nobib]{tufte-book}
%\documentclass[nofonts,a4paper,twoside]{book}

\usepackage[german]{babel}
\usepackage{currfile,hyperxmp}

\usepackage{filemod}
   \usepackage{dsfont}
\usepackage[
    type={CC},
    modifier={by-sa},
    version={4.0},
    imagewidth = 17mm,
 ]{doclicense} 
  
 \usepackage[T1]{fontenc}
\usepackage[utf8]{inputenc}

 

\usepackage[refsegment=chapter,style=authoryear-comp,natbib=true,url=true,sorting=nyvt,
isbn=false]{biblatex}

\addbibresource{literature.bib}
  
 
%rm -rf `biber --cache`


%\AtBeginBibliography{\urlstyle{rm}}

\RequirePackage{fontawesome}

\DeclareFieldFormat{doi}{%
  \ifhyperref
    {\href{http://doi.org/#1}{\small \faExternalLink}}
    {\nolinkurl{#1}}}

\DeclareFieldFormat{url}{%
  \ifhyperref
    {\href{#1}{\small \faExternalLink}}
    {\nolinkurl{#1}}}
    
\renewbibmacro*{doi+eprint+url}{%   
  \iftoggle{bbx:url}     
    {\iffieldundef{doi}{\usebibmacro{url+urldate}}{}}     
    {}%   
  \newunit\newblock   
  \iftoggle{bbx:eprint}     
    {\usebibmacro{eprint}}     
    {}%   
  \newunit\newblock   
  \iftoggle{bbx:doi}     
    {\printfield{doi}}     
    {}}  

\usepackage{amssymb,amsmath}
\usepackage{mathtools,bm}
 
\usepackage{modiagram}
\usepackage{chemformula}
\usepackage{chemfig}
\renewcommand*\printatom[1]{\ensuremath{\mathsf{#1}}}


\usepackage{tikz,tikz-3dplot}

\DeclareUnicodeCharacter{0393}{$\Gamma$} 
\DeclareUnicodeCharacter{03C3}{$\sigma$} 


\newcommand{\inputtikz}[1]{%

 \tikzexternalenable
  \tikzsetnextfilename{#1}%
  \input{#1.tikz}%
  \tikzexternaldisable

}


\usetikzlibrary{math,matrix,fit,positioning,intersections}

\usetikzlibrary{calc}
\usetikzlibrary{arrows.meta} %needed tikz library

\usepackage{standalone}
\usepackage{pgfplots}
 \pgfplotsset{compat=newest}
\usepgfplotslibrary{groupplots}
\usepgfplotslibrary{fillbetween}

\tikzset{>=latex}

\usepackage{tikzorbital}
 \usepackage{tikzsymbols}
\usetikzlibrary{quotes,angles}

\usepackage{currfile,hyperxmp}


% \pgfplotsset{
% tufte line/.style={
%     axis line style={draw opacity=0},
%     ytick=\empty,
%     axis x line*=bottom,
%     x axis line style={
%       draw opacity=1,
%       gray,
%       thick
% },
%  %   yticklabel=\pgfmathprintnumber{\tick}
%   }
%   }

% \tikzset{
% mymat/.style={
%     matrix of math nodes,
%     left delimiter=|, right delimiter=|,
%     align=center,
%     column sep=-\pgflinewidth,
% }
% %,mymats/.style={
% %    mymat,
% %    nodes={draw,fill=#1}
% %} 
%  }
 
% \newcommand{\myarrow}[5]{\draw[#4](#1.south -| #2)  -- ++(#3 :6mm) node[above,pos=0.55]{$#5$};
% } 

% \newcommand{\interactLp}[3]{\myarrow{#1-#2-1}{#1.west}{-135}{<-}{#3}} 
% \newcommand{\interactLm}[3]{\myarrow{#1-#2-1}{#1.west}{+135}{->}{#3}} 
% \newcommand{\interactRp}[3]{\myarrow{#1-#2-2}{#1.east}{ -45}{<-}{#3}} 
% \newcommand{\interactRm}[3]{\myarrow{#1-#2-2}{#1.east}{ +45}{->}{#3}}  

% \newcommand{\interactout}[2]{\myarrow{#1-1-1}{#1.west}{+135}{->,dashed}{#2}} 


\newcommand{\benzene}[8]{%
\tikzmath{\x1 = #1; \dx1 = 0.5; \dx2 = 0.9; \ps=0.5;}
\tikzmath{\x2 = \x1 + \dx1 ;}
\tikzmath{\x3 = \x2 + \dx2 ;}
\tikzmath{\x4 = \x3 + \dx1 ;}

\tikzmath{\y1 = #2; \dy = 0.5;}
\tikzmath{\y2 = \y1 + \dy ;}
\tikzmath{\y3 = \y2 + \dy ;}

\orbital[pos = {(\x1,\y2)},scale=#3 * \ps]{pz}
\orbital[pos = {(\x2,\y1)},scale=#4 * \ps]{pz}
\orbital[pos = {(\x3,\y1)},scale=#5 * \ps]{pz}
\orbital[pos = {(\x4,\y2)},scale=#6 * \ps]{pz}
\orbital[pos = {(\x3,\y3)},scale=#7 * \ps]{pz}
\orbital[pos = {(\x2,\y3)},scale=#8 * \ps]{pz}

\draw (\x1,\y2) -- (\x2,\y1) -- (\x3,\y1) -- (\x4,\y2) --(\x3,\y3) 
-- (\x2,\y3) -- (\x1,\y2);
}

%from https://tex.stackexchange.com/questions/464580/drawing-a-perspective-ellipse-with-tikz/464588#464588
% and https://tex.stackexchange.com/a/438695/121799

\makeatletter
%along x axis
\define@key{x sphericalkeys}{radius}{\def\myradius{#1}}
\define@key{x sphericalkeys}{theta}{\def\mytheta{#1}}
\define@key{x sphericalkeys}{phi}{\def\myphi{#1}}
\tikzdeclarecoordinatesystem{x spherical}{% %%%rotation around x
    \setkeys{x sphericalkeys}{#1}%
    \pgfpointxyz{\myradius*cos(\mytheta)}{\myradius*sin(\mytheta)*cos(\myphi)}{\myradius*sin(\mytheta)*sin(\myphi)}}

%along y axis
\define@key{y sphericalkeys}{radius}{\def\myradius{#1}}
\define@key{y sphericalkeys}{theta}{\def\mytheta{#1}}
\define@key{y sphericalkeys}{phi}{\def\myphi{#1}}
\tikzdeclarecoordinatesystem{y spherical}{% %%%rotation around x
    \setkeys{y sphericalkeys}{#1}%
    \pgfpointxyz{\myradius*sin(\mytheta)*cos(\myphi)}{\myradius*cos(\mytheta)}{\myradius*sin(\mytheta)*sin(\myphi)}}

%along z axis
\define@key{z sphericalkeys}{radius}{\def\myradius{#1}}
\define@key{z sphericalkeys}{theta}{\def\mytheta{#1}}
\define@key{z sphericalkeys}{phi}{\def\myphi{#1}}
\tikzdeclarecoordinatesystem{z spherical}{% %%%rotation around x
    \setkeys{z sphericalkeys}{#1}%
    \pgfpointxyz{\myradius*sin(\mytheta)*cos(\myphi)}{\myradius*sin(\mytheta)*sin(\myphi)}{\myradius*cos(\mytheta)}}


\makeatother % https://tex.stackexchange.com/a/438695/121799


\usetikzlibrary{external}
\tikzexternalize[prefix=tikz_external/]




\usepackage{graphicx}
\setkeys{Gin}{width=\linewidth,totalheight=\textheight,keepaspectratio}


\usepackage{booktabs}
\usepackage{url}
\usepackage{hyperref}

%\usepackage{units}

\usepackage{chemformula}

\usepackage{braket}
\setcounter{secnumdepth}{0}

% citations
%\usepackage{natbib}
%\bibliographystyle{plainnat}
%\setcitestyle{round} 

% pandoc syntax highlighting
%\usepackage{color}
%\usepackage{fancyvrb}



% longtable
\usepackage{longtable,booktabs}
\usepackage{multicol}
\usepackage[normalem]{ulem}

% morefloats
\usepackage{morefloats}

\usepackage{calc}
\usepackage{tcolorbox}




%% -- tint overrides
%% fonts, using roboto (condensed) as default
\usepackage[sfdefault,condensed]{roboto}
%% also nice: \usepackage[default]{lato}

%% colored links, setting 'borrowed' from RJournal.sty with 'Thanks, Achim!'
%\RequirePackage{color}
%\definecolor{link}{rgb}{0.1,0.1,0.8} %% blue with some grey
%\hypersetup{
%  colorlinks,%
%  citecolor=link,%
%  filecolor=link,%
%  linkcolor=link,%
%  urlcolor=link
%}

%% macros
\makeatletter

%% -- tint does not use italics or allcaps in title
\renewcommand{\maketitle}{%     
  \newpage
  \global\@topnum\z@% prevent floats from being placed at the top of the page
  \begingroup
    \setlength{\parindent}{0pt}%
    \setlength{\parskip}{4pt}%
    \let\@@title\@empty
    \let\@@author\@empty
    \let\@@date\@empty
    \ifthenelse{\boolean{@tufte@sfsidenotes}}{%
      %\gdef\@@title{\sffamily\LARGE\allcaps{\@title}\par}%
      %\gdef\@@author{\sffamily\Large\allcaps{\@author}\par}%
      %\gdef\@@date{\sffamily\Large\allcaps{\@date}\par}%
      \gdef\@@title{\begingroup\fontseries{b}\selectfont\LARGE{\@title}\par}%
      \gdef\@@author{\begingroup\fontseries{l}\selectfont\Large{\@author}\par}%
      \gdef\@@date{\begingroup\fontseries{l}\selectfont\Large{\@date}\par}%
    }{%
      %\gdef\@@title{\LARGE\itshape\@title\par}%
      %\gdef\@@author{\Large\itshape\@author\par}%
      %\gdef\@@date{\Large\itshape\@date\par}%
      %\gdef\@@title{\begingroup\fontseries{b}\selectfont\LARGE\@title\par\endgroup}%
      %\gdef\@@author{\begingroup\fontseries{l}\selectfont\Large\@author\par\endgroup}%
      %\gdef\@@date{\begingroup\fontseries{l}\selectfont\Large\@date\par\endgroup}%
      \gdef\@@title{\begingroup\fontseries{b}\fontsize{28}{60}\selectfont\@title\par\endgroup}%
      \gdef\@@author{\begingroup\fontseries{l}\fontsize{16}{20}\selectfont\@author\par\endgroup}%
      \gdef\@@date{\begingroup\fontseries{l}\fontsize{16}{20}\selectfont\@date\par\endgroup}%
    }%
    %\phantom{XXX}%
    \vspace{12pc}%
    \@@title%
    \vspace{4pc}%
    \@@author
    \@@date
  \endgroup
  \thispagestyle{plain}% suppress the running head
  \tuftebreak% add some space before the text begins
  \@afterindentfalse\@afterheading% suppress indentation of the next paragraph
}

%% -- tint does not use italics or allcaps in section/subsection/paragraph
\titleformat{\chapter}%
  [display]% shape
  {\relax\ifthenelse{\NOT\boolean{@tufte@symmetric}}{\begin{fullwidth}}{}}% format applied to label+text
  %{\itshape\huge\thechapter}% label
  {\huge \kapitelname \thechapter}% label
  {0pt}% horizontal separation between label and title body
  %{\huge\rmfamily\itshape}% before the title body
  {\fontseries{b}\selectfont\huge}% before the title body
  [\ifthenelse{\NOT\boolean{@tufte@symmetric}}{\end{fullwidth}}{}]% after the title body

\titleformat{\section}%
  [hang]% shape
  %{\normalfont\Large\itshape}% format applied to label+text
  {\fontseries{b}\selectfont\Large}% format applied to label+text
  {\thesection}% label
  {1em}% horizontal separation between label and title body
  {}% before the title body
  []% after the title body

\titleformat{\subsection}%
  [hang]% shape
  %{\normalfont\large\itshape}% format applied to label+text
  {\fontseries{m}\selectfont\large}% format applied to label+text
  {\thesubsection}% label
  {1em}% horizontal separation between label and title body
  {}% before the title body
  []% after the title body

\titleformat{\paragraph}%
  [runin]% shape
  %{\normalfont\itshape}% format applied to label+text
  {\fontseries{l}\selectfont}% format applied to label+text
  {\theparagraph}% label
  {1em}% horizontal separation between label and title body
  {}% before the title body
  []% after the title body

%% -- tint does not use italics here either
% Formatting for main TOC (printed in front matter)
% {section} [left] {above} {before w/label} {before w/o label} {filler + page} [after]
\ifthenelse{\boolean{@tufte@toc}}{%
  \titlecontents{part}% FIXME
    [0em] % distance from left margin
    %{\vspace{1.5\baselineskip}\begin{fullwidth}\LARGE\rmfamily\itshape} % above (global formatting of entry)
    {\vspace{1.5\baselineskip}\begin{fullwidth}\fontseries{m}\selectfont\LARGE} % above (global formatting of entry)
    {\contentslabel{2em}} % before w/label (label = ``II'')
    {} % before w/o label
    {\rmfamily\upshape\qquad\thecontentspage} % filler + page (leaders and page num)
    [\end{fullwidth}] % after
  \titlecontents{chapter}%
    [0em] % distance from left margin
    %{\vspace{1.5\baselineskip}\begin{fullwidth}\LARGE\rmfamily\itshape} % above (global formatting of entry)
    {\vspace{1.5\baselineskip}\begin{fullwidth}\fontseries{m}\selectfont\LARGE} % above (global formatting of entry)
    {\hspace*{0em}\contentslabel{2em}} % before w/label (label = ``2'')
    {\hspace*{0em}} % before w/o label
    %{\rmfamily\upshape\qquad\thecontentspage} % filler + page (leaders and page num)
    {\upshape\qquad\thecontentspage} % filler + page (leaders and page num)
    [\end{fullwidth}] % after
  \titlecontents{section}% FIXME
    [0em] % distance from left margin
    %{\vspace{0\baselineskip}\begin{fullwidth}\Large\rmfamily\itshape} % above (global formatting of entry)
    {\vspace{0\baselineskip}\begin{fullwidth}\fontseries{m}\selectfont\Large} % above (global formatting of entry)
    {\hspace*{2em}\contentslabel{2em}} % before w/label (label = ``2.6'')
    {\hspace*{2em}} % before w/o label
    %{\rmfamily\upshape\qquad\thecontentspage} % filler + page (leaders and page num)
    {\upshape\qquad\thecontentspage} % filler + page (leaders and page num)
    [\end{fullwidth}] % after
  \titlecontents{subsection}% FIXME
    [0em] % distance from left margin
    %{\vspace{0\baselineskip}\begin{fullwidth}\large\rmfamily\itshape} % above (global formatting of entry)
    {\vspace{0\baselineskip}\begin{fullwidth}\fontseries{m}\selectfont\large} % above (global formatting of entry)
    {\hspace*{4em}\contentslabel{4em}} % before w/label (label = ``2.6.1'')
    {\hspace*{4em}} % before w/o label
    %{\rmfamily\upshape\qquad\thecontentspage} % filler + page (leaders and page num)
    {\upshape\qquad\thecontentspage} % filler + page (leaders and page num)
    [\end{fullwidth}] % after
  \titlecontents{paragraph}% FIXME
    [0em] % distance from left margin
    %{\vspace{0\baselineskip}\begin{fullwidth}\normalsize\rmfamily\itshape} % above (global formatting of entry)
    {\vspace{0\baselineskip}\begin{fullwidth}\fontseries{m}\selectfont\normalsize\rmfamily} % above (global formatting of entry)
    {\hspace*{6em}\contentslabel{2em}} % before w/label (label = ``2.6.0.0.1'')
    {\hspace*{6em}} % before w/o label
    %{\rmfamily\upshape\qquad\thecontentspage} % filler + page (leaders and page num)
    {\upshape\qquad\thecontentspage} % filler + page (leaders and page num)
    [\end{fullwidth}] % after
}{}

% tint: no smallcaps in header 
% The 'fancy' page style is the default style for all pages.
\fancyhf{} % clear header and footer fields
\ifthenelse{\boolean{@tufte@twoside}}
%  {\fancyhead[LE]{\thepage\quad\smallcaps{\newlinetospace{\plaintitle}}}%
%    \fancyhead[RO]{\smallcaps{\newlinetospace{\plainauthor}}\quad\thepage}}
%  {\fancyhead[RE,RO]{\smallcaps{\newlinetospace{\plaintitle}}\quad\thepage}}
  {\fancyhead[LE]{\thepage\quad{\newlinetospace{\plaintitle}}}%
    \fancyhead[RO]{{\newlinetospace{\plaintitle}}\quad\thepage}}%
  {\fancyhead[RE,RO]{{\newlinetospace{\plaintitle}}\quad\thepage}}
  



\makeatother




\renewcommand{\chaptermark}[1]{\markboth{#1}{}}%


\ifthenelse{\boolean{@tufte@twoside}}
  {\fancyhead[LE]{\thepage\quad{\newlinetospace{Atome und Moleküle}}}%
    \fancyhead[RO]{{\newlinetospace{\leftmark}}\quad\thepage}}%
  {\fancyhead[RE,RO]{{\newlinetospace{c}}\quad\thepage}}
  
 
%\makeatletter
\fancypagestyle{mystyle}{%
\fancyhf{}%
\fancyfoot[L]{%
\begin{minipage}{17mm}
\doclicenseImage
\end{minipage}
\begin{minipage}{90mm}
 \footnotesize
 \doclicenseLongText
\end{minipage}%
}% 
%\fancyfoot[L]{\doclicenseThis}% 
}
%\makeatother

\usepackage{etoolbox}
\patchcmd{\chapter}{\thispagestyle{plain}}{\thispagestyle{mystyle}}{}{}



\hypersetup{
 linktocpage,
  colorlinks,
  citecolor=Maroon,
  filecolor=Maroon,
  linkcolor=RoyalBlue,
  urlcolor=RoyalBlue
}


\usepackage[theme=default-plain,charsperline=62]{jlcode}
\usepackage{siunitx}

%default, default-plain, grayscale, grayscale-plain and darkbeamer.




 
\newcommand{\kapitelname}{Kapitel\ }
\newcommand{\chapterauthors}{Markus Lippitz}
\newcommand{\lastmod}{\Filemodtoday{\currfilepath}}


\newcommand{\addtochapter}{%
\vspace*{-12mm}{
\setlength{\parindent}{0pt}
\chapterauthors  \newline \lastmod
}
\vspace*{12mm}
}

\makeatletter
\let\stdchapter\chapter
\renewcommand*\chapter{%
  \@ifstar{\starchapter}{\@dblarg\nostarchapter}}
\newcommand*\starchapter[1]{\stdchapter*{#1}}
\def\nostarchapter[#1]#2{\stdchapter[{#1}]{#2} \addtochapter}
\makeatother

\makeatletter
  \def\my@tag@font{\scriptsize}
  \def\maketag@@@#1{\hbox{\m@th\normalfont\color{gray}\my@tag@font#1}}
  \let\amsmath@eqref\eqref
  \renewcommand\eqref[1]{{\let\my@tag@font\relax\amsmath@eqref{#1}}}
\makeatother

\newcounter{questions}[chapter]

\newenvironment{questions}{
\subsection{\normalsize Zum Weiterdenken}
\begin{enumerate} \small
\setcounter{enumi}{\value{questions}}
}{
\setcounter{questions}{\value{enumi}}
\end{enumerate} 
}

\newtcolorbox{zusammen}{%
  breakable,
  enhanced jigsaw,
 % borderline west={1pt}{0pt}{black},
  sharp corners,
  %boxrule=0pt,
  %frame hidden,
  left=1ex,right=1ex,
  fonttitle={\bfseries},
  coltitle={black},
  title={Zusammenfassung:\ },
  attach title to upper}
  
  
  \newcommand{\pluto}[1]{%
  %
  \edef\cfd{\currfiledir}%
  \StrGobbleRight{\cfd}{1}[\mystring]%
  %
  \sidenote{%
  \begin{tikzpicture}
  [baseline={([yshift=-2pt]current bounding box.center)}]
  \definecolor{redline}{RGB}{201,61,57}
  \definecolor{redfill}{RGB}{214,102,97}
  \definecolor{blueline}{RGB}{148,91,176}
  \definecolor{bluefill}{RGB}{170,125,192}
  \definecolor{greenline}{RGB}{59,151,46}
  \definecolor{greenfill}{RGB}{107,171,91}
  \path[draw=redline,fill=redfill,line width=0.8pt] (0,-5.4pt) circle (4.4pt);
  \path[draw=blueline,fill=bluefill,line width=0.8pt] (0,0) circle (4.4pt);
  \path[draw=greenline,fill=greenfill,line width=0.8pt] (0,5.4pt) circle (4.4pt);
  \end{tikzpicture} \ \ 
  \href{https://raw.githubusercontent.com/MarkusLippitz/Festkoerper-II/main/\mystring/pluto/#1.jl}{download}
  \ \ 
  \href{https://binder.plutojl.org/v0.19.12/open?url=https\%253A\%252F\%252Fraw.githubusercontent.com\%252FMarkusLippitz\%252FFestkoerper-II\%252Fmain\%252F\mystring\%252Fpluto\%252F#1.jl}{run on binder}
  }}  
  

  \newcommand{\phet}[1]{%
  \sidenote{\href{http://phet.colorado.edu/simulations/sims.php?sim=#1}{Simulation '\detokenize{#1}'}}
  }



  \newcommand{\bp}{\boldsymbol{p}}
  \newcommand{\bk}{\boldsymbol{k}}
  \newcommand{\br}{\boldsymbol{r}}
  \newcommand{\bj}{\boldsymbol{j}}
  \newcommand{\bx}{\boldsymbol{x}}
  \newcommand{\by}{\boldsymbol{y}}
  \newcommand{\bz}{\boldsymbol{z}}
  \newcommand{\bs}{\boldsymbol{s}}
  \newcommand{\bl}{\boldsymbol{l}}
  % \newcommand{\bD}{\boldsymbol{\mathcal{D}}}
  % \newcommand{\bE}{\boldsymbol{\mathcal{E}}}
  % \newcommand{\bB}{\boldsymbol{\mathcal{B}}}
  % \newcommand{\bH}{\boldsymbol{\mathcal{H}}}
  % \newcommand{\bP}{\boldsymbol{\mathcal{P}}}
  % \newcommand{\bM}{\boldsymbol{\mathcal{M}}}
  % \newcommand{\bS}{\boldsymbol{\mathcal{S}}}  
  
  \newcommand{\bD}{\boldsymbol{D}}
  \newcommand{\bE}{\boldsymbol{E}}
  \newcommand{\bB}{\boldsymbol{B}}
  \newcommand{\bH}{\boldsymbol{H}}
  \newcommand{\bP}{\boldsymbol{P}}
  \newcommand{\bM}{\boldsymbol{M}}
  \newcommand{\bS}{\boldsymbol{S}}
  \newcommand{\bJ}{\boldsymbol{J}}
  \newcommand{\bL}{\boldsymbol{L}}
  \newcommand{\bF}{\boldsymbol{F}}
  \newcommand{\bI}{\boldsymbol{I}}
  \newcommand{\beps}{\boldsymbol{\epsilon}}
  \newcommand{\bmu}{\boldsymbol{\mu}}


  \newcommand{\lit}[1]{#1}
  \newcommand{\ziel}[1]{#1}
  \newcommand{\wort}[1]{\emph{#1}}

  \newcommand{\goal}[1]{{\setlength{\parindent}{0pt}\emph{#1}}}

\usepackage[titletoc]{appendix}
\renewcommand{\appendixname}{Anhang}
\renewcommand{\appendixtocname}{Anhang}
\renewcommand{\appendixpagename}{Anhang}

%\includeonly{1_grenzen/1_grenzen}
%\includeonly{2_quantisierung/2_quantisierung}
%\includeonly{3_wellenfunktionen/3_wellenfunktionen}
%\includeonly{5_H_atom/5_H_atom}
%\includeonly{6_Periodensystem/6_Periodensystem}
\includeonly{7_Licht/7_Licht}


 
\begin{document}
 
  \tikzexternaldisable


\title{Atome und Moleküle}

\author{Markus Lippitz}
\date{\today}


\maketitle


\newpage
\thispagestyle{empty}

% \hfill

% \vfill

% \noindent \textit{cite as}\\
% \noindent Lippitz, Markus, 2023.  \\
% \noindent Festkörperphysik II - Skript zur Vorlesung (Sommer 2023). Zenodo. \\
% \noindent \url{https://doi.org/10.5281/zenodo.8279873}
% %

\tableofcontents

%\renewcommand{\lastmod}{\ \ }
\renewcommand{\chapterauthors}{\ \ }

\chapter*{Vorwort}

Dies ist das Vorlesungsskript meiner Vorlesung 'Aufbau der Materie I'. Sie ist eine Kursvorlesung für  Lehramts-Studierende im dritten Jahr des Bachelorstudiums. Bei der Auswahl und Gewichtung der Themen folgt sie sehr stark dem in Bayreuth Üblichen. 

Neben dem Skript gibt es zu jedem Kapitel  insgesamt circa eine Stunde 'Vorlesung' auf Video\sidenote{\href{https://mms.uni-bayreuth.de/Panopto/Pages/Sessions/List.aspx?folderID=67075f81-052d-4361-8090-b27d00f5e207}{mms.uni-bayreuth.de}}, in der ich mündlich durch den Text führe und dabei an den Rand kritzle.
Ich habe den Eindruck, dass es mir beim Sprechen leichter fällt, die Dinge in einen Zusammenhang zu bringen als beim Schreiben, da ich mich traue, schlampiger zu sein. Zur Vorbereitung gab es dann noch ein online multiple-choice Quiz.  Im Plenum  besprechen wir offene Fragen und diskutierten Aufgaben ähnlich zu Eric Mazurs 
ConcepTests.\sidenote{\href{https://mazur.harvard.edu/research-areas/peer-instruction}{mazur.harvard.edu}}  Schließlich gibt es die in der Physik üblichen Übungszettel und Kleingruppen-Übungen. Manche  Beispiele benutzen Julia\sidenote{\href{https://julialang.org/}{julialang.org}}  und Pluto.\sidenote{\href{https://github.com/fonsp/Pluto.jl}{Pluto.jl}} 



Dieses Skript ist 'work in progress', und wahrscheinlich nie wirklich fertig. Es beruht zum Teil auf meinem Skript zur Molekülphysik\sidenote{\href{https://github.com/Lippitz-Lab/Molekuele-und-Festkoerper}{Molekuele-und-Festkoerper}}. Trotzdem wird es noch Fehler enthalten. Wenn Sie Fehler finden, sagen Sie es mir bitte. 
Die aktuellste Version des Vorlesungsskripts finden Sie auf github.\sidenote{\href{https://github.com/Lippitz-Lab/EPM1}{EPM1}}  Ich habe alles unter eine CC-BY-SA-Lizenz gestellt (siehe Fußzeile). In meinen Worten: Sie können damit machen, was Sie wollen. Wenn Sie Ihre Arbeit der Öffentlichkeit zur Verfügung stellen, erwähnen Sie mich und verwenden Sie eine ähnliche Lizenz. 


Der Text wurde mit der LaTeX-Klasse 'tufte-book' von Bil Kleb, Bill Wood und Kevin Godby\sidenote{\href{https://tufte-latex.github.io/tufte-latex/}{tufte-latex}} gesetzt, die sich der Arbeit von Edward Tufte\sidenote{\href{https://www.edwardtufte.com/}{edwardtufte.com}} annähert. Ich habe viele der Modifikationen angewandt, die von Dirk Eddelbüttel im R-Paket 'tint' eingeführt wurden\sidenote{\href{https://dirk.eddelbuettel.com/code/tint.html}{tint: tint is not Tufte}}. Die Quelle ist vorerst LaTeX, nicht Markdown.




\vspace{2\baselineskip}

Markus Lippitz \\ Bayreuth, 7. Februar 2025

 
 








\renewcommand{\lastmod}{10. September 2024}
\renewcommand{\chapterauthors}{Markus Lippitz}

\chapter{Grenzen der klassischen Physik}



\goal{By the end of this chapter, you should be able to draw, calculate and align a ray's path through an optical system.}



\section{Overview}

s.a. Demtröder 3, Kap. 2


37.1 Matter and Light 1116

37.2 The Emission and Absorption of Light 1116

37.3 Cathode Rays and X Rays 1119

37.4 The Discovery of the Electron 1121

37.5 The Fundamental Unit of Charge 1124

37.6 The Discovery of the Nucleus 1125

37.7 Into the Nucleus 1129

37.8 Classical Physics at the Limit 1131

\phet{Radio_Waves_and_Electromagnetic_Fields}

PLUS UV Katastrophe

2.1 Schwarzkörper und Hohlräume [2]1	



2.4 Wien’sches Verschiebegesetz 4	

2.5 Treibhaus

6.2 Franck-Hertz-Versuch2	hier ??



% 4. Rutherford Scattering
% Sim: Rutherford Scattering
% • In interviews we found that after instruction some students described the plum pudding model
% as cloud of negative charge filled with protons. They’re probably mixing it up with the
% Schrodinger model.


%--------------------
\printbibliography[segment=\therefsegment,heading=subbibliography]

\renewcommand{\lastmod}{10. September 2024}
\renewcommand{\chapterauthors}{Markus Lippitz}

\chapter{Quantisierung}



\goal{By the end of this chapter, you should be able to draw, calculate and align a ray's path through an optical system.}

Ich kann das Planck’sche Strahlungsgesetz herleiten und seinen Zusammenhang mit quantisierter Strahlung erklären. 

Ich kann wichtige Versuche zur Quantisierung (Photoeffekt, Compton-Effekt, Frank-Herz-Versuch, Millikan-Versuch, Spektrallinien) erklären und darstellen, wie diese eine Quantenhypothese unterstützen. 

Ich kann das Modell aus den Bohr’schen Postulaten herleiten, es in relevanten Fällen anwenden und seine Grenzen erklären.

\section{Overview}

s.a. Demtröder 3, Kap. 3


38.1 The Photoelectric Effect 1138


38.2 Einstein's Explanation 1141

38.3 Photons 1144

38.4 Matter Waves and Energy Quantization 1148

38.5 Bohr's Model of Atomic Quantization 1151

38.6 The Bohr Hydrogen Atom 1155

38.7 The Hydrogen Spectrum 1160

PLUS Planck’sches Strahlungsgesetz

\phet{Photoelectric_Effect}
\phet{Quantum_Wave_Interference}
\phet{Neon_Lights_and_Other_Discharge_Lamps}
\phet{DavissonGermer_Electron_Diffraction}
\phet{Fourier_Making_Waves}

\url{https://www.spektrum.de/magazin/100-jahre-quantentheorie/827483}

\url{https://www.spektrum.de/magazin/bedroht-diequantenverschraenkung-einsteins-theorie/1002937}

2.2 Moden eines Hohlraums 2	

2.3 Planck’sches Strahlungsgesetz [2]3	

3.3 Compton-Effekt [2] 3	

3.4 Strahlungsdruck und Impuls des Photons [2]4	

6.2 Franck-Hertz-Versuch2	hier ??


% 2. Photoelectric Effect
% Sim: Photoelectric Effect
% • This is a much harder topic for students than professors think. For details, see:
% www.colorado.edu/physics/EducationIssues/papers/McKagan_etal/photoelectric.pdf
% • Common student difficulties (many can be resolved with sim):
% - think voltage rather than light takes electrons off plate
% - think current increases with speed of electrons
% - can’t explain basic function of experiment
% - can’t explain classical model of light
% - can’t explain why PE experiment leads to photon model of light
% • A general problem that first appears here is that some students have no ability to think
% hypothetically and can’t separate what was expected classically from what really happens.


% 5. Atomic Spectra and Discharge Lamps
% Sim: Discharge Lamps
% • We teach spectra before the Bohr model in order to emphasize how Bohr was able to explain
% the observed spectra with his model.
% • Students often have trouble with the idea that the energy of light corresponds to the difference
% between the levels rather than the values of the levels. They need lots of explicit practice to
% get this distinction straightened out.
% • We get lots of questions about how the electron chooses which level to jump down to, and
% how it decides when to jump down. These questions are useful later for emphasizing why the
% Schrodinger model of the atom is better than the Bohr model.
% • The simulation and associated homework really help students build a clear model of how a
% discharge lamp work. The one place they had trouble was relating this model to what they
% see in a real discharge lamp, even though we did a demo with real discharge lamps and
% diffraction gratings. It’s important to be really explicit in this demo about how the physical
% lamps relate to the model in the sim.
% • When reminding students of Coulomb potential energy, they remember the equation kq1q2/r,
% but often don’t realize that this is the same as –ke²/r.
% • The idea of how fluorescent lights work is harder for students than you might think because
% they have trouble with the idea that red+blue+green light l



% 7. Balmer Series
% • We emphasize the point that Balmer came up with his formula by playing around with
% numbers and didn’t know what it meant. This is probably lost of students who think all of
% physics is like that.


% 8. Bohr and deBroglie Models of the atom
% Sim: Models of The Hydrogen Atom
% • For details about why and how we teach this topic, see:
% www.colorado.edu/physics/EducationIssues/papers/McKagan_etal/BohrModel_McKagan_etal.pdf
% • This is really an opportunity to teach modeling and the significance of Bohr explaining where
% Balmer’s equation came from and deBroglie explaining why there are fixed energy levels.
% This is a really difficult section for students who have trouble thinking hypothetically.
% • In the Bohr model, students often mix up total and potential energy, for example, thinking
% that -13.6eV is the potential energy. This confusion is confounded by the way the total
% energy lines are drawn on top of the potential energy curves.

%--------------------
\printbibliography[segment=\therefsegment,heading=subbibliography]

\renewcommand{\lastmod}{10. September 2024}
\renewcommand{\chapterauthors}{Markus Lippitz}

\chapter{Wellenfunktionen}



\goal{By the end of this chapter, you should be able to draw, calculate and align a ray's path through an optical system.}

Ich kann das Konzept einer Materiewelle benutzen um Experimente mit Teilchen zu erklären, insbesondere deren statistischen Aspekte.


\section{Overview}



\phet{Quantum_Wave_Interference}

\url{https://www.spektrum.de/magazin/komplementaritaet-und-welle-teilchen-dualismus/822095}



% 3. Probability and Randomness and Wave particle duality
% Sim: Quantum Wave Interference
% • When we ask how students visualize light, ~40% have “Bohmian” view of particle traveling
% alongside EM wave.
% • We use sim to demonstrate how the double slit experiment shows that light must be both a
% wave that goes through both slits and a particle that hits the screen at a single location. This
% lecture led to an unexpected onslaught of deep, fundamental questions that took up nearly an
% entire class period. Many students ask whether the particle is actually inside wave with a
% definition location that we just don’t know. Students get pretty frustrated with this class.


% 9. Double slit and Davisson Germer experiment
% Sims: Quantum Wave Interference, Davisson Germer: Electron Diffraction
% • Students have a much harder time thinking of electrons as waves than photons, because
% electrons have mass.
% • Students often think that the size of the wave packet, rather than the wavelength, should
% determine the spacing of the interference pattern.
% • We have noticed that students often miss the point of the Davisson Germer experiment. They
% remember that electrons were only detected at certain angles, but cannot explain why. They
% view the electrons as particles that happen to bounce off at certain angles for some reason
% they can’t understand, rather than recognizing how the observations can be explained by the
% wave nature of electrons. We have found two things that really help to address this:
% - Start with a review of the double slit experiment, a context where students understand
% interference much better, talking about how you would like to do this to test deBroglie’s
% hypothesis, and then explain why this is really hard to do and then talk about how the
% Davisson Germer experiment is analogous.
% - Use the Davisson Germer sim to illustrate how wave interference leads to peaks in
% intensity at certain angles.

% 10. Wave functions and probability
% • When we first introduce wave functions with arbitrary functions, students often don’t
% recognize these as waves because they think “waves” are sine waves.
% • “Wave number” is usually new and unfamiliar to students, and it’s worth spending 5 minutes
% to discuss why we define this quantity and how it relates to wavelength.

% 11. Wave packets and uncertainty principle
% Sims: Quantum Wave Interference, Quantum Tunneling, Fourier: Making Waves
% • We introduce wave packets early because they are much more intuitive than plane waves and
% easier to relate to “particles.”



\section{Überblick}

Wir werden in diesem und dem folgenden Kapitel die Grundlagen der Quantenmechanik untersuchen. Die Gliederung folgt wiederum den Kapiteln 39 und 40 von \cite{Knight_physics}. In diesem Kapitel führen wir das Konzept der Wellenfunktionen ein und im nächsten Kapitel die Schrödingergleichung, mit der man Wellenfunktionen findet. Wenn Sie bereits eine Vorlesung über die Quantenmechanik besucht haben, wird Ihnen vieles bekannt vorkommen. Im restlichen Teil der Vorlesung werden wir die Quantenmechanik benutzen, um Atome und Moleküle zu beschreiben.


\section{Das Doppelspaltexperiment mit Lichtwellen}

Kommen wir noch einmal auf das Doppelspaltexperiment von Young zurück, das Sie bereits in der Optik behandelt haben. Wir behandeln Licht als Welle und nicht als Teilchen. Eine ebene Welle fällt auf zwei (sehr dünne) Spalten in einem ansonsten undurchlässigen Schirm. Man kann sich vorstellen, dass von jedem Punkt in den Spalten eine Huygenssche Elementarwelle $E_{1,2}$ ausgeht, die man schreiben kann als 
\begin{equation}
    E_{i} = a_i \, \sin ( k \, r_i - \omega t) \quad \text{mit}. \quad i = 1,2
\end{equation}
mit der Amplitude $a$ und dem Abstand $r$ vom Spalt. $k = 2 \pi / \lambda $ ist die Länge des Wellenvektors, auch Wellenzahl genannt.
Die Kreisfrequenz ist $\omega = c k$.


Auf dem Schirm überlagern sich die Wellen. Das Superpositionsprinzip besagt, dass das resultierende elektromagnetische Feld genau die Summe der beiden Felder ist, also $E = E_1 + E_2$. In der Optik haben Sie gesehen, dass sich daraus die folgende Amplitude $A(x)$ am Ort $x$ des Schirms ergibt
\begin{equation}
    A(x) = 2 a\cos \left( \frac{\pi d}{\lambda L} \, x \right)
\end{equation}
mit dem Abstand $L$ zwischen Spaltebene und Schirm und dem Abstand $d$ zwischen den beiden Spalten.\sidenote{Oben haben wir angenommen, dass die Spalten sehr dünn sind. Daher geht die Spaltweite nicht ein und die Gleichung wird einfacher.} Die Amplitude $A$ ist maximal ($A = 2a$), wenn sich zwei Wellenberge am Schrim treffen, und minimal ($A = -2 a$), wenn sich zwei Täler treffen.  Dazwischen gibt es Nulldurchgänge. 

Im Experiment wird nicht die Amplitude der Wellen beobachtet, sondern ihre Intensität $I \propto A^2$, d.h.
\begin{equation}
    I(x) = C \, \cos^2 \left( \frac{\pi d}{\lambda L} \, x \right)
\end{equation}
mit einer Proportionalitätskonstanten $C$. Die Nulldurchgänge der Amplitude ergeben somit ein Minimum der Intensität (destruktive Interferenz). Die Maxima der Amplitude ergeben unabhängig vom Vorzeichen ein Maximum der Intensität (konstruktive Interferenz). Nur die Intensität beschreibt also die experimentell beobachtbare Realität.

\section{Wahrscheinlichkeit}

Bevor wir zur Beschreibung durch Photonen kommen, müssen wir kurz über Wahrscheinlichkeiten sprechen. In einem Gedankenexperiment werfen Sie mit verbundenen Augen Dartpfeile auf eine Wand. Jeder Pfeil trifft die Wand, aber nicht immer an der gleichen Stelle. Nach 100 Würfen ergibt sich ein Muster wie in Abbildung XXX. Wo wird der nächste Wurf treffen? Das können wir nicht mit Sicherheit sagen. Aber wir können eine Wahrscheinlichkeit angeben. Bisher sind $N_A =45$ von $N = 100$ Würfen im Bereich A gelandet. Ob von den nächsten 100 Würfen wieder genau 45 Würfe hier landen, ist eher unwahrscheinlich. 100 Würfe sind einfach zu wenig. Deshalb definiert man die Wahrscheinlichkeit $P_A$ als Grenzwert für sehr viele Versuche.
\begin{equation}
    P_A =  \lim_{N \rightarrow \infty} \frac{N_A}{N}
\end{equation}
Für Abbildung XXX können wir also nur $P_A \approx 0.45$ schreiben. 


Die Wahrscheinlichkeiten für sich ausschließende Ergebnisse addieren sich. Die Wahrscheinlichkeit, im Bereich A oder B zu landen, beträgt also
\begin{equation}
    P_{A oder B} = \lim_{N \rightarrow \infty} \frac{N_A + N_B}{N} = P_A + P_B
\end{equation}
und die Summe aller Wahrscheinlichkeiten ist eins, also 
\begin{equation}
    P_A + P_B + P_C = 1
\end{equation}

Der erwartete Wert $N_{erw}$ für die Anzahl der Treffer im Bereich A ergibt sich durch Umformung zu 
\begin{equation}
    N_{A, erw} = P_A \, N
\end{equation}
Unsere beste Vorhersage für die Anzahl der Treffer nach 60 Versuchen ist also $0.45 \cdot 60 = 27 $. Die tatsächlich beobachtete Anzahl der Treffer kann davon abweichen. Je mehr Versuche $N$ durchgeführt werden, desto geringer wird die Abweichung sein.



\section{Interpretation des Interferenzmusters von Photonen }

Im letzten Kapitel haben wir bereits gesehen, welches Punktmuster Photonen auf einem Schirm hinter einem Doppelspalt hinterlassen (siehe Abb. XXX). Dieses wollen wir nun etwas genauer betrachten. Wie ein Foto in der Zeitung besteht das Bild auf dem Schirm aus vielen Punkten, die mehr oder weniger dicht beieinander liegen. Und wie bei den Dartpfeilen können wir nicht sagen, wo das nächste Photon detektiert wird. Aber es gibt Regionen, in denen dies mehr oder weniger wahrscheinlich ist.

Wir definieren entlang der Ortskoordinaten $x$ einen schmalen Streifen der Breite $\delta x$ und Höhe $H$ und zählen alle Photonen, die in diesen Streifen gefallen sind. Diese Zahl nennen wir hier $N(x, \delta x)$. Sie hängt natürlich von der Position $x$ ab, ob es sich um eine hellere oder dunklere Stelle des Sinterferenzmusters handelt. Wir definieren die Wahrscheinlichkeit $WK$ wieder als Grenzwert für eine Gesamtzahl von Photonen $N$.
\begin{equation}
    WK(x, \delta x) = \lim_{N \rightarrow \infty} \, \frac{N(x, \delta x)}{N}
\end{equation}
Von einer guten Theorie erwarten wir, dass diese Wahrscheinlichkeit $WK(x, \delta x)$ vorhersagt, so dass wir die erwartete Anzahl von Photonen in diesem Intervall berechnen können, d.h. 
\begin{equation}
    N(x, \delta x)_{ erw} = WK(x, \delta x) \, N
\end{equation}
Wir werden nicht in der Lage sein, den Auftreffpunkt des nächsten Photons vorherzusagen. Aber wir werden in der Lage sein, die Wahrscheinlichkeit und damit die Anzahl der zu erwartenden Photonen in einem schmalen Streifen zu berechnen.


\section{Photonen- und Wellenbild vereinigen}

XXX Korrespondenzprinzip hier oder später


Wie erhält man die Wahrscheinlichkeit $WK(x, \delta x)$? Hier hilft uns die Energie des Lichts auf dem Schirm. Die Energie, die in dem Streifen mit der Breite $\delta x$ und der Höhe $H$ an der Position $x$ einfällt, sei $E(x, \delta x)$. Für sie gilt 
\begin{equation}
    E(x, \delta x) = I(x) \, \delta x \, H
\end{equation}
wobei angenommen wird, dass $\delta x$ so klein ist, dass sich die Intensität $I(x)$ über die Breite des Streifens nicht wesentlich ändert. Im Photonenbild trägt jedes Photon die Energie $h \nu$ bei. Die Anzahl der Photonen, die pro Sekunde\sidenote{Die Tilde bezeichnet hier Raten (pro Sekunde) im Gegensatz zu Gesamtzahlen.} auf dem Streifen ankommen, ist also 
\begin{equation}
    \tilde{N}(x, \delta x) =\frac{ E(x, \delta x)}{h \nu}
\end{equation}
und damit die Wahrscheinlichkeit
\begin{equation}
    WK(x, \delta x) =  \frac{\tilde{N}(x, \delta x) }{\tilde{N}} =
    \frac{ E(x, \delta x}{h \nu \, \tilde{N}} = 
    \frac{ I(x) H \delta x}{h \nu \, \tilde{N}}    
\end{equation}
Inbesondere ist  $I(x) \propto |A(x)|^2$ und damit
\begin{equation}
    WK(x, \delta x)  \propto |A(x)|^2 \delta x
\end{equation}
Die Wahrscheinlichkeit, ein Photon im Intervall $(x, x+\delta x)$ zu treffen, ist also proportional zum Quadrat der Wellenamplitude an diesem Ort. Diese Gleichung verbindet das Wellenbild mit dem Photonenbild. Sie bildet die Grundlage für die Interpretation der Quantenmechanik.

XXX Korrespondenzprinzip


\subsection{Wahrscheinlichkeitsdichte}

Das Interferenzmuster ändert sich kontinuierlich entlang der Koordinate $x$. Eigentlich möchte man die Ereignisse nicht auf einem Streifen der Breite $\delta x$ zählen, sondern jedem Ort einen Wert zuordnen. Dies ist mit der \emph{Wahrscheinlichkeitsdichte} möglich. Sie ist wie die Massen- oder Ladungsdichte eine Größe, die multipliziert mit einer Länge, einer Fläche oder einem Volumen eine Masse, eine Ladung oder eine Wahrscheinlichkeit ergibt. Um den Unterschied zwischen Länge, Fläche und Volumen deutlich zu machen, schreibt man gerne das Infinitesimale dazu ($dx$, $dA$, $dV$). Damit gilt 
\begin{equation}
    WK(x, \delta x)  = \int_{x}^{x + \delta x} P(x') \, dx' = P(x) \delta x 
\end{equation}
und somit
\begin{equation}
    P(x) dx  = C'  |A(x)|^2 dx
\end{equation}
mit einer Proportionalitätskonstanten $C'$.
Die Wahrscheinlichkeitsdichte $P$ ist im Gegensatz zur Wahrscheinlichkeit $WK(x, \delta x) $ unabhängig von der Streifenbreite. Diese Gleichung gilt für Photonen in jeder Situation, auch wenn wir sie für das Doppelspaltexperiment hergeleitet haben.


\section{Die Wellenfunktion}

Im letzten Kapitel haben wir im Davidson-Germer-Experiment gesehen, dass Elektronen auch Welleneigenschaften besitzen. Elektronen verhalten sich also an einem Doppelspalt wie Photonen. Auch hier können wir am Schirm die Anzahl der eintreffenden Elektronen pro Flächenintervall bestimmen. Auch hier können wir eine Wahrscheinlichkeitsdichte $P(x)$ messen, die das Interferenzmuster im Doppelspaltexperiment beschreibt und damit die Wahrscheinlichkeit, ein Elektron in der Nähe des Ortes $x$ zu finden.

Für Photonen haben wir gerade gesehen, dass die Wahrscheinlichkeitsdichte $P(x)$ mit dem Quadrat der Amplitude $A(x)$ der optischen Wellen zusammenhängt. Diese Form der Beschreibung wollen wir auch für Elektronen übernehmen. Wir nehmen also an, dass es auch hier eine \emph{Wellenfunktion} $\Psi(x)$ gibt, die die Wahrscheinlichkeitsdichte analog zur Amplitude $A(x)$ beschreibt, also
\begin{equation}
    P(x) dx  = |\Psi(x)|^2 dx
\end{equation}
Im Gegensatz zum optischen Fall haben wir hier die Proportionalitätskonstante auf Eins gesetzt, da wir $\Psi(x)$ neu definieren und nicht auf eine aus der Optik bekannte Größe $A(x)$ zurückgreifen müssen.

Diese Gleichung \emph{definiert} die Wellenfunktionen $\Psi(x)$. Das Experiment bestimmt aber nur $|\Psi(x)|^2$, nicht $\Psi(x)$. Insbesondere können wir nichts über das Vorzeichen von $\Psi(x)$ sagen\sidenote{wie wir später sehen werden, kann $\Psi(x)$ auch komplex sein, so dass uns ein unbekannter Faktor $\exp(i\phi)$ bleibt.}

Im Gegensatz zu einer elektromagnetischen Welle schwingt nichts mit der Wellenfunktion $\Psi(x)$.  $\Psi$ verhält sich wie eine Welle, ist aber nicht mit einer Auslenkung von irgendetwas verbunden. Wir können auch nicht $\Psi(x)$ selbst messen, sondern nur $|\Psi(x)|^2$. Es hat sich aber gezeigt, dass man mit der Annahme einer wellenartigen Funktion $\Psi$ den Ausgang von Experimenten sehr gut beschreiben kann. 

Mit der Wellenfunktion haben wir aber erst die Hälfte. Wir brauchen noch Regeln, wie wir die Wellenfunktion in einer gegebenen Situation bestimmen können und wie sich eine gegebene Wellenfunktion mit der Zeit verändert. Dazu dient die Schrödingergleichung, die im nächsten Kapitel behandelt wird.


\subsection{Normierung}

Wie bei den Dartpfeilen muss die Summe aller sich gegenseitig ausschließenden Wahrscheinlichkeiten eins ergeben. Irgendwo muss das Elektron detektiert werden, es kann nicht verschwinden. Für diskrete Bereiche $A$, $B$, $C$ usw. schreibt man dies als
\begin{equation}
    \sum_{j = A, B, C, \dots} P_j = 1
\end{equation}
Mit der Wahrscheinlichkeitsdichte wird die Summe ein Integral, d.h.
\begin{equation}
    \int_{-\infty}^{+\infty} P(x) dx = 1
\end{equation}
Dies hat Konsequenzen für die Wellenfunktion $\Psi(x)$. Da $P(x) = |\Psi(x)|^2$, muss also 
\begin{equation}
    \int_{-\infty}^{+\infty} |\Psi(x)|^2 dx = 1
\end{equation}
Dies wird Normierungsbedingung genannt, oder eine Wellenfunktion ist normiert, wenn sie diese Bedingung erfüllt. Noch einmal: Wir können nur etwas über das Quadrat der Wellenfunktion sagen, aber nichts über die Wellenfunktion selbst, und damit auch nichts über ein Integral der Wellenfunktion selbst.


\section{Wellenpakete}

Welle und Teilchen sind zwei klassische Konzepte, die sich gegenseitig ausschließen. Keines von beiden kann allein das Doppelspaltexperiment beschreiben und den Welle-Teilchen-Dualismus auflösen. Klassische Wellenpakete zeigen jedoch viele Eigenschaften, die wir auch auf der atomaren Skala finden. Sie können deshalb als Modell dienen.

Betrachten wir ein Wellenpaket, wie es in Abbildung XXX skizziert ist. Im Gegensatz zu einer Sinuswelle ist das Wellenpaket räumlich und zeitlich begrenzt. Dadurch ähnelt es einem Teilchen. Gleichzeitig hat ein Wellenpaket aber auch eine Wellenlänge und schwingt wie eine Welle in Raum und Zeit. Aber auch Wellenpakete sind kein ideales Modell. Letztlich beschreibt nur die Quantenmechanik Objekte auf atomarer Skala korrekt.

Überlagern sich zwei Sinuswellen mit den Frequenzen $f_1$ und $f_2$, so entsteht eine Schwebung (engl beating). Die Amplitude der Gesamtwelle ändert sich periodisch mit der Schwebungsfrequenz
\begin{equation}
    f_{beat} = | f_1 - f_2 | = \frac{1}{T_{beat}} \label{eq:3_tbeat}
\end{equation}
Es entsteht also eine Abfolge von Wellenpaketen mit dem Abstand $T_{beat}$. Nennen wir die Differenz der beiden Frequenzen $\Delta f$ und die Länge der Wellenpakete $\Delta t$ (also ihren Abstand $T_{beat}$, also $T_{beat} = \Delta t$), so können wir Gl. \ref{eq:3_tbeat} schreiben als
\begin{equation}
    \Delta t \, \Delta f = 1
\end{equation}
Das ist zunächst trivial, aber es ist ein Vorbote von etwas Größerem. Wenn sich die beiden Frequenzen annähern, werden die Wellenpakete länger.

Um nicht einen Zug von Wellenpaketen, sondern ein einziges Paket zu erhalten, müssen viele Sinuswellen überlagert werden. Zum Zeitpunkt $t=0$ überlagern sich alle konstruktiv (siehe Skizze XXX), zu allen anderen Zeitpunkten jedoch mehr oder weniger destruktiv. Insbesondere gibt es für große Zeiten zu jeder Welle immer eine andere, die diese gerade auslöscht. Der Zusammenhang zwischen den Frequenzkomponenten und dem zeitlichen Verlauf wird durch die Fourier-Transformation hergestellt. Eine Eigenschaft der Fourier-Transformation ist die Pulsdauer-Bandbreiten-Grenze
\begin{equation}
    \Delta t \, \Delta f \approx 1
\end{equation}
Der genaue Wert der Konstanten auf der rechten Seite hängt von der zeitlichen Form des Wellenpakets ab und auch davon, wie genau man die Breiten $\Delta t$ und $\Delta f$ definiert. Auf diese Details soll hier nicht eingegangen werden. Ich verstecke alles in 'ungefähr eins'. Die Bedeutung ist unabhängig von diesen Details: Ein Wellenpaket, das aus der Superposition verschiedener Sinuswellen gebildet wird, kann nicht beliebig kurz sein. Es hat eine minimale Länge $\Delta t$, die sich aus seiner Breite im Frequenzraum $\Delta f$ ergibt. Dies ist schon eine Eigenschaft der Fourier-Transformation und gilt daher auch in der klassischen Physik.


\subsection{Bandbreite}

Die Pulsdauer-Bandbreitenbegrenzung der Fourier-Transformation hat verschiedene technische Konsequenzen. Immer dann, wenn die Breite im Frequenzbereich begrenzt ist, ist auch die zeitliche Dauer nach unten begrenzt. Ein Beispiel hierfür sind kurze Laserpulse. Um einen Laserpuls der Länge $\Delta t = 100 fs = 100 \cdot 10^{-15}$s zu erzeugen, benötigt man ein Spektrum, das mindestens 
\begin{equation}
    \Delta f = \frac{1}{100 fs} = 10 \cdot 10^{12} Hz = 10 THz
\end{equation}
breit ist. Bei einer Zentralwellenlänge von 800~nm entspricht dies einer Breite von etwa 20~nm.


\subsection{Unschärfe}

Die Breiten $\Delta t$ und $\Delta f$ können auch als \emph{Unschärfe} verstanden werden. Bei einem Wellenpaket der Zeitlänge $\Delta t$ können wir nicht mehr eindeutig sagen, wann es am Detektor eintrifft. Ist die Ankunftszeit der Anfang des Pakets oder das Ende oder das Maximum? Ebenso können wir nicht mehr eindeutig sagen, welche Frequenz und damit welche Wellenlänge es hat, da verschiedene Sinuswellen zum Wellenpaket beitragen. Dies ist eine grundsätzliche Eigenschaft von Wellenpaketen und hat nichts mit experimentellen Unzulänglichkeiten zu tun.

Diese beiden Unschärfen sind miteinander verknüpft. Wenn wir ein Wellenpaket erzeugen wollen, das zeitlich sehr gut definiert ist, also eine kleine zeitliche Unschärfe $\Delta t$ hat, dann verlangt die Pulsdauer-Bandbreiten-Grenze, dass die Frequenzunschärfe $\Delta f$ besonders groß ist. Und umgekehrt: Wenn die Frequenz in einem Wellenpaket sehr genau bekannt sein soll, $\Delta f$ also klein sein soll, dann kann man nur sehr ungenau angeben, wann dieses Wellenpaket eintrifft.

Technisch ist 
\begin{equation}
    \Delta t \, \Delta f \approx 1
\end{equation}
eine untere Grenze. Aufgrund von Rauschen oder anderen technischen Faktoren kann die zeitliche Dauer oder die Bandbreite noch größer sein. Dieses Produkt aus Pulsdauer und Bandbreite definiert die Grenze des Wissens, das über ein beliebiges Wellenpaket gewonnen werden kann.


\section{Die Heisenbergsche Unschärfe-Relation}

Wir wollen nun ein Teilchen durch ein Wellenpaket beschreiben. Das Teilchen mit der Masse $m$ bewegt sich mit der Geschwindigkeit $v_x$ entlang der Achse $x$. Seine de Broglie Wellenlänge ist $\lambda = h / p_x$ mit $p_x = m v_x$ der x-Komponente des Impulses. Die Periodendauer des Wellenpakets sei $\Delta t$. Dann ist die räumliche Ausdehnung $\Delta x$.
\begin{equation}
    \Delta x = v_x \Delta t = \frac{p_x}{m} \, \Delta t
\end{equation}

Die Frequenz $\nu$ der Materiewelle des Teilchens ergibt sich aus dessen de Broglie-Wellenlänge via\sidenote{hier ist Phasen- gleich Gruppengeschwindigkeit}
\begin{equation}
    \nu \lambda = v
\end{equation}
also 
\begin{equation}
    \nu = \frac{v}{\lambda } = \frac{p_x / m}{h / p_x} = \frac{p_x^2}{h m}
\end{equation}
Durch Ableitung finden wir
\begin{equation}
    \Delta \nu = \frac{2 p_x}{h m} \, \Delta p_x
\end{equation}
Wenn wir all dies in die Gleichung für Pulsdauer und Bandbreite einsetzen (und die Frequenzen auf der atomaren Skala als $\nu$ schreiben), dann erhalten wir
\begin{equation}
    \Delta t \Delta \nu = \frac{m}{p_x} \Delta x \, \, \frac{2 p_x}{h m} \, \Delta p_x = \frac{2}{h} \, \Delta x \, \Delta p_x  
\end{equation}
Die Pulsdauer-Bandbreiten-Grenze ist ja eine untere Grenze. Darum schreibe ich das $\approx$ jetzt als $\ge$ und wir erhalten
\begin{equation}
    \Delta x \, \Delta p_x  \ge  \frac{h}{2}
\end{equation}
Dies ist die \emph{Heisenbergsche Unschärferelation}, in dieser Form auch Orts-Impuls-Unschärfe genannt. Wie oben hängt die rechte Seite der Gleichung von der genauen Definition der beiden Breiten $\Delta x$ und $\Delta p_x$ ab. Aber auch hier sind die Details nicht so wichtig, solange ein $h$ auftaucht.

\paragraph*{Was bedeutet das?} Die Heisenbergsche Unschärferelation beschreibt die Grenze unseres Wissens über ein quantenmechanisches Teilchen. Wir können nicht gleichzeitig den Ort und den Impuls (oder die Geschwindigkeit) beliebig genau kennen. Je genauer wir den Ort kennen wollen, desto ungenauer muss unser Wissen über den Impuls sein und umgekehrt.

Messunsicherheiten sind ein Charakteristikum der klassischen Physik. Jede Messung ist mit einer Unsicherheit behaftet. In der klassischen Physik ist dies jedoch ein technisches Problem. Mit etwas mehr Aufwand könnte man die Unsicherheit reduzieren. Außerdem sind zwei Messungen, z.B. Ort und Geschwindigkeit, voneinander unabhängig, und wir könnten beide Unsicherheiten vernachlässigen.

Die Heisenbergsche Unschärferelation besagt, dass dies in der atomaren Welt nicht der Fall ist. Es handelt sich nicht mehr um ein technisches Problem, sondern um eine grundsätzliche Grenze unseres Wissens. Wie auch immer wir es technisch machen, Ort und Impuls können nicht beliebig genau bestimmt werden. In dieser Hinsicht sind Atomteilchen Wellenpaketen ähnlich.

\paragraph*{Nebenbemerkung} Wenn Sie eine Vorlesung über Quantenmechanik besucht haben, werden Sie gesehen haben, dass die Heisenbergsche Unschärferelation mit den Operatoren $\hat{A}$, $\hat{B}$ geschrieben wird als\footcite{Nolting-QM} 
\begin{equation}
    \Delta A \, \Delta B \ge \frac{1}{2} \left|   \braket{ [\hat{A}, \hat{B}]  } \right|
\end{equation}
wobei die eckige Klammer $[ \cdots]$ den Kommutator darstellt\sidenote{$[\hat{A}, \hat{B}] = \hat{A} \hat{B} - \hat{B} \hat{A}$} und die spitze Klammer $\braket{\cdots}$ den Erwartungswert bezeichnet. Auf diese Weise erhält man die oben hergeleitet Orts-Impuls-Unschärfe.

Die Pulsdauer-Bandbreiten-Unschärfe, oder durch Multiplikation mit $h$ die Energie-Zeit-Unschärfe
\begin{equation}
    \Delta E \, \Delta t \ge h
\end{equation}
ist aber eigentlich keine Heisenbergsche Unschärferelation im engeren Sinne, weil es in der Quantenmechanik keinen Zeitoperator gibt. Sie ist 'nur' eine Eigenschaft der Fourier-Transformation. Diese Unterscheidung wird aber nicht oft geamcht.



%--------------------


\printbibliography[segment=\therefsegment,heading=subbibliography]

\renewcommand{\lastmod}{28. Oktober 2024}
\renewcommand{\chapterauthors}{Markus Lippitz}

\chapter{Beispiele aus der (1d) Quantenmechanik}




% 12. Wave equations and Differential equations
% • It is worth emphasizing that the way we solve differential equations in physics, by just
% “guessing” the solution, is completely different from the way they have been taught to solve
% differential equations in math classes. Many students make this section way too hard by
% attempting to use complicated methods they have learned in math classes.

% 13. Schrodinger equation for free particle
% Sim: Quantum Tunneling
% • You can use the Quantum Tunneling sim to demonstrate free particles by just setting the
% potential to “constant.”
% • Students often have difficulty understanding the meaning of complex wave functions. This
% can perhaps best be illustrated by the observation that students frequently ask, “What is the
% physical meaning of the imaginary part of the wave function?” but never ask about the
% physical meaning of the real part, even though both have the same physical significance


% 14. Potential Energy
% • We have found that without explicit instruction on how to relate potential energy diagrams to
% physical systems, most students don’t know what a potential energy diagram means or how it
% relates to anything real. When we do give explicit instruction on this, students start asking a
% lot of questions, and it becomes clear what a struggle it is for them to make sense of it.
% • The fact that we often use the symbol V for potential energy and use the words “potential”
% and “potential energy” interchangeably leads to a lot of students thinking that V is actually
% the electric potential. It is worth emphasizing repeatedly that this is NOT what it means.

% 15. Infinite and Finite Square Wells
% Sim: Quantum Bound States
% • We illustrate a finite square well with the physical example of an electron in a short wire, and
% illustrate an infinite square well with the same system with a really big work function. We
% justify why this potential energy represents this system by building it up from a microscopic
% model of the atoms in the wire. Before we did this, we found that students often mixed up
% wells and barriers. Afterwards, this happened much less.
% • The practice of drawing potential energy, total energy, and wave function on the same graph
% leads students to confuse these quantities. We have several clicker questions to elicit and
% address this confusion.

% 16. Quantum Tunneling, Alpha decay and other applications of Tunneling
% Sim: Quantum Tunneling
% • Language such as “potential well,” “step,” and “barrier” often leads students to interpret
% potential energy diagrams as physical objects. It is worth pointing out that these words are
% only analogies, and using examples such as an electron tunneling through an air gap where
% there is clearly no physical barrier.
% • There is a lot of research showing that students often believe that energy is lost in tunneling,
% and we have incorporated a tutorial and homework designed to address this belief. Two main
% reasons that students think this are that they mix up energy and wave function and interpret
% the exponential decay of the wave function as energy loss, or that they think of a classical
% object penetrating a physical barrier, in which case there is always dissipation. To address the
% first reason, we avoid drawing the wave function and the energy on the same graph, and ask
% several clicker questions designed to elicit and address this confusion. To address the second
% reason, we emphasize that there is no dissipation in the Schrodinger equation.
% • While plane waves are mathematically simple, conceptually it is quite difficult to imagine a
% wave that extends forever in space and time, especially when it is tunneling. The language we
% use to describe tunneling is time-dependent. For example, we say that a particle approaches a
% barrier from the left, and then part of it is transmitted and part of it is reflected. This language
% is difficult to reconcile with a picture of a particle that simultaneously incident, transmitted,
% and reflected, for all time. We find that it works much better to start instruction with wave
% packets, using a qualitative description and the Quantum Tunneling sim, and then show how
% plane waves make the math easier.
% • Determining the potential energy function for a physical example such as an STM or α decay
% actually requires understanding many steps and approximations, and is not trivial for students.
% If you simply present students with these potential energy functions, they usually don’t know
% how to relate them to the physical systems they are supposed to represent. We have many
% clicker questions designed to help students build a model for the potential energy functions
% for STMs and α decay.

% 17. Reflection and Transmission
% Sim: Quantum Tunneling
% • There is a lot of math here and it’s very easy to get lost in the math and forget why you’re
% doing it. When asking students to work through it in homework, it’s very useful to ask them
% to stop after each step in the math and explain in words what they just did.

% 18. Superposition, measurement, and expectation values
% Sim: Quantum Bound States
% • Modern Physics textbooks typically do not cover superposition and measurement.
% We do, because it seems to us that if you don’t talk about measurement, you don’t know what
% you’re actually doing in QM or how it relates to the real world.
% • We have chosen to cover expectation values in some semesters and not others. It helps to
% relate it to more familiar examples such as grade distributions and gambling.



\section{Überblick}

In diesem Kapitel betreiben wir Quantenmechanik. Sie werden lernen, wie die Schrödingergleichung verwendet werden kann, um die von Bohr postulierten stationären Zustände zu finden.

Wir besprechen die zentralen eindimensionalen Beispiele: Teilchen im Kasten und im Potentialtopf, harmonischer Oszillator und quantenmechanisches Tunneln. Die Vorgehensweise ist eigentlich immer gleich: Wir modellieren das System durch sein Potential $U(x)$ und suchen nach Lösungen der Schrödingergleichung. Wir verzichten hier auf Berechnungen, sondern versuchen zu raten bzw. die Formen plausibel zu machen. Viele Eigenschaften der Wellenfunktion können erkannt werden, ohne die Differentialgleichung wirklich zu lösen.

Als übergreifende Eigenschaft werden wir finden, dass die Wellenfunktion in den klassisch zugänglichen Bereichen des Potentials oszilliert und die Anzahl der Bäche mit der Quantenzahl zunimmt. In den klassisch verbotenen Bereichen fällt die Wellenfunktion exponentiell ab, ist aber nicht notwendigerweise sofort Null. Dies führt dazu, dass ein quantenmechanisches Teilchen durch eine dünne Barriere 'tunneln' kann, also durch die Barriere hindurchgeht und nicht darüber. Das klingt zunächst seltsam, ist aber auch bei Lichtwellen der Fall. Der Grund dafür liegt im \emph{Korrespondenzprinzip}, das die Aussagen der Quantenmechanik über mikroskopische Phänomene mit der makroskopischen Welt verbindet.

Die Gliederung folgt wiederum  Kapitel 40 von \cite{Knight_physics}. Weiterhin finden sich gute andere Darstellungen in \cite{Haliday_Resnick}, \cite{Demtröder_ep3}, \cite{Haken_wolf_I} und \cite{Harris_moderne_Physik}.




\section{Die Schrödinger-Gleichung}

Die Schrödingergleichung wurde 1925 von Erwin Schrödinger aufgestellt. Sie beschreibt, welche Wellenfunktionen $\Psi$ für ein quantenmechanisches System zulässig sind. Sie ist ähnlich fundamental wie die Newtonschen Gesetze und kann wie diese nicht hergeleitet, sondern nur plausibel gemacht werden.

Für ein Teilchen der Masse $m$, das sich nur in der Raumrichtung $x$ bewegt und dabei die potentielle Energie $U(x)$ besitzt, lautet die Schrödingergleichung
\begin{equation}
   - \frac{\hbar^2}{2m} \frac{d^2}{dx^2} \Psi(x) + \left[ U(x) - E \right] \Psi(x) = 0 \quad .
   \label{eq:4_SG_1d}
 \end{equation}
 Es handelt sich also um eine Differentialgleichung zweiter Ordnung, wie man es von einer Wellengleichung erwarten würde.


\paragraph*{Nebenbemerkung} In einer allgemeineren Form wird die Schrödingergleichung oft  geschrieben als
\begin{equation}
    \hat{H} \, \Psi = E \, \Psi \quad .
    \label{eq:4_SG_op_1d}
\end{equation}
Dabei nennt man $\hat{H}$ den Hamiltonoperator. Ein Operator ist eine Rechenvorschrift, die auf eine Funktion (hier $\Psi$) wirkt und wieder eine Funktion liefert. Im obigen Fall ist der Hamilton-Operator also
\begin{equation}
    \hat{H} = - \frac{\hbar}{2m} \frac{d^2}{dx^2} + U(x) \quad .
\end{equation}
Es gibt die zweite Ableitung nach dem Ort, aber es fehlt noch, was abgeleitet werden soll; deshalb Operator und nicht einfach Konstante oder Funktion. 
Eine Gleichung der Form Gl. \ref{eq:4_SG_op_1d} heißt Eigenwertgleichung und $E$ heißt \emph{Eigenwert}, weil der Operator wieder die gleiche Funktion $\Psi(x)$ ergibt, nur multipliziert mit einem skalaren Wert $E$.

\section{Plausibilitätsbetrachtung}

Nach de Broglie hat ein Teilchen mit der Masse $m$ und der Geschwindigkeit $v$ eine de Broglie-Wellenlänge 
\begin{equation}
    \lambda = \frac{h}{p} = \frac{h}{ m v} \quad .
\end{equation}
Seine Wellenfunktion oszilliert daher mit dieser Wellenlänge, z. B. sinusförmig
\begin{equation}
    \Psi(x) = \Psi_0 \, \sin\left( \frac{2 \pi x}{\lambda} \right) \quad .
    \label{eq:4_sg_plausi_psi}
\end{equation}

Lassen Sie uns überprüffen, ob dieses $\Psi(x)$ die Schrödingergleichung löst.
Durch die zweimalige Ableitung nach dem Ort wird aus dem Sinus wieder ein Sinus mit einem Vorfaktor:
\begin{equation}
    \frac{d^2}{dx^2} \Psi(x)  = - \left(\frac{2 \pi}{\lambda} \right)^2 \, \Psi(x) \quad .
\end{equation}
Eingesetzt in Gl.~\ref{eq:4_SG_1d} ergibt dies
\begin{equation}
     \frac{\hbar}{2m}  \left(\frac{2 \pi}{\lambda} \right)^2 \,  \Psi(x) + \left[ U(x) - E \right] \Psi(x) \overset{?}{=} 0 \quad .
     \label{eq:4_Sg_plausi}
\end{equation}
Unser Teilchen ist hier ein freies Teilchen. Es wirken keine äußeren Kräfte. Das Potential ist $U(x) = 0$. Die Energie $E$ steckt vollständig in der kinetischen Energie $K$ , also 
\begin{equation}
    E =  K = \frac{1}{2} \, m \,  v^2 =  \frac{1}{2m} \, \left( \frac{h}{\lambda} \right)^2 \quad .
    \label{eq:4_Ekin_lambda}
\end{equation}
Mit $\hbar = h / (2 \pi)$ ist Gl.~\ref{eq:4_Sg_plausi} für alle Werte von $E$ bzw. der de Broglie Wellenlänge $\lambda$ erfüllt.

Hätten wir nicht von vornherein  Gl.~\ref{eq:4_sg_plausi_psi} vermutet, so hätten wir diese Wellenfunktion mit Hilfe der Schrödinger-Gleichung finden können.\sidenote{Häufig löst man eine Differentialgleichung, indem man eine Lösung 'errät' und dann testet.}

Was geschieht, wenn das Teilchen nicht frei ist? Die Kräfte, die dann auf das Teilchen wirken, werden durch das Potential $U(x)$ beschrieben. Die Gesamtenergie $E$ bleibt erhalten, so dass für die kinetische Energie $K(x)$ gilt
\begin{equation}
    K(x) = E - U(x) \quad .
\end{equation}
Das ist genau der Term in der Klammer der Schrödingergleichung Gl \ref{eq:4_SG_1d}! Über Gl. \ref{eq:4_Ekin_lambda} gibt es einen Zusammenhang zwischen der kinetischen Energie und der Broglie-Wellenlänge. Wenn das Potential räumlich nicht konstant ist, ändert sich die de Broglie-Wellenlänge der Wellenfunktion. Wenn weniger Energie für die kinetische Energie 'übrig' bleibt, wird die Wellenlänge größer bzw. die Geschwindigkeit kleiner. Abbildung  \ref{fig:4_potential_slope_WF} zeigt ein Beispiel.

\begin{marginfigure}
    \inputtikz{\currfiledir potential_slope_WF}
    \caption{Wenn das Potentail $U(c)$ röumlich variiert, dann ändert sich die Wellenlänge der Wellenfunktion.}
    \label{fig:4_potential_slope_WF}
\end{marginfigure}




\section{Modelle in der Quantenmechanik}

Auch in der klassischen Physik haben wir Modelle betrachtet und vereinfachende Annahmen gemacht, z.B. dass keine Reibung wirkt oder dass die Masse in einem Punkt konzentriert ist. Die ganze Kunst der Physik besteht darin, gerade so viele Annahmen zu machen, dass das Wesen der Situation erhalten bleibt und man sie noch einfach beschreiben kann.

In der Quantenmechanik machen wir ähnliche Annahmen. Um die zugrundeliegenden Prinzipien klar zu sehen, müssen wir einige Dinge vernachlässigen. Es gibt zwei wesentliche Unterschiede zwischen klassischen und quantenmechanischen Modellen. Die Schrödinger-Gleichung benutzt das Potential $U$ und nicht die Kraft. Wir modellieren also Potentiale und nicht Kräfte, wie wir es in der klassischen Physik sehr oft getan haben. Und das Ergebnis unseres Modells, die Wellenfunktion $\Psi$, ist selbst nicht direkt beobachtbar. Wir können $\Psi(x)$ nicht messen, und die Wahrscheinlichkeitsdichte $|\Psi(x)|^2$ wird im Labor nur selten gemessen. Wir müssen das Modell benutzen, um Aussagen über beobachtbare Größen wie Ladungen und Ströme, Absorptionslinien und Übergangsraten zu machen.


\section{Lösen der Schrödingergleichung}

Wenn wir nun ein Modell aufgestellt haben, d.h. wenn wir uns für einen räumlichen Verlauf des Potentials $U(x)$ entschieden haben, dann müssen wir nur noch die Schrödinger-Gleichung lösen. Das ist die Rechenvorschrift, die zu den von Niels Bohr postulierten stabilen Zuständen führt.

Die Schrödinger-Gleichung ist eine Differentialgleichung zweiter Ordnung. Die Mathematik kennt analytische und numerische Methoden zu ihrer Lösung. Diese sollen hier aber nicht behandelt werden. Wir benutzen den dritten Weg, wie wir ihn oben beschritten haben: Wir 'erraten' eine Lösung, indem wir uns einfach davon überzeugen, dass eine Funktion die Gleichung erfüllt. Das ist möglich, weil andere die ersten beiden Wege gegangen sind und die interessanten Formen des Potentials und damit die möglichen Formen der Lösung begrenzt sind.

An die Wellenfunktion $\Psi(x)$ können einige physikalische Anforderungen gestellt werden. Nicht alle mathematisch möglichen Lösungen der Gleichung sind auch physikalisch relevant. Dies war schon in der klassischen Physik der Fall, als wir z.B. forderten, dass Massen positiv sind, auch wenn eine Gleichung für $m<0$ ebenfalls gelöst wurde. In der Quantenmechanik ergeben sich die Anforderungen an $\Psi$ aus der Interpretation von $|\Psi(x)|^2$ als Wahrscheinlichkeitsdichte: $\Psi$ darf nicht so 'komisch' sein, dass $|\Psi(x)|^2$ keine Wahrscheinlichkeitsdichte mehr sein kann. Daraus folgende diese \emph{Randbedingungen}
\begin{enumerate} \setlength{\itemsep}{0pt}
    \item $\Psi(x)$ muss stetig sein
    \item $\Psi(x)$ muss gleich Null sein an Orten, an denen das Teilchen physikalisch nicht sein kann
    \item  $\Psi(x)$ strebt gegen Null für $x$ gegen $\pm \infty$
    \item $\Psi(x)$ muss normiert sein
\end{enumerate}


Eine sehr nützliche Eigenschaft der Schrödingergleichung ist ihre Linearität. Wenn $\Psi_1$ und $\Psi_2$ Lösungen der Schrödingergleichung für die gleiche Energie $E$ sind, dann sind es auch die Linearkombinationen der beiden.
\begin{equation}
    \Phi = a_1 \, \Psi_1 + a_2 \, \Psi_2 \quad .
\end{equation}
Die Normierungsanforderung legt dann die $a_i$ teilweise  fest.


\subsection{Quantisierung} 

Wir haben die Quantisierung der Energie und anderer Größen als zentrales Element der Quantenmechanik angesprochen. In der Schrödingergleichung scheint auf den ersten Blick nichts quantisiert zu sein. Sobald aber das Potential $U(x)$ die Bewegung des Teilchens einschränkt, gibt es nicht mehr für jede Energie $E$ eine Lösung. Für die allermeisten Werte von $E$ ist die Differentialgleichung unlösbar. Es gibt dann nur wenige diskrete Lösungen mit bestimmten, quantisierten Werten $E_n$.


\section{Beispiel: Teilchen im Kasten}

Wir hatten das Teilchen im Kasten bereits im Kapitel über die Quantisierung behandelt. Damals war es aber nur eine Hypothese, dass die de Broglie Wellenlänge zu stehenden Wellen im Kasten führen muss. Jetzt können wir es besser. Unsere Herangehensweise ist typisch für solche Aufgaben.


\subsection{Potential}

Zuerst müssen wir das Potential $U(x)$ finden, das unsere Situation 'Teilchen im Kasten' beschreibt. Der Kasten hat die Länge $L$ und die Wände sind starr, geben nicht nach und absorbieren keine Energie. Dies sind, wie so oft, vereinfachende Annahmen.

Als Konsequenz dieser Annahmen kann sich das Teilchen im Bereich $0 < x < L$ frei bewegen. An den Enden, d.h. bei $x=0$ und $x=L$, wird das Teilchen unabhängig von seiner Energie reflektiert. Die Bereiche $x<0$ und $x>L$ sind verboten. Das Teilchen kann die Box nicht verlassen. Dieser Zustand wird durch das Potential 
\begin{equation}
    U(x) = \left\{ 
        \begin{matrix}
            0 & 0 \le x \le L \\
            \infty & \text{sonst}
        \end{matrix}
    \right.
\end{equation}
beschrieben. Innerhalb des Kastens hat das Teilchen nur kinetische Energie, das Potential ist Null. Aber egal wie groß die Gesamtenergie des Teilchens ist, sie reicht niemals aus, um die unendlich große potenzielle Energie außerhalb des Kastens aufzubringen. Das Teilchen kann den Kasten nicht verlassen.


\subsection{Randbedingungen}

\begin{marginfigure}
    \inputtikz{\currfiledir kasten_potential}
    \caption{Potential: Teilchen im Kasten}
    \label{fig:4_kasten_potential}
\end{marginfigure}

Wir zeichnen das Potenzial (Abb.  \ref{fig:4_kasten_potential}). Wir schneiden die Potentialstufen, die ins Unendliche gehen würden, ab und ersetzen sie durch einen Pfeil nach oben.

Das Teilchen kann sich physikalisch nicht außerhalb des Kastens befinden. Daher muss die Wellenfunktion dort Null sein, d.h.
\begin{equation}
    \Psi(x) = 0 \quad \text{für} \quad x < 0  \quad \text{oder} \quad x > L \quad .
 \end{equation}
 Daher ist die Wahrscheinlichkeitsdichte $|\Psi(x)|^2$ in diesen Bereichen gleich Null.

Die Wellenfunktion muss ebenfalls stetig sein. Deshalb muss sie auch an den Rändern des Kastens Null sein, d. h.
\begin{equation}
    \Psi(x) = 0 \quad \text{für} \quad x = 0  \quad \text{oder} \quad x = L \quad .
 \end{equation}
 Unabhängig von der Form der Wellenfunktion im Kasten muss es an den Wänden einen Knoten geben.



\subsection{Lösung der Gleichung}

Wir suchen eine Lösung der Schrödingergleichung Gl.~\ref{eq:4_SG_1d} für $U=0$, also nur im Bereich $x0<x<L$. Alles andere haben wir schon. Eine Lösung bedeutet, dass wir sowohl eine Funktion $\Psi(x)$ als auch die zugehörige Energie $E$ finden müssen.
Wir schreiben die Schrödingergleichung Gl.~\ref{eq:4_SG_1d} mit $\beta^2 = 2 m E / \hbar^2$ als
\begin{equation}
    \frac{d^2}{dx^2} \Psi(x) = - \beta^2 \Psi(x)  \quad .
  \end{equation}
Diese Differentialgleichung wird durch Sinus und Kosinus gelöst, also ist 
\begin{equation}
    \Psi(x) = A \sin \beta x  \, + \, B \cos \beta x
\end{equation}
die allgemeine Lösung. 

Die Unbekannten $A$ und $B$ müssen nun so gefunden werden, dass die Randbedingungen erfüllt sind. Für $x=0$ muss $\Psi$ Null sein:
\begin{equation}
    \Psi(0) =  A \sin   0  \, + \, B \cos   0 = B \overset{!}{=} 0 \quad \text{also} \quad B = 0 \quad .
\end{equation}
Ebenso muss $\Psi$ bei $x=L$ Null sein, also 
\begin{equation}
    \Psi(L) = A \sin \beta L \overset{!}{0} \text{also} \quad \beta L = n \, \pi \quad \text{mit} \quad n = 1, 2, 3, \dots \quad .
\end{equation}
Der Fall $n=0$ ist physikalisch unsinnig, da hier $\Psi = 0$ für alle $x$, also überhaupt kein Teilchen vorhanden wäre.

Damit haben wir eine unendliche Menge an   Wellenfunktionen gefunden
\begin{equation}
    \Psi_n(x) = A \sin \beta_n x = A \sin \left(  \frac{n\pi}{L} x \right) \quad \text{mit} \quad n = 1, 2, 3, \dots \quad .
\end{equation}
Die Konstante $A$ werden wir unten aus der Normierung bestimmen.



\subsection{Energie-Eigenwerte}

Dann interessieren uns die möglichen Werte der Energie $E$, die zu diesen Wellenfunktionen $\Psi_n$ gehören. Mit der Definition von $\beta$ finden wir
\begin{equation}
    \beta_n = \frac{\sqrt{ 2 m E_n}}{\hbar} = \frac{n\pi}{L} \quad \text{mit} \quad n = 1, 2, 3, \dots
\end{equation}
und so 
\begin{equation}
    E_n = \frac{\pi^2 \, \hbar^2}{2 m L^2} \, n^2 \quad \text{mit} \quad n = 1, 2, 3, \dots \quad .
    \label{eq:4_En_kasten}
\end{equation}
Damit haben wir die Quantisierung. Nur diese Energiewerte sind für ein Teilchen im Kasten möglich. Dies ist eine Folge des Kastens, des \textit{confinement}, nicht eine Eigenschaft des Teilchens selbst.  Abb.  \ref{fig:4_kasten_eigenwerte} skizziert diese Leiter der Energie-Eigenwerte. In solchen Skizzen hat nur die vertikale Achse als Energieachse eine Bedeutung. Die horizontale Achse hat keine physikalische Bedeutung und dient nur der graphischen Darstellung.

\begin{marginfigure}
    \inputtikz{\currfiledir kasten_eigenwerte}
    \caption{Energieeigenwerte des Teilchens im Kasten}
    \label{fig:4_kasten_eigenwerte}
\end{marginfigure}

Damit haben wir eine Theorie, mit der wir die Ergebnisse des letzten Kapitels erreichen können, ohne phänomenologische Annahmen machen zu müssen.


\subsection{Normierung}

Schließlich muss die Wellenfunktion $\Psi$ noch normiert werden. Das Integral über den Betrag von $\Psi$ muss eins ergeben, da das Teilchen irgendwo sein muss, also
\begin{equation}
    \int_{-\infty}^{+\infty} \left| \Psi(x) \right|^2 \,dx = 1 \quad .
\end{equation}
Da der Kasten das Teilchen auf den Bereich zwischen $0$ und $L$ beschränkt, genügt es, über dieses Intervall zu integrieren, d.h. 
\begin{equation}
    \int_0^{L} \left|  A_n \sin \left(  \frac{n\pi}{L} x \right) \right|^2 \,dx =
   A_n^2  \int_0^{L} \sin^2 \left(  \frac{n\pi}{L} x \right)  \,dx = 
   A_n^2  \frac{L}{2} = 1
\end{equation}
also
\begin{equation}
    A_n = \sqrt{\frac{2}{L}} \quad \text{für alle }n \quad .
\end{equation}
Damit haben wir die normierte Wellenfunktion gefunden:
\begin{equation}
    \Psi_n(x) = \left\{
    \begin{matrix}
        \sqrt{\frac{2}{L}}  \sin \left(  \frac{n\pi}{L} x \right) \quad &\text{für} \quad & 0 \le x \le L \\
        0  &  \text{für} &  x < 0 \text{ oder } x > L 
    \end{matrix}
    \right. \quad .
    \label{eq:4_psi_kasten}
\end{equation}

\section{Interpretation}

Wir wissen  für das Teilchen im Kasten, dass die möglichen diskreten Werte der Energie durch Gl.  \ref{eq:4_En_kasten} gegebn sind. Es gibt eine minimale Energie 
\begin{equation}
E_1 = \frac{\pi^2 \, \hbar^2}{2 m L^2}
\end{equation}
und alle anderen steigen quadratisch in $n$ an. Die Wellenfunktion ist eine stehende Welle, gegeben durch Gl. \ref{eq:4_psi_kasten}. Die Zahl $n$ beschreibt die Anzahl der Bäuche. An den Wänden des Kastens befinden sich immer Knoten.
Dies sind die stationären Zustände des Systems, wie Niels Bohr sie gefordert hat.

Die Wahrscheinlichkeitsdichte ist gegeben durch
\begin{equation}
    P_n(x) = |\Psi_n(x)|^2 = \frac{2}{L} \, \sin^2 \left(  \frac{n\pi}{L} x \right) \quad .
    \label{eq:4_WK_kasten}
\end{equation}
Abbildung  \ref{fig:4_topf_unendlich} zeigt dies für die ersten Werte von $n$. Es gibt also Bereiche entlang des Ortes $x$, in denen die Wahrscheinlichkeit, das Teilchen zu finden, gleich Null ist. Und es gibt $n$ Bereiche, in denen die Wahrscheinlichkeit maximal wird.

\begin{marginfigure}
    \inputtikz{\currfiledir topf_unendlich}
    \caption{Teilchen im Kasten, 1 nm breit, hier mit Energie-Eigenwerten für ein Elektron.}
    \label{fig:4_topf_unendlich}
\end{marginfigure}

Skizzen vom Typ \ref{fig:4_topf_unendlich} werden gerne verwendet, um Wellenfunktionen und Eigenenergien gleichzeitig darzustellen. Dabei ist jedoch Vorsicht geboten. Das Potential $U(x)$ ist leicht verständlich und wird 'wie üblich' gezeichnet. Die Eigenenergie $E_n$ hat keine $x$-Abhängigkeit. Nur die Lage der Linie auf der y-Achse ist von Bedeutung. Die Energie $E_n$ und das Potential $U(x)$ haben jedoch eine gemeinsame Energie-y-Achse. Die Wellenfunktion $\Psi_n(x)$ teilt sich mit dem Potential die x-Achse. Man verschiebt aber gerne den Nullpunkt von $\Psi_n(x)$ so, dass er auf $E_n$ liegt. Und natürlich hat $\Psi_n(x)$ eine andere Einheit als Energie. Es gibt also eine zweite Skala für die y-Achse, aber die zeichnet man eigentlich nie ein.


\begin{questions}
    \item In dieser Simulation\phet{Quantum_Bound_States} können Sie das alles ausprobieren. 
\end{questions}

\section{Nullpunktenergie}

Eine wichtige und für die Quantenmechanik charakteristische Eigenschaft des Teilchens im Kasten ist die \emph{Nullpunktenergie}. Der Zustand der niedrigsten Energie ist 
\begin{equation}
    E_1 = \frac{\pi^2 \, \hbar^2}{2 m L^2} > 0 \quad .
\end{equation}
Obwohl das Potential $U(x)$ Null ist, hat das Teilchen selbst im Grundzustand eine Energie, die etwas höher ist als der 'Boden' des Potentialtopfes. Diese Energie muss in der kinetischen Energie enthalten sein, d.h. das Teilchen bewegt sich! In der Quantenmechanik befindet sich ein Teilchen im Kasten nie in Ruhe. Selbst im Grundzustand bewegt es sich mit 
\begin{equation}
    \frac{1}{2} m v^2 = \frac{\pi^2 \, \hbar^2}{2 m L^2}  \quad .
\end{equation}

Nach der Heisenbergschen Unschärferelation muss das so sein. Wir wissen, dass das Teilchen irgendwo im Kasten ist. Daher ist die Ortsunschärfe $\Delta x = L$. Würde sich das Teilchen nicht bewegen, wäre seine Geschwindigkeit gleich Null und damit genau bekannt. Die Geschwindigkeitsunschärfe und damit die Impulsunschärfe wäre also ebenfalls Null, womit die Heiensbergsche Unschärferelation verletzt wäre. 
Die Impulsunschärfe kann also nicht Null sein. Ein Teilchen, das sich nur in einem begrenzten Raumbereich aufhalten kann, muss in Bewegung sein, muss eine Impulsunschärfe haben.


\section{Korrespondenzprinzip}
Der Welle-Teilchen-Dualismus ist uns an verschiedenen Stellen begegnet. Objekte auf der atomaren Skala können nicht mehr nur als Welle oder nur als Teilchen beschrieben werden, sondern sind beides oder nichts von beidem. Unsere Bilder sind zu schwach, um diese Objekte zu beschreiben. Es hat sich jedoch gezeigt, dass man den Übergang von der mikroskopischen zur makroskopischen Welt nutzen kann, um die Konsistenz der mikroskopischen Beschreibung zu überprüfen. 

Die Idee ist, dass ein mikroskopisches Modell für große Quantenzahlen in ein klassisches makroskopisches Modell übergehen muss. Dies ist das \emph{Korrespondenzprinzip} von Niels Bohr. Versuchen wir es für das Teilchen im Kasten. Die betrachtete Größe ist die Wahrscheinlichkeitsdichte $P(x)$ (Gl. \ref{eq:4_WK_kasten}). Wir stellen uns einen großen makroskopischen Kasten vor, in dem sich ein Teilchen mit der Geschwindigkeit $v(x)$ periodisch hin und her bewegt, weil es am Ende reflektiert wird. Die Periodendauer ist $T$ und die Länge des Kastens wiederum $L$. Die Wahrscheinlichkeit, das Teilchen am Ort $x$ in einem Streifen der Breite $\delta x$ zu finden, hängt von der Zeit $\delta t$ ab, die das Teilchen benötigt, um die Strecke $\delta x = v(x) \delta t$ zu durchlaufen. Wenn wir berücksichtigen, dass das Teilchen den Ort $x$ zweimal pro Umlauf passiert, dann ist die Wahrscheinlichkeit
\begin{equation}
    WK(x, \delta x) = \frac{\delta t}{2 T } = \frac{\delta x}{2T v(x)}
\end{equation} 
und die Wahrscheinlichkeitsdichte
\begin{equation}
    P(x) = \frac{WK(x, \delta x)}{\delta x} = \frac{1}{2 T v(x)} \quad .
\end{equation}
Dies gilt für jede Form der Bewegung, also für jedes Geschwindigkeitsprofil $v(x)$.

Für das Teilchen im Kasten ist die Geschwindigkeit konstant, also
\begin{equation}
    v(x) = \frac{2L}{T} \quad \text{und} \quad P(x) = \frac{1}{L} \quad .
\end{equation}
Im klassischen Fall ist die Wahrscheinlichkeitsdichte über die Position im Kasten konstant. Wie passt das mit dem $\sin^2$ aus Gl.~\ref{eq:4_WK_kasten} zusammen? Mit zunehmender Quantenzahl $n$ wird die Oszillation von $P_n(x)$ immer schneller. Für sehr große $n$ entstehen dann räumlich so dicht beieinander liegende Streifen, dass diese nicht mehr aufgelöst werden können und nur noch ein Mittelwert beobachtet wird. Dieser Mittelwert ist genau $1/L$!


\section{Beispiel: Potentialtopf}

Nun machen wir unser Teilchen-im-Kasten-Modell etwas realistischer. Das Potential $U(x)$ soll außerhalb des Kastens nicht mehr unendlich groß sein, sondern einen endlichen Wert $U_0$ annehmen, also
\begin{equation}
    U(x) = \left\{ 
        \begin{matrix}
            0 & 0 \le x \le L \\
            U_0 & \text{sonst}
        \end{matrix}
    \right. \quad .
\end{equation}
Alternativ kann der Energienullpunkt auch auf das Plateau außerhalb des Kastens gelegt werden und der Boden auf $-U_0$. Der Energienullpunkt ist immer verschiebbar.

Im klassischen Fall ist das Teilchen im Kasten gebunden und kann ihn nicht verlassen, solange seine Energie $E < U_0$ ist. Für größere Energien kann das Teilchen dem Kasten entkommen.

\begin{marginfigure}
    \inputtikz{\currfiledir topf_endlich}
    \caption{Potentialtopf für ein Elektron, 1 nm breit und 3 eV tief. Es gibt nur diese 3 gebundenen Zustände.}
    \label{fig:4_topf_endlich}
\end{marginfigure}

In der Quantenmechanik ist die Situation ähnlich.  Die Berechnung ist in den üblichen Lehrbüchern der Quantenmechanik enthalten. Hier diskutieren wir nur die Ergebnisse. Abbildung  \ref{fig:4_topf_endlich} zeigt ein Beispiel für einen Elektron in einem Kasten mit einer Breite von 1~nm und einer Tiefe von 3~eV, wie er für Elektronen in Halbleitern typisch ist.
\begin{itemize}\setlength{\itemsep}{0pt}
    \item Die Zustände sind weiterhin quantisiert. Nur bestimmte diskrete Energiewerte können angenommen werden.
    \item Es gibt nur eine endliche Anzahl von gebundenen Zuständen mit $E < U_0$. Die genaue Anzahl hängt von den Parametern des Topfes ab.
    \item Die Wellenfunktionen sind ähnlich denen des Teilchens im Kasten. Die Anzahl der Bäuche entspricht der Quantenzahl $n$.
    \item Der Hauptunterschied besteht darin, dass nun die Wellenfunktionen und damit auch die Wahrscheinlichkeitsdichte in den klassisch verbotenen Bereich hineinreichen. Das Teilchen wird an dieser niedrigen Wand reflektiert, dringt dabei aber teilweise in die Wand ein.
\end{itemize}

Wenn man die Wellenfunktionen im klassisch verboteten Bereich $x > L$ berechnet, findet man
\begin{equation}
\Psi(x) = \Psi_L \, e^{- (x-L)/ \eta} \quad \text{für} \quad x > L
\end{equation}
wobei $\Psi_L$ der Wert der Wellenfunktion am Ort $x=L$ ist, der über die Lösung im Bereich $0 \le x \le L$ gefunden werden muss. Die charakteristische  Abfall-Länge $\eta$ ist
\begin{equation}
    \eta = \frac{\hbar}{\sqrt{2m ( U_0 - E)}} \quad \text{für} \quad U_0 > E \quad .
\end{equation}
Nach dieser Länge $\eta$ ist die Wellenfunktion in der Wand auf $e^{-1} \approx 0.37$ abgefallen. Für makroskopische Teilchen ist die Eindringtiefe oder Abfall-Länge $\eta$ vernachlässigbar, nicht aber für Quantenobjekte. Diese sind etwas 'unscharf' und werden nicht direkt an der Wand reflektiert, sondern etwas später.

Das Teilchen im Potentialtopf ist ein häufig verwendetes Modell. Es beschreibt z.B. sehr gut Neutronen und Protonen in Atomkernen oder Elektronen in gezielt strukturierten Halbleitern, wie sie in Halbleiterlasern verwendet werden.

\section{Wellenfunktionen raten}

Für viele Zwecke ist es ausreichend, die ungefähre Form der Wellenfunktion bei gegebenem Potential zu erraten. Dazu müssen die oben definierten Randbedingungen erfüllt sein. Zwei weitere Punkte kommen hinzu:
\begin{description}
    \item[de Broglie Wellenlänge] Die de Broglie Wellenlänge ist proprotional zu $1/p$, bzw. $1/\sqrt{E_{kin}}$. Über 'tiefen' Stellen des Potentials $U$ ist $E_{kin}$ größer und damit die Wellenlänge kürzer.
    \item[Wahrscheinlichkeitsdichte] Nach dem Korrespondenzprinzip ist die Wahrscheinlichkeitsdichte an Stellen größer, an denen das Teilchen langsamer ist. An 'flachen' Stellen des Potentials ist $E_{kin}$ kleiner und damit die Amplitude der Wellenfunktion größer.
\end{description}

Man zeichnet also zunächst das Potential $U(x)$ und trägt den Energieeigenwert $E$ ein. Die Wellenfunktion oszilliert im klassisch erlaubten Bereich, d.h. wenn $E > U(x)$. Die Quantenzahl $n$ gibt die Anzahl der Bäuche der Oszillation an. Wellenlänge und Amplitude der Oszillation variieren mit $U(x)$, wie gerade beschrieben. Bei unendlich hohen Potentialwällen ist die Wellenfunktion gleich Null, ansonsten fällt sie exponentiell in den klassisch verbotenen Bereich ab. Die Abklinglänge ist umso größer, je kleiner der Abstand zwischen $E$ und der Ebene $U_0$ des Potentials ist.


\section{Beispiel: Harmonischer Oszillator}

Der harmonische Oszillator ist eines der zentralen Modelle der klassischen Physik. Es wirkt eine Rückstellkraft, die proportional zur Auslenkung $x$ ist. Diese wird durch ein parabelförmiges Potential beschrieben:
\begin{equation}
    U(x) = \frac{1}{2} k x^2n\quad .
    \label{eq:4_harm_pot}
\end{equation}
Es handelt sich quasi um einen Potentialtopf mit gekrümmtem Boden.
Ein klassisches Teilchen schwingt mit der Kreisfrequenz $\omega$
\begin{equation}
    \omega = \sqrt{\frac{k}{m}} \quad .
\end{equation}

In der Quantenmechanik lösen wir die Schrödingergleichung Gl. \ref{eq:4_SG_1d} mit dem harmonischen Potential \ref{eq:4_harm_pot}. Die Lösungen sind Hermitesche Funktionen, deren erste 3 lauten
\begin{align}
    \Psi_1(x) = & A_1 \, e^{-x^2 / 2 b^2} \\
    \Psi_2(x) = & A_2 \, \frac{x}{b} \, e^{-x^2 / 2 b^2} \\
    \Psi_3(x) = & A_3 \, \left(1- \frac{2x^2}{b^2} \right) \, e^{-x^2 / 2 b^2} 
\end{align}
mit der Abkürzung 
\begin{equation}
    b = \sqrt{\frac{\hbar}{m \omega}} \quad .
\end{equation}
Die Länge $b$ ist gerade die maximale Auslenkung eines klassischen Oszillators bei der Energie des Grundzustandes $n=1$. Die Konstanten $A_i$ sind so gewählt, dass die Wellenfunktionen $\Psi_i$ normiert sind. Auch diese Wellenfunktionen zeigen eine Oszillation und auch hier gibt die Quantenzahl $n$ die Anzahl der Bäuche an. Da die Wände des Potentialtopfs nicht unendlich hoch sind, ragen die Wellenfunktionen etwas über die klassischen Umkehrpunkte bei $U(x) = E_n$ hinaus.

\begin{marginfigure}
    \inputtikz{\currfiledir vib_state_wf}
    \caption{Wellenfunktionen und Wahrscheinlichkeitsdichte im harmonischen Oscillator}
\end{marginfigure}

Die zugehörigen Eigenenergien $E_i$ sind
\begin{equation}
    E_n = \left( n - \frac{1}{2}\right) \, \hbar \omega \quad \text{mit} \quad n = 1, 2, 3, \dots \quad .
\end{equation}
Die Zustände im quantenmechanischen harmonischen Oszillator sind also äquidistant im Abstand $\hbar\omega$, im Gegensatz zum Kasten, in dem die Abstände quadratisch zunehmen. Auch hier gibt es wieder eine Nullpunktsenergie $E_1 = \hbar\omega/2$, die mit der Nullpunktbewegung des Oszillators verbunden ist. Ein quantenmechanisches Pendel steht also niemals still.

Auch hier gilt das Korrespondenzprinzip. Für große Quantenzahlen geht die Bewegung des quantenmechanischen harmonischen Oszillators in die klassische Bewegung über. Die Skizze \ref{fig:4_harm_osz_korrepondenz} zeigt die Wahrscheinlichkeitsdichte für den Zustand $n=11$ im Vergleich zum klassischen Fall. Gut zu erkennen ist die erhöhte Aufenthaltswahrscheinlichkeit an den Umkehrpunkten der Bewegung, an denen die Geschwindigkeit gering ist.

\begin{marginfigure}
    \inputtikz{\currfiledir harm_osz_korrepondenz}
    \caption{Klassische (dick) und quantenmechanische (dünn, $n=11$) Wahrscheinlichkeitsdichte des harmonischen Oszillators.}
    \label{fig:4_harm_osz_korrepondenz}
\end{marginfigure}

Dieses Modell des quantenmechanischen harmonischen Oszillators beschreibt z.B. gut die Schwingung von Atomen, die in Molekülen oder Festkörpern gebunden sind.

\section{Beispiel: Tunneln durch eine Barriere}

Als letztes Beispiel betrachten wir eine Barriere wie in Abb.  \ref{fig:4_barrier} skizziert ist. Das Potential $U(x)$ ist überall Null, nur in einem kurzen Bereich nimmt es den Wert $U_0 > 0$ an. Die Übergänge dazwischen sind nicht so wichtig und werden hier als trapezförmig gezeichnet.

Ein klassisches Teilchen mit der Energie $0 < E < U_0$ würde sich über weite Strecken mit konstanter Geschwindigkeit bewegen und dann den Potentialberg hinaufklettern, bis am Ort $x_0$ der Punkt $U(x_0) = E$ erreicht ist. Dann ist alle kinetische Energie in potentielle Energie umgewandelt, das Teilchen kommt zum Stillstand und rollt den Berg wieder hinunter, hat also seine Bewegungsrichtung geändert. Nur wenn ein Teilchen die Energie $E > U_0$ besitzt, kann es die Barriere überwinden.

\begin{marginfigure}
    \inputtikz{\currfiledir barrier}
    \caption{Tunneln durch eine Barriere. Dargestellt ist der Realteil der Wellenfunktion $\Psi(x)$. Der grau unterlegte Bereich $U > E$ ist klassisch verboten.}
    \label{fig:4_barrier}
\end{marginfigure}


Anders verhält es sich in der Quantenmechanik. Wir haben bereits gesehen, dass quantenmechanische Teilchen in den klassisch verbotenen Bereich eintreten können und ihre Wellenfunktion dort exponentiell abfällt. Dies geschieht nun auch an der Barriere für $E < U_0$. Wenn aber die Barriere so dünn ist, dass die Wellenfunktion auch am hinteren Ende noch nicht auf Null gefallen ist, dann tritt das Teilchen dort aus und bewegt sich normal weiter. Man sagt, dass das Teilchen durch die Barriere getunnelt ist. 

Auf beiden Seiten\sidenote{diese Rechnung nimmt senkrechte Wände der Barriere an} der Barriere, in den klassisch zulässigen Bereichen, wird die Wellenfunktion des Teilchens durch Sinus- und Kosinusfunktionen beschrieben, wie wir es ganz am Anfang des Kapitels getan haben. Alternativ kann auch eine komplexwertige Wellenfunktion angenommen werden, z. B. links von der Barriere mit der Breite $w$ bei $0 < x < w$.
\begin{equation}
    \Psi(x < 0) = \Psi_0 e^{i \beta x}
\end{equation}
mit einer komplexwertigen Konstanten $\Psi_0$  und $\beta = \sqrt{2 m E }/ \hbar$ wie oben. Da wir die Wellenfunktion $\Psi$ selbst nicht beobachten können, ist es auch kein Problem, wenn sie komplexwertig ist. Beobachtbar ist das Betragsquadrat $|\Psi|^2$, das auch hier existiert.

Innerhalb der Barriere fällt die Wellenfunkion dann exponentiell ab
\begin{equation}
    \Psi(0 \le x \le w) = \Psi_{Kante} e^{- x / \eta}
\end{equation}
mit einer Abfall-Länge $\eta$ wie oben 
\begin{equation}
    \eta = \frac{\hbar}{\sqrt{2m ( U_0 - E)}} \quad \text{für} \quad U_0 > E \quad .
\end{equation}

Für die gesamte Wellenfunktion werden die drei Teile links, Barriere und rechts so zusammengesetzt, dass die Übergänge stetig sind.  Der Ansatz als Sinuswelle oder Exponentialfunktion ist nicht normierbar, was in diesem Fall toleriert werden muss. Alternativ könnte man mit einem Wellenpaket arbeiten, was aber komplizierter ist.

Wie auch Abbildung \ref{fig:4_barrier} zeigt, ist die Amplitude der Wellenfunktion nach der Barriere kleiner als vorher. Dies kann man als Tunnel-Wahrscheinlichkeit beschreiben, d.h. die Wahrscheinlichkeit, dass ein Teilchen, das links auf die Barriere trifft, rechts wieder herauskommt (andernfalls wird es reflektiert, geht aber nie verloren). Diese Wahrscheinlichkeit beträgt
\begin{equation}
    P_{tunnel} = \frac{|\Psi(0)|^2)} {|\Psi(w)|^2} = e^{-2 w / \eta} \quad .
\end{equation}
Diese Wahrscheinlichkeit $P_{tunnel}$ bzw. die Abklinglänge $\eta$ hängt stark von der Breite und Form der Barriere und der Differenz zwischen der Teilchenenergie $E$ und der Barrierenhöhe $U_0$ ab.

Im \emph{Rastertunnelmikroskop} (engl. scanning tunneling microscope, STM) wird dieser Effekt ausgenutzt. Elektronen tunneln durch einen kleinen ($< 1$~nm) Vakuumspalt zwischen einer Metallspitze und der leitfähigen Probe. Die Stromstärke der tunnelnden Elektronen wird als Funktion der Spitzenposition gemessen und als Bild aufgetragen. Manchmal wird eine Pseudo-3D-Darstellung verwendet, bei der Bereiche mit höheren Strömen als Berge dargestellt werden.



\begin{questions}
    \item In dieser Simulation\phet{Quantum_Tunneling_and_Wave_Packets} können Sie das alles ausprobieren. 
\end{questions}

\subsection{Korrespondenz zur Optik}

Photonen sind Quantenteilchen, die der Quantenmechanik unterliegen. Gleichzeitig wird Licht aber auch durch die Elektrodynamik und die Optik gut beschrieben. Viele Experimente mit Quantenteilchen lassen sich mit Photonen durchführen. Die Ergebnisse müssen aufgrund des Korrespondenzprinzips mit denen der Optik übereinstimmen.

Die Reflexion an einer Barriere eines Quantenteilchens entspricht in der Optik der Totalreflexion am Übergang von einem dichten zu einem dünnen Medium. Auch hier dringt das Lichtfeld etwas in das dünne Medium ein (ein Bruchteil der Wellenlänge), wird aber dennoch vollständig reflektiert.

Man kann jedoch ein zweites, ebenfalls optisch dichtes Medium nahe an das erste heranbringen und nur einen dünnen Bereich des dünnen Mediums stehen lassen. Durch diesen Spalt kann das Photon tunneln. Klassisch wird dies als frustrierte Totalreflexion bezeichnet, die genau denselben Gesetzen folgt. Dieser Effekt wird manchmal in optischen Strahlteilern ausgenutzt.




%\section{Anhang: numerische Rechnung}

\section{Zusammenfassung}

\textit{Schreiben Sie hier ihre persönliche Zusammenfassung des Kapitels auf. Konzentrieren Sie sich auf die wichtigsten Aspekte.}

\vspace*{10cm}



%--------------------
\printbibliography[segment=\therefsegment,heading=subbibliography]

\renewcommand{\lastmod}{1. November 2024}
\renewcommand{\chapterauthors}{Markus Lippitz}

\chapter{Quantentheorie des H-Atoms}



\goal{By the end of this chapter, you should be able to draw, calculate and align a ray's path through an optical system.}

Ich kann die Schrödinger-Gleichung für das Wasserstoff-Atom (ohne Spin und Magnetfeld) aufstellen, ihren Lösungsweg skizzieren und Eigenschaften der Lösung beschreiben.

Ich kann anhand des Stern-Gerlach-Versuchs den Spin des Elektrons erklären.


\section{Overview}

s.a. Demtröder 3, Kap. 5

41.1 The Hydrogen Atom: Angular Momentum and Energy 1231

41.2 The Hydrogen Atom: Wave Functions and Probabilities 1234

Relativistische Korrekturen

41.3 The Electron's Spin 1237
Inkl. Stern-Gerlach, Zeeman-Effekt

5.3 Spin-Bahn-Kopplung bei Wasserstoff 3	***	

Hyperfeinstruktur ?? wohl erst später, nur erwähnen

5.6 Addition von Drehimpulsen 6	Wdh	6 

5.5 Pauli-Prinzip5	Wdh	5 

5.7 Helium 7	**	

\phet{Models_of_the_Hydrogen_Atom}

\url{https://phet.colorado.edu/en/simulations/stern-gerlach}


%review : %https://phet.colorado.edu/sims/cheerpj/hydrogen-atom/latest/hydrogen-atom.html?simulation=hydrogen-atom


% 19. Hydrogen atom
% Sims: Models of the Hydrogen Atom, Rutherford Scattering (also falstad.com/qmatom)
% • It is extremely important in this section to relate the Schrodinger model of the atom back to
% the discussion of models of the atom earlier in the course (section 8), and show how this is the
% next step in the progression of models. Otherwise, students are likely to view this section as
% just one more example of a solution of the Schrodinger equation and not realize that we are
% actually talking about another model of the atom.
% • We have a homework in which we ask students to work through the simulations and explain
% the reasons for and limitations of each models. It’s amazing how difficult this is for students.


\section{Schrödinger-Gleichung in 3 Dimensionen}
In der Quantenmechanik ist das Wasserstoffatom nur eine spezielle Form eines Potentialtopfs, nämlich ein dreidimensionaler Topf mit kugelsymmetrischem Potential, das durch das Coulombpotential gegeben ist. Das Potential hängt also nur vom Abstand $r$ zwischen Kern und Elektron ab, nicht von einer Richtung\sidenote{Ich verwende 'fette' Buchstaben wie $\br$ für Vektoren und 'dünne' Buchstaben wie $r$ für die Länge dieser Vektoren.}
\begin{equation}
    U(r) = - \frac{1}{4 \pi \epsilon_0} \, \frac{e^2}{r}
\end{equation}
weil sowohl Kern als auch Elektron jeweils die Ladung $\pm e$ tragen.

Im letzten Kapitel hatte ich die Schrödingergleichung in einer Dimension $x$ geschrieben, hier nur in 3 Dimensionen, ganz analog dazu
\begin{equation}
    - \frac{\hbar^2}{2m} \,\nabla^2 \, \Psi(\br) + \left[ U(\br) - E \right] \Psi(\br) = 0 \quad .
    \label{eq:5_SG_3d}
  \end{equation}
mit dem Quadrat des Nabla-Operators $\nabla^2$ als Abkürzung für
\begin{equation}
    \nabla^2 = \frac{\partial^2}{\partial x^2 } + \frac{\partial^2}{\partial y^2 } + 
    \frac{\partial^2}{\partial z^2 } 
\end{equation}
d.h. die Summe der doppelten (partiellen) Ableitungen in den drei Raumrichtungen. Wie im letzten Kapitel werden wir das nie wirklich selbst ausrechnen, sondern uns nur die Lösung anschauen. Die Rechnung findet man in jedem Buch zur Quantenmechanik.

\section{Quantenzahlen des Wasserstoff-Atoms}

In einer Dimension haben wir im letzten Kapitel gesehen, dass die Beschränkung des Teilchens auf einen Raumbereich zur Quantisierung der Energie und damit zur Quantenzahl $n$ führt, mit der wir die möglichen Energiewerte durchnummeriert haben. Nur diese Energien waren möglich, nur diese Wellenfunktionen lösten die Schrödingergleichung. Das gleiche gilt in drei Dimensionen. Schränkt man das Teilchen in drei Raumrichtungen ein, so erhält man drei Quantisierungen. Drei verschiedene Größen können nur bestimmte quantisierte Werte annehmen, wenn die Schrödingergleichung für das kugelsymmetrische Coulombpotential gelöst werden soll. Dies sind
\begin{description}
    \item[Hauptquantenzahl] Die Zahl $n$, die die Energien durchnummeriert, wird nun Hauptquantenzahl genannt, da weitere Quantenzahlen hinzukommen. Die zugehörigen Eigenenergien sind
 \begin{equation}
E_n = - \frac{1}{n^2} \left( \frac{1}{4 \pi \epsilon_0} \, \frac{e^2}{2 a_B }\right) = - \frac{13.6 \, eV}{n^2} \quad \text{mit} \quad n = 1,2, 3 \dots       
    \end{equation}
    mit dem Bohr-Radius $a_B = 4 \pi \epsilon_0 \hbar^2 / (m e^2) \approx 0.5$\AA. Das sind die gleichen Energien, die wir auch im Bohrmodell gefunden haben.

    \item[Drehimpuls-Quantenzahl] Der Bahndrehimpuls $\bL$ des Elektrons ist in seiner Länge $L$ quantisiert
    \begin{equation}
        L = \hbar \sqrt{l (l+1) }  \quad \text{mit} \quad l = 0, 1,2, \dots , n-1
    \end{equation}
Diese Zahl $l$ wird (Bahn)Drehimpuls-Quantenzahl genannt.

\item[Magnetische Quantenzahl] Die z-Komponente $L_z$ des Bahndrehimpulses $\bL$ ist ebenfalls quantisiert 
\begin{equation}
    L_z = m \hbar   \quad \text{mit} \quad m = -l, -l+1, \dots, 0, \dots, l-1, l
\end{equation}
Diese Zahl $m$ wird magnetische Quantenzahl genannt. Den Grund für diesen Namen sehen wir unten.

\end{description}
Jeder stationäre Zustand des Wasserstoffatoms ist also durch drei Zahlen $(n,l,m)$ definiert. Jede dieser Zahlen beschreibt eine physikalische Eigenschaft des Atoms, wobei die Energie im Wasserstoffatom nur von der Hauptquantenzahl $n$ abhängt.

In anderen Atomen und bei einer Erweiterung des Modells aufgrund der Relativitätstheorie (siehe unten) hat dann auch die Drehimpulsquantenzahl $l$ einen Einfluss auf die Energie. Man bezeichnet daher die Zustände der Elektronen in Atomen mit den beiden Zahlen $n$ und $l$ (nicht aber $m$). Dazu kodiert man die Drehimpuls-Quantenzahl als Buchstaben nach folgendem Schema
\begin{equation}
    l = 0, 1, 2, 3 \quad \text{ergibt Buchstaben \ \ s, p, d, f}
\end{equation}
Der Zustand $(n,l) = (1,0)$ wird 1s genannt. Der Zustand $(n,l) = (3,2)$ heißt 3d. Da die magnetische Quantenzahl $m$ insgesamt $2l+2$ Werte annehmen kann, ist der Zustand 3d 7-fach entartet, der Grundzustand 1s dagegen nicht.



Abbildung XXX zeigt die möglichen Zustände (ohne $m$-Entartung). Die Bedingung $n > l$ führt zu dieser dreieckigen Anordnung. Alle Zustände mit gleichem $n$ haben die gleiche Energie. Die Zustände liegen mit zunehmender Hauptquantenzahl $n$ immer näher an der Ionisationsgrenze $E=0$. Die Abhängigkeit von $n$ ist genau wie beim Bohr-Modell.


\section{Quantisierung des Drehimpulses}

Wir müssen noch etwas genauer auf den Drehimpuls eingehen. Wir haben schon beim Bohrmodell gesehen, dass der Bahndrehimpuls quantisiert ist. Damals konnte er nur ganzzahlige Vielfache von $\hbar$ annehmen. Jetzt ist es ähnlich, nur die Werte sind etwas anders, nämlich wie oben.
\begin{equation}
    L = \hbar \sqrt{l (l+1) } = 0, \sqrt{2} \hbar, \sqrt{6} \hbar, \sqrt{12} \hbar, \dots
\end{equation}
mit einer ganzen Zahl $l \ge 0$.

Dies ist die \emph{Länge} des Drehimpulsvektors $\bL$. Seine drei kartesischen Komponenten sind $L_x$, $L_y$ und $L_z$ und natürlich
\begin{equation}
    L^2 = L_x^2 + L_y^2+ L_z^2
\end{equation} 
Jede der kartesischen Komponenten muss kleiner als die Länge sein, also 
\begin{equation}
    L_{x,y,z}^2 \le L^2
\end{equation}
Die Besonderheit des Drehimpulses in der Quantenmechanik ist nun, dass eine beliebige Komponente $L_{x,y,z}$ und die Länge $L$ zusammen gleichzeitig ohne Unschärfe gemessen werden können. Über die beiden anderen Komponenten kann man dann aber nichts mehr sagen, außer dass sie zusammen die richtige Gesamtlänge ergeben müssen. Wenn man also $L$ und $L_z$ gemessen hat, kann man nur noch sagen
\begin{equation}
    L_x^2 + L_y^2 = L^2 - L_z^2
\end{equation}
Die Aufteilung zwischen $L_x$ und $L_y$ ist jedoch nicht festgelegt. Man kann sie sich als einen Vektor $\bL$ vorstellen, dessen Spitze auf einem Kreis liegt, der durch Gl. xxx beschrieben wird, oder als einen Vektor, der einen Kegel beschreibt. Da $L_z$ ebenfalls quantisiert ist, gibt es $2l+1$ solcher Kreise bzw. Kegel.

Unabhängig davon, welche der drei kartesischen Komponenten gemessen wird, sind die beiden anderen immer innerhalb der genannten Grenzen unbestimmt. Typischerweise legt man das Koordinatensystem so an, dass die gemessene Komponente $L_z$ ist. In der Quantenmechanik ist es auch unerheblich, ob man diese Komponente tatsächlich misst oder nur messen kann\sidenote{wie beim Elektron im Doppelspalt}. Wie wir weiter unten sehen werden, ist der Drehimpuls mit einem magnetischen Moment verbunden, dessen Orientierung im Magnetfeld einen Energiebeitrag liefert. Die Orientierung eines äußeren Magnetfeldes definiert also die Richtung der z-Koordinate. Daher wird die Quantenzahl $m$ von $L_z$ auch als magnetische Quantenzahl bezeichnet.


\begin{marginfigure}
    \inputtikz{\currfiledir vector3d}
    \caption{Skizze eines Drehimpulsvektors mit unbekannter xy-Komponente.}
\end{marginfigure}
    


Eine Konsequenz der Quantisierung von $L$ und $L_z$ ist, dass der Vektor $\bL$ niemals exakt in z-Richtung orientiert sein kann. Es kann nie $L_z = L$ sein, weil $L_z$ ein ganzzahliges Vielfaches von $\hbar$ ist, $L$ aber diesen $\sqrt{l(l+1)}$-Term hat, der immer etwas größer als $l$ ist. Nur im Grenzfall sehr großer $l$ (im Korrespondenzprinzip) ist eine reine z-Orientierung möglich.

Weiterhin ist bemerkenswert, dass das Elektron im Grundzustand, d.h. im Zustand 1s, also $n=1$ und $l=0$, einen Bahndrehimpuls von Null hat. Ein klassisches Teilchen würde sich in diesem Fall überhaupt nicht auf einer geschlossenen Bahn bewegen. Für das Elektron in der Quantenmechanik ist das aber kein Problem.

\begin{marginfigure}
    \inputtikz{\currfiledir vector2d}
    \caption{Mögliche Orientierung von  Drehimpuls-artiger Vektoren mit $l=1/2$ (links) und $l=2$ (rechts). Der Abstand der Hilfslinien beträgt $1/2 \hbar$ bzw. $1\hbar$.}
\end{marginfigure}

Es gibt nicht nur Vektoren, die einem klassischen Drehimpuls entsprechen, sondern auch anderen Größen, die sich sehr ähnlich einem Drehimpuls verhalten, wie beispielsweise der Spin des Elektrons oder des Kerns. Bahndrehimpulse haben immer ganzzahlige Quantenzahlen $l$,$m$, Spins können auch halbzahlig sein. Immer ist der Abstand zwischen benachbarten Quantenzahlen aber eins. 



\section{Wellenfunktionen}

Die Wellenfunktion $\Psi$, die die Schrödingergleichung löst, lässt sich am einfachsten in sphärischen Koordinaten schreiben, also statt $x,y,z$ als Funktion von $r,  \theta,\phi$. Die Lösungen besteht aus   drei Teilen: die Normierung, der Radialanteil $R(r)$ und der Winkelanteil $Y(\theta, \phi)$, also 
\begin{equation}
    \Psi_{n,l,m} (r, \theta, \phi) = A_{n,m} \, R_{n,l}(r) \, Y_l^m (\theta, \phi)
\end{equation}
Die $R_{n,l}(r)$ sind dabei Laguerre-Polynome und die $Y_l^m(\theta, \phi)$ Kugelflächenfunktionen.
Die ersten Wellenfunktionen sind (XXX check, also def of aB)
\begin{align}
    n,l,m  & & A_{n,m}                      & &R_{n,l}(r) & & Y_l^m(\theta, \phi) \\
   1,0,0   && \frac{1}{\sqrt{\pi a_B^3}}    & &e^{-r / a_B} && \\
   %
    2,0,0  & & \frac{1}{4 \sqrt{2 \pi a_B^3}}   & & \left( 2 - \frac{r}{a_B}\right) e^{-r /2 a_B} && \\
    2,1,0  & & \frac{1}{4 \sqrt{2 \pi a_B^3}}   & & \frac{r}{a_B} e^{-r /2 a_B} && \cos \theta \\
    2,1,\pm 1  & & \frac{1}{8 \sqrt{\pi a_B^3}}   & & \frac{r}{a_B} e^{-r /2 a_B} && \sin \theta \, e^{\pm i \phi} 
    %
\end{align}


\section{Darstellung von 3d-Wellenfunktionen}

Im letzten Kapitel haben wir die eindimensionale Wellenfunktion $\Psi(x)$ entweder direkt auf der x-Achse aufgetragen oder als Wahrscheinlichkeitsdichte $P(x) = |\Psi(x)|^2$ dargestellt. Im Eindimensionalen ist dies leicht möglich. Im Dreidimensionalen ordnet die Wahrscheinlichkeitsdichte jedem Punkt $\br$ im Raum einen Wert zu. Dies ist schwierig darzustellen. Zum einen ist Papier oder ein Bildschirm immer nur zweidimensional. Zum anderen verdecken 'vordere' Werte die dahinter liegenden, wenn der Wert z.B. als Farbe kodiert ist.

Es gibt verschiedene Möglichkeiten, die jedoch alle ihre Vor- und Nachteile haben (Die Abbildungen XXX zeigen Beispiele):
\begin{description}
    \item[Punktwolke] Man kann die Detektion des Elektrons an verschiedenen Orten simulieren und in eine Punktwolke eintragen. Je näher die Punkte beieinander liegen, desto höher ist die Wahrscheinlichkeitsdichte. Die Gesamtzahl der Punkte muss klein gehalten werden, damit die Wolke noch einigermaßen durchsichtig ist. Am Computer kann man die Wolke drehen, um einen dreidimensionalen Eindruck zu erhalten.
    \item[Iso-Flächen] Man kann Flächen $P(\br) = |\Psi(\br)|^2 = const.$ zeichnen, ähnlich den Isobaren im Wetterbericht. Wenn die Konstante gut gewählt ist, erhält man eine Vorstellung von der Wellenfunktion. Diese kann dreidimensional dargestellt oder mit einer Ebene geschnitten werden.
    \item[Schnitte] Man kann $P(x,y)$ farbkodiert auf einer Ebene darstellen. Die Ebene enthält fast immer den Atomkern. Manchmal ist es hilfreich, eine zweite Ebene senkrecht dazu darzustellen.
    \item[Radiananteil] Anstatt zu fragen, wie groß die Wahrscheinlichkeit ist, das Elektron am Ort $\br$ zu finden, kann man fragen, wie groß die Wahrscheinlichkeit ist, es im Abstand $r$ vom Kern zu finden. Dies ist die \emph{radiale Wahrscheinlichkeitsdichte} $P_r(r)$.
    \begin{equation}
        P_r(r) = \iint P(r, \theta, \phi) \, d\theta d\phi = 4 \pi r^2 |R_{n,l}(r)|^2
    \end{equation}
    Der Term $4 \pi r^2$ berücksichtigt, dass mit zunehmendem Radius $r$ die Anzahl der Möglichkeiten und also das Volumen der Kugelschale zunimmt.
\end{description}


Die verschiedenen Darstellungen scheinen sich zu widersprechen. Alle s-Wellenfunktionen (also $l=0$) haben ein Maximum bei $P(\br = 0)$, aber eine Nullstelle bei $P_r(r=0)$. Der wahrscheinlichste Ort für ein Elektron ist daher der Kern. Gleichzeitig ist der wahrscheinlichste Abstand vom Kern weit von Null entfernt.  Dies sind jedoch zwei verschiedene Fragen. Für kleine Abstände reduziert die $r^2$-Abhängigkeit in $P_r$ die Wahrscheinlichkeit drastisch. Für große Abstände gibt es viel mehr Möglichkeiten auf der Kugeloberfläche. Jede dieser Möglichkeiten ist für sich jedoch unwahrscheinlicher als der Ort auf dem Kern.


Betrachtet man die radiale Aufenthaltswahrscheinlichkeit $P_r(r)$, so stellt man fest, dass die Wellenfunktionen mit dem größten $l$ bei gegebenen $n$, also 1s, 2p, 3d usw., ihre Maxima bei den von Bohr erwarteten Radien, also $1 a_B$, $4 a_B$ und $9 a_B$ haben. Die anderen Wellenfunktionen liegen mit einer gewissen Wahrscheinlichkeit weiter außen. Dies ist eine weitere Konsequenz des Korrespondenzprinzips. Für große Quantenzahlen $l$ werden die Elektronen immer klassischer, die Bahnen immer kreisförmiger, immer näher am Bohrschen Modell. Bei kleinen Bahndrehimpulsen ist die Bahn nicht mehr so kreisförmig, bis hin zur s-Wellenfunktion ohne Bahndrehimpuls.

Die Schrödingergleichung gibt also die von Bohr postulierten Bahnen wieder, aber nur als Maximum der Wahrscheinlichkeit, den Bahnradius zu messen. Mit einer gewissen Wahrscheinlichkeit sind auch andere Bahnradien möglich. Noch einmal: In der Quantenmechanik bewegt sich das Elektron nicht auf einer geschlossenen Bahn. Allerdings entspricht die Wahrscheinlichkeitsdichte, das Elektron in einem bestimmten Abstand vom Kern zu finden, dem klassischen Bahnmodell.



\section{Relativistische Korrekuten}

sishe Demtröder

Wenn man die Energien der Zustände sehr genau vernmitsst, ebsilswesie durch die optische Spektroksooie der ÜPbgernänge zwixhne ihnen, dann findet man leichet Abwehcingen von den durch Gl. XXX bescherieben Eerngien. Einige Beiträge dazu will ich hie rkurz aufführen

* Relativistische Masenzunahmen , geht mit Sommerfolf-Konsatnte approic v/c, spaltet l-entartung auf

* Darwin Term: Ppsition des Elektrons nicht genau bestimmt, nur bis auf de broglie Wellenlnge, Eleltkron 'sieht' viele Werte des Potentials. Hängt von psi(0) ab, evrscheibt also nur s-WF

* SPin Bahn-Kopplung, 

\section{Das Stern-Gerlach Experiment}

Ich greife hier dem Kapitel über Atome im Magnetfeld etwas vor.
Die Bahnbewegung des Elektrons kann als Strom aufgefasst werden und ist mit einem magnetischen Moment $\bmu$ verbunden.
\begin{equation}
    \bmu = \frac{e \hbar}{2 m_e} \,  \bL
\end{equation}
Die Messung des magnetischen Moments gibt also Auskunft über den Bahndrehimpuls $\bL$.
Um 1922 wollten Otto Stern und Walter Gerlach deshalb das magnetische Moment von Atomen messen und damit den Bahndrehimpuls bestimmen. Rückblickend haben sie damit den Spin des Elektrons gefunden.\footcite{SEP_Stern_gerlach}

Die potentielle Energie eines magnetischen Moments in einem magnetischen Feld beträgt
\begin{equation}
    U_B = - \bm \cdot \bB
\end{equation}
Daher wirkt eine Kraft $\bF_B$
\begin{equation}
    \bF_B = - \text{grad} ( U_B ) =  \text{grad} (\bm \cdot \bB ) = \mu_z \cdot \frac{\partial B}{\partial z}
\end{equation}
Im letzten Schritt wurde die übliche Annahme getroffen, dass das Magnetfeld in z-Richtung orientiert ist. Eine räumliche Änderung (in z-Richtung) des Magnetfeldes bewirkt somit eine Kraft auf einen magnetischen Dipol, die proportional zur z-Komponente des Dipols ist.

Dies kann man sich auch anschaulich vorstellen, wenn man sich den magnetischen Dipol als Stabmagnet vorstellt. Befindet sich der Nordpol des Magneten an einer größeren z-Koordinate, so ist die Kraft auf ihn größer als die Kraft auf den Südpol, weil das Feld mit z stärker wird. Daraus ergibt sich eine Nettokraft in positiver z-Richtung. 

Im Experiment erzeugten Stern und Glerach einen Gradienten im Magnetfeld, indem sie einen Polschuh des Magneten kleiner machten als den anderen, so dass dort die Feldlinien enger zusammenliefen und das Feld stärker war. Durch ein solches inhomogenes Magnetfeld ließen sie zunächst Silberatome und später Wasserstoffatome laufen. Dazu wird Silber in einem Ofen erhitzt und der austretende Silberdampf durch Blenden kollimiert, so dass ein atomarer Silberstrahl entsteht. Dieser wird in dem Magnetfeld abgelenkt und auf einer Glasplatte detektiert.

Bei ausgeschaltetem Magneten ergibt sich eine horizontale Linie, die der horizontalen Blende entspricht. Bei eingeschaltetem Magneten werden die Atome entsprechend ihrem magnetischen Moment vertikal abgelenkt. Für einen klassischen, nicht quantisierten Bahndrehimpuls wurde eine kontinuierliche Verteilung der Atome in vertikaler Richtung erwartet. Stern und Gerlach wählten Silber, weil sich hier alle Elektronen bis auf eines in voll besetzten Schalen befinden, wie wir im nächsten Kapitel sehen werden. Tatsächlich bleibt nur der Einfluss dieses letzten Elektrons übrig, der Rest hebt sich gegenseitig auf. Für ein Elektron im Zustand $l=1$ erwartet man 3 Linien, abhängig von der minimalen Quantenzahl $m=-1, 0, +1$. Ein solches Ergebnis würde bedeuten, dass der Bahndrehimpuls quantisiert ist.

\section{Der Spin}

Stern und Gerlach fanden nicht drei, sondern nur zwei Linien und entdeckten damit den Elektronenspin, bevor dieses Konzept überhaupt erfunden wurde.\sidenote{Details siehe \cite{SEP_Stern_gerlach}} Heute wissen wir, dass sich das relevante Elektron im Silberatom im Zustand $l=0$ befindet, der Bahndrehimpuls also keine Aufspaltung liefert. 

Ein Elektron hat eine Masse und eine Ladung, die für die Gravitationskraft und die Coulombkraft relevant sind. Es hat sich herausgestellt, dass ein Elektron noch eine weitere Eigenschaft besitzt, ein magnetisches Moment. Dieses innere magnetische Moment des Elektrons nennt man Spin.  Ein geladener Ball, der sich um sich selbst dreht, hätte solch ein magnetisches Moment, das mit dieser Drehung verbunden ist. Das Elektron dreht sich nicht wirklich, sondern nur in unserer Vorstellung.

Die Drehung des Elektrons um sich selbst ist mit einem Drehimpuls $\bS$, dem Spin, verbunden. Für Elektronen ist der Spin \begin{equation}
    S = | \bS | = \hbar \sqrt{s (s+1)} \quad \text{mit} \quad s = \frac{1}{2}
\end{equation}
Auch andere quantenmechanische Teilchen besitzen diesen eingebauten Drehimpuls, den Spin, der auch andere Längen $S$ und Quantenzahlen $s$ annehmen kann.

Analog zum Bahndrehimpuls ist auch die Orientierung des Spins quantisiert. Die Spanne der möglichen Werte ist wiederum $\hbar$, so dass nur die Werte 
\begin{equation}
   S_z = m_s \hbar \quad \text{mit} \quad m_s = \pm \frac{1}{2}
\end{equation}
möglich sind. Der Zustand $m_s = + 1/2$ mit $S_z = \hbar /2$ wird als 'spin up' bezeichnet, der andere als 'spin down', was manchmal durch $\uparrow$ und $\downarrow$ symbolisiert wird.\sidenote{Wie oben kann der Spin nicht vollständig in z-Richtung zeigen}.

XXX einsetin de haas


\section{Spin-Bahn-Kopplung}

\section{Addition von Drehimpulsen ?}



\section{Hyperfeinstuktur}

\section{Strahliungskorrektur = Lamb shift ?}

\section{Zusammenfassung}

\textit{Schreiben Sie hier ihre persönliche Zusammenfassung des Kapitels auf. Konzentrieren Sie sich auf die wichtigsten Aspekte.}

\vspace*{10cm}



%--------------------
\printbibliography[segment=\therefsegment,heading=subbibliography]
 

\renewcommand{\lastmod}{15. November 2024}
\renewcommand{\chapterauthors}{Markus Lippitz}

\chapter{Die restlichen Atome des Periodensystems}



\section{Überblick}

Nachdem wir uns im letzten Kapitel mit dem Wasserstoffatom beschäftigt haben, wollen wir nun alle anderen Atome des Periodensystems betrachten. Die etwas ungleiche Verteilung der Aufmerksamkeit auf die Elemente rührt daher, dass wir schon bei zwei Elektronen pro Atom mit starken Näherungen beginnen müssen. Die Elektronen stoßen sich gegenseitig ab, was nur mit großem Aufwand modelliert werden kann. Wir betrachten hier also eher allgemeine Prinzipien als konkrete Rechnungen.

Eine zentrale Rolle wird das Pauli-Prinzip spielen: Zwei ununterscheidbare Elektronen können nicht in allen Quantenzahlen übereinstimmen. Es können also nicht alle Elektronen im gleichen Zustand sein. Dies ist ein wesentlicher Beitrag dazu, dass sich die Elemente in ihren chemischen Eigenschaften unterscheiden.

Als Konsequenz des Pauli-Prinzips ergeben sich die Hundschen Regeln, Abkürzungen bei der Suche nach dem Zustand mit der minimalen Energie. Damit ist es möglich, für die meisten Elemente die Quantenzahlen aller Elektronen im Grundzustand des Atoms anzugeben.

Sowohl das Pauli-Prinzip als auch die Hundschen Regeln benutzen die Addition von Drehimpulsen oder drehimpulsähnlichen Größen wie dem Spin. Wir führen einen Gesamtdrehimpuls und einen Bahndrehimpuls ein, wobei die Quantenmechanik manchmal von der rein geometrischen Addition abweicht.

Dieses Kapitel folgt weitgehend dem Kapitel 8 von \cite{Harris_moderne_Physik}. Gut zu lesen ist auch \cite{Demtröder_ep3} und auch \cite{Heintze_WTA}.

\section{Zwei Teilchen in einem Kasten}

Alle Atome mit Ausnahme von Wasserstoff haben mehr als ein Elektron. Wir müssen also unser quantenmechanisches Modell auf zwei und mehr Teilchen erweitern. Dazu kehren wir, \cite{Harris_moderne_Physik} folgend, zunächst zum Teilchen-im-Kasten-Modell zurück, sperren nun aber zwei Teilchen in den Kasten ein. Wie bisher implizit angenommen, brauchen unsere Teilchen keinen Platz, sie schließen sich nicht gegenseitig aus. Jedes Teilchen $i=1,2$ befindet sich an seinem Ort $x_i$ (hier in einer Dimension). Jedes Teilchen besitzt eine kinetische Energie. Und es gibt ein Potential $U(x_1, x_2)$, das den Kasten für jedes Teilchen beschreibt. Später können wir damit auch die Coulomb-Abstoßung von Teilchen modellieren. Aber das brauchen wir jetzt noch nicht. Die Schrödingergleichung lautet
\begin{equation}
    \left(
        - \frac{\hbar^2}{2m} \, \frac{\partial^2}{\partial x_1^2} 
        - \frac{\hbar^2}{2m} \, \frac{\partial^2}{\partial x_2^2} 
     \right)
     \Psi(x_1, x_2)
     \, + \,
     U(x_1, x_2) \, \Psi(x_1, x_2)
     = E \, \Psi(x_1, x_2)
\end{equation}
mit der Wellenfunktion $\Psi(x_1, x_2)$, die nun von zwei Ortskoordinaten abhängt, nämlich den Positionen der beiden Teilchen.

Für einen Topf mit starren Wänden ergibt\sidenote{Rechnung in \cite{Harris_moderne_Physik}} sich
\begin{equation}
    \Psi(x_1, x_2) = \Psi_{n}(x_1) \, \Psi_{n'}(x_2) \quad
    \text{mit} \quad \Psi_n(x) = \sqrt{\frac{2}{L}} \, \sin \frac{n \pi x}{L}
\end{equation}
und ganzzahligen Quantenzahlen $n \ge 1$.

Die alten Zustände, Quantenzahlen, Potentiale und Wellenfunktionen, die wir für ein einzelnes Teilchen gefunden haben, nennen wir \emph{Einteilchen-Zustand} und \emph{Einteilchen-Wellenfunktion}. Die neuen, zusammengesetzten für mehrere Teilchen heißen entsprechend \emph{Mehrteilchen-Wellenfunktion}. $\Psi(x_1, x_2)$ ist eine Mehrteilchen-Wellenfunktion, die als Produkt der Einteilchen-Wellenfunktionen $ \Psi_n(x)$ geschrieben wird.

\subsection{Wahrscheinlichkeitsdichte}

Die Zweiteilchen-Wahrscheinlichkeitsdichte $P(x_a, x_b) = | \Psi(x_a, x_b)|^2$ beschreibt die Wahrscheinlichkeit, Teilchen 1 am Ort $x_a$ und Teilchen 2 am Ort $x_b$ oder in einem Intervall $dx$ um diese Orte herum zu finden.

Betrachten wir ein Beispiel. Die Quantenzahlen seien $n = 4$ und $n' = 3$. Damit ist 
\begin{equation}
    P(x_1, x_2) = |  \Psi(x_1, x_2)|^2 = \frac{4}{L^2} \, \sin^2 \frac{4 \pi x_1}{L} \, \sin^2 \frac{3 \pi x_2}{L} \quad .
\end{equation}
Die Wahrscheinlichkeit, das Teilchen 1 in der Mitte des Kastens bei $x_1 = L/2$ zu finden, ist Null, da $\Psi_4(x)$ dort einen Knoten hat, also 
\begin{equation}
    P\left(\frac{L}{2}, x_2\right) = \frac{4}{L^2} \, \sin^2 \frac{4 \pi (L/2)}{L} \, \sin^2 \frac{3 \pi x_2}{L}  = 0 \quad .
\end{equation}
Diese Wahrscheinlichkeit ist unabhängig von der Position $x_2$ des zweiten Teilchens. Die Wahrscheinlichkeit, das zweite Teilchen in der Mitte des Kastens zu finden, ist jedoch nicht durchgängig Null, da $\Psi_3(x)$ dort einen Bauch hat
\begin{equation}
    P\left(x_1, \frac{L}{2}\right) = \frac{4}{L^2} \, \sin^2 \frac{4 \pi x_1}{L} \, \sin^2 \frac{3 \pi L/2}{L}  \neq 0
\end{equation}
zumindest für manche Werte von $x_1$.


Es kann aber nicht sein, dass Teilchen 1 nicht in der Mitte des Kastens ist, Teilchen 2 aber schon, wenn sie ununterscheidbar sind! Wir können die Quantenteilchen nicht mit Namensschildern versehen. Da der Abstand zwischen den Teilchen beliebig klein sein kann, können wir einem Teilchen auch nicht immer folgen.

Diese Ununterscheidbarkeit der Quantenteilchen ist der zentrale Punkt. Die Wahrscheinlichkeitsdichte und andere beobachtbare Größen dürfen sich nicht ändern, wenn wir die Namen, die Quantenzahlen der ununterscheidbaren Teilchen vertauschen. Man sagt, dass die Wahrscheinlichkeitsdichte unter Vertauschung der Indizes symmetrisch sein muss.

\section{Symmetrische und antisymmetrische Wellenfunktion}

Die Symmetrie der Wahrscheinlichkeitsdichte erreichen wir durch Linearkombination der obigen Zweiteilchen-Wellenfunktion mit ihrer vertauschten Variante:
\begin{align}
    \Psi_S(x_1, x_2) = & \Psi_{n}(x_1) \, \Psi_{n'}(x_2) \, + \, \Psi_{n'}(x_1) \, \Psi_{n}(x_2) & \text{symmetrisch}
    \label{eq:6_sym_WF} \\
    \Psi_A(x_1, x_2) = & \Psi_{n}(x_1) \, \Psi_{n'}(x_2) \, - \, \Psi_{n'}(x_1) \, \Psi_{n}(x_2) & \text{anti-symmetrisch} 
    \label{eq:6_asym_WF} 
\end{align}
Im zweiten Summanden sind die Quantenzahlen gegenüber dem ersten vertauscht und damit auch der Name des Teilchens. Da die Schrödingergleichung linear ist, sind diese Linearkombinationen auch Lösungen der Schrödingergleichung.
Durch Ausmultiplizieren des Betragsquadrats stellt man fest, dass beide Wellenfunktionen zu einer Wahrscheinlichkeitsdichte führen, die sich bei Vertauschung der Indizes nicht ändert, also symmetrisch ist.


\begin{marginfigure}
    \inputtikz{\currfiledir kasten_sym}
    \caption{Wahrscheinlichkeitsdichten zu $ \Psi(x_1, x_2)$,  $ \Psi_S(x_1, x_2)$ und  $ \Psi_A(x_1, x_2)$.}
\end{marginfigure}

Die Abbildung  zeigt die Wahrscheinlichkeitsdichte $ P(x_1, x_2) $ für die erste, noch nicht symmetrisierte Wellenfunktion $ \Psi(x_1, x_2)$ sowie für die symmetrische $ \Psi_S(x_1, x_2)$ und die asymmetrische $ \Psi_A(x_1, x_2)$. Dies ist eine zweidimensionale Darstellung im $x_1$--$x_2$ Raum, aber der Kasten ist eindimensional mit der Achse $x$.


Die Wahrscheinlichkeitsdichten sind bis zu einem gewissen Grad ähnlich. Alle haben 12 Peaks. Bei der nicht symmetrisierten Variante kann man jedoch die Achse $x_1$ von der Achse $x_2$ unterscheiden.
Die symmetrisierten Varianten sind entlang der Linie $x_1 = x_2$ spiegelsymmetrisch. Für $\Psi_S(x_1, x_2)$ liegen die höchsten Peaks entlang dieser $x_1 = x_2$-Linie. Für $\Psi_A(x_1, x_2)$ liegen die höchsten Peaks in den anderen Ecken, d.h. wenn $|x_1 - x_2|$ am größten wird. Zwei Teilchen in der symmetrischen Zweiteilchen-Wellenfunktion neigen dazu, sich am gleichen Ort zu befinden. Zwei Teilchen in der antisymmetrischen Zweiteilchen-Wellenfunktion sind im Mittel sehr weit voneinander entfernt.


\section{Mehr Quantenzahlen und Spin}

Die Wahrscheinlichkeitsdichte muss immer symmetrisch sein, wenn die Teilchen vertauscht werden, unabhängig vom Potential und den zur Beschreibung notwendigen Quantenzahlen. Wir können jedoch die obige Schreibweise beibehalten, indem wir unter dem Index $n$ von $\Psi_n(x)$ die Menge aller benötigten Quantenzahlen verstehen. Im Wasserstoffatom umfasst $n$ also $\{ n, l, m_l, m_s \}$ und $\{ 1, 0, 0, \uparrow \}$ ist der Grundzustand. Insbesondere bezieht sich die Bezeichnung 'symmetrisch' oder 'antisymmetrisch' auf alle Quantenzahlen zusammen. Diese Sorte Symmetrie nennt man Austauschsymmetrie.



\section{Das Pauliprinzip}


Alle Quantenteilchen besitzen die Eigenschaft 'Spin', auch wenn dieser manchmal Null ist. Man unterscheidet zwischen \emph{Fermionen} mit halbzahligem Spin ($1/2$, $3/2$, $5/2$, usw.) und \emph{Bosonen} mit ganzzahligem Spin. Die Tabelle zeigt Beispiele. Bosonen haben eine symmetrische Mehr\-teil\-chen-Wellenfunktion (Gl. \ref{eq:6_sym_WF}) und Fermionen bilden eine antisymmetrische Mehrteilchen-Wellenfunktion (Gl. \ref{eq:6_asym_WF}). Wenn wir Atome mit mehr als einem Elektron beschreiben wollen, müssen wir dafür sorgen, dass die Mehrteilchen-Wellenfunktion anti-symmetrisch ist, denn Elektronen sind Fermionen, da sie den Spin $1/2$ besitzen.

\begin{marginfigure}
    \begin{tabular}{ll}
        Fermionen & \\
        \hline
        Elektron e$^-$ & $1/2$ \\
        Proton p & $1/2$ \\
        Neutron n & $1/2$ \\
        Neutrino $\nu$ & $1/2$ \\
        Omega $\Omega^-$ & $3/2$ \\
    & \\
    Bosonen & \\
    \hline
    Pion $\pi^0$ & $0$ \\
    $\alpha$-Teilchen & $0$ \\
   Photon $\gamma$ & $1$ \\
   Deuteron d & $1$ \\
    Graviton & $2$ \\
    \end{tabular}
    \caption{Beispiele für  Fermionen und Bosonen und deren Spin}
\end{marginfigure}

Die weitreichendste Konsequenz ist das nach Wolfgang Pauli benannte Pauli-Prinzip. Wenn zwei Elektronen in allen Quantenzahlen übereinstimmen, also $n = n'$ in unserer obigen Nomenklatur, dann ist die antisymmetrische Zweiteilchen-Wellenfunktion Null:
\begin{equation}
    \Psi_A(x_1, x_2) =  \Psi_{n}(x_1) \, \Psi_{n}(x_2) \, - \, \Psi_{n}(x_1) \, \Psi_{n}(x_2) = 0 \quad ,
\end{equation}
also kann  diesen Fall also nicht geben. Das ist das Ausschlussprinzip von Pauli: Zwei ununterscheidbare Fermionen können nicht im selben Einteilchenzustand sein, d.h. sie können nicht in allen Quantenzahlen übereinstimmen. Dieses Prinzip gilt für jedes Paar von Fermionen, wenn es mehr als zwei gibt.

Das Wort 'ununterscheidbar' ist wichtig. Ein Elektron und ein Proton können in den Quantenzahlen übereinstimmen. Auch zwei Elektronen können in den Quantenzahlen übereinstimmen, wenn sie nur weit genug voneinander entfernt sind, so dass sie nicht unbemerkt die Plätze tauschen können. 

Für Bosonen gilt das Pauli-Prinzip nicht. Das Plus in Gl. \ref{eq:6_sym_WF} stört nicht, wenn $n = n'$.

Eine Folge des Pauli-Prinzips ist, dass sich in einem Atom mit $Z$ Protonen und damit auch $Z$ Elektronen nicht alle Elektronen im Grundzustand $1s$ befinden können. Man füllt die Zustände in aufsteigender Reihenfolge auf, bis alle Elektronen untergebracht sind. Dabei darf kein Zustand doppelt besetzt sein, so dass wir die $Z$ energetisch niedrigsten Zustände besetzen. Die chemischen Eigenschaften eines Elements ergeben sich dann aus den letzten besetzten und den ersten unbesetzten Zuständen des Atoms.


\section{Näherung der unabhängigen Elektronen}

Wie findet man diese energetisch niedrigsten Zustände? Zunächst müssen wir berücksichtigen, dass die Kernladung nun $Z$ ist und nicht mehr eins wie beim Wasserstoffatom. Überall dort, wo im letzten Kapitel ein Term $e^2$ vorkam, müsste man nun $Z \, e^2$ schreiben.\sidenote{Das ist im POtential $U$, den Eigneenergien $E_n$, dem Bohr'schen Radis $a_B$ und der Feinstrukturkonstanten $a$} Damit ist 
\begin{equation}
    E_n \propto Z^2  \quad .
\end{equation}
Das  berücksichtigt aber nur die Ladung im Kern. Die Ladung der Elektronen, die zum Wasserstoff hinzukommen, ist aber auch $Z-1$, also nicht wirklich kleiner.\sidenote{Das ist der Unterschied zu einem Planetensystem. Dort ist   die Masse aller Planeten zusammen viel kleiner als die der Sonne.} Wir können sie nicht vernachlässigen. Eigentlich müssten wir die Schrödingergleichung mit dem Potential
\begin{equation}
    U(r_1, r_2, \dots, r_Z) = -\frac{e^2}{4 \pi \epsilon_0} \, \sum_{i=1}^Z  \left[  \frac{Z}{r_i} - \sum_{j=1}^{i-1}  \frac{1}{|r_i - r_j|} \right]
\end{equation} 
lösen. Das ist näherungsweise möglich, aber nicht einfach und Thema der Vielteilchen-Quantenmechanik. Hier machen wir, wie fast überall, die \emph{Näherung der unabhängigen Elektronen}. Wir nehmen an, dass sich jedes Elektron in einem effektiven Potential bewegt, das aus der Kernladung und einer Wolke der übrigen Elektronen besteht. Diese Elektronenwolke schirmt den Kern etwas ab. Ein sehr weit außen befindliches Elektron erfährt dann quasi nur noch eine einzige positive Restladung, alles andere ist abgeschirmt. Ganz innen sieht das Elektron dann einen Kern der Ladung $Z$. Das abgeschirmte Coulombpotential ist also
\begin{equation}
    U(r) =  -\frac{e^2}{4 \pi \epsilon_0} \, \frac{1}{r} \cdot
    \left\{
 \begin{matrix}
     Z \quad  & \text{für} \quad  r  \rightarrow  0 \\
     1     & \text{für} \quad  r \rightarrow \infty \\
 \end{matrix}
    \right. \quad .
\end{equation}
Alle Abhängigkeiten der Form $|r_i - r_j|$ werden vernachlässigt. Die Abbildung \ref{fig:6_coulomb_schirm}
zeigt ein solches Potential für Lithium. Das 2s-Elektron erfährt außen ein Potential, das dem des Wasserstoffatoms sehr ähnlich ist ($\propto -1/r$). Nach innen geht es jedoch in das von \ch{Li^{2+}} über, also einem dreifach positiv geladenen Kern ($\propto - 3/r$). Dies kann über eine effektive Kernladungszahl $Z_\text{eff}(r)$ modelliert werden.


\begin{marginfigure}
    \inputtikz{\currfiledir coulomb_schirm}
    \caption{Abgeschirmtes Coulombpotential am Beispiel von \ch{Li}. }
    \label{fig:6_coulomb_schirm}
\end{marginfigure}

Wie beim Wasserstoffatom handelt es sich um ein Zentralpotential. Die Winkelabhängigkeit der Lösungen ist daher identisch. Der Radialanteil der Wellenfunktion ändert sich etwas. Die Eigenenergien verschieben sich quadratisch mit steigendem $Z$ zu negativeren Werten, da das Potential viel negativer wird. Gleichzeitig werden aber auch mehr Elektronen eingefüllt, so dass der höchste besetzte Zustand mit steigendem $Z$ weiter vom Grundzustand entfernt ist. Wir werden sehen, dass sich diese beiden Effekte nahezu ausgleichen.


\section{Einfluss des Bahn-Drehimpulses}

Im vorigen Kapitel haben wir gesehen, dass die Wellenfunktionen mit dem maximalen $l$ bei gegebener Hauptquantenzahl $n$, also $l=n-1$, am kreisförmigsten sind und zu einem Maximum in der radialen Aufenthaltswahrscheinlichkeit führen, das gut durch den zugehörigen Bohrschen Bahnradius $n \cdot a_B$ beschrieben wird. Wellenfunktionen mit kleinerem $l$ beschreiben Elektronen mit elliptischeren Bahnen, die sich mit größerer Wahrscheinlichkeit weiter innen befinden. Zustände mit kleinem $l$ sind daher energetisch niedriger als Zustände mit großem $l$. Diese Unterschiede sind größer als die der im letzten Kapitel beschriebenen Spin-Bahn-Kopplung, die ebenfalls die $l$-Entartung im Wasserstoffatom aufhebt.

Berechnet man die Summe der Aufenthaltswahrscheinlichkeiten aller Einteilchen-Elektronenzustände mit gegebenem $n$, so ist diese kugelsymmetrisch und wird von den Elektronen mit maximalem $l$ dominiert, d.h. von denen nahe  der Bohr'schen Bahn. 
Man sagt, dass diese Elektronen mit der gleichen Hauptquantenzahl $n$ eine \emph{Schale} bilden, die manchmal mit den großen Buchstaben K, L, M, N bezeichnet wird.  Aufgrund der gerade beschriebenen Drehimpulsabhängigkeit spaltet sich jede Schale in $n$ Unterschalen auf, die manchmal mit römischen Ziffern bezeichnet werden. In jede Unterschale passen nach dem Pauli-Prinzip $2 (2l +1)$ Elektronen, da es $2l +1$ mögliche Werte der magnetischen Quantenzahl $m_l$ gibt und dann noch je zwei Möglichkeiten für die Orientierung $m_s = \pm 1/2$ des Spins. In jeder Schale befinden sich $2n^2$ Zustände.

Empirisch zeigt sich, dass die energetische Anordnung der Unterschalen und damit die Reihenfolge des Auffüllens der Elektronen mit steigendem $n+l$ erfolgt. Bei Unterschalen mit gleichem $n+l$ werden die mit kleinerem $n$ zuerst aufgefüllt. Dies ist in Abbildung  \ref{fig:6_state_3d}. Man erkennt auch, dass sich die Reihenfolge der Unterschalen mit der Kernladungszahl ändert, unsere einfachen Regeln also nicht alles erfassen.

%XXX mark filled states in fig states-3d ?

\begin{marginfigure}
    \inputtikz{\currfiledir state_3d}
    \caption{Schematische Verschiebung der Zustände mit steigender Kernladungszahl. Zwischen der 4s und 4p-Schale wird bei den Elementen  \ch{Sc} ($Z=21$) bis \ch{Zn} ($Z=30$) die 3d-Schale gefüllt.}
    \label{fig:6_state_3d}
\end{marginfigure}

\section{Das Periodensystem der Elemente}

Die chemischen Eigenschaften der Elemente werden wesentlich durch die Anzahl und die Bindungsenergie der Elektronen in der noch nicht vollständig gefüllten Schale bestimmt. Diese Elektronen werden als Valenzelektronen bezeichnet, da sie die Wertigkeit des Elements bestimmen. Sehr häufig ist dies auch die äußerste, also die energiereichste Schale. Da jede Unterschale nur $2 (2l +1)$ Elektronen aufnehmen kann, sind die chemischen Eigenschaften periodisch in der Kernladungszahl $Z$. Daraus ergibt sich das Periodensystem der Elemente. Abbildung \ref{fig:6:_PSE_states} zeigt die experimentell bestimmte energetische Lage und Besetzung der Zustände der ersten Elemente des Periodensystems. Die Energie besetzter Zustände wird bestimmt, indem ein Elektron durch ein Röntgenphoton aus einem solchen Zustand herausgeschlagen wird (XPS). Die Energie unbesetzter Zustände kann bestimmt werden, indem ein Valenzelektron zu einem solchen Zustand angeregt und die Übergangsenergie bestimmt wird.

\begin{figure}
    \inputtikz{\currfiledir PSE_states}
    \caption{Verlauf der Zustände und deren Besetzung.}
    \label{fig:6:_PSE_states}
\end{figure}

\begin{description}
    \item[Wasserstoff (\ch{H})] Das Wasserstoffatom besitzt nur ein Elektron, das sich im 1s-Zustand bei -13,6~eV befindet.
  
  \item[Hellium (\ch{He})] Helium hat zwei Protonen. Die Energie des 1s-Zustandes wäre daher eigentlich um den Faktor $n^2 = 4$ negativer als die des Wasserstoffs, also -54,4~eV. Die beiden Elektronen stoßen sich aber ab, so dass sie im Potential weiter außen liegen, also etwas weniger negativ. Damit liegt der 1s-Zustand bei etwa -24.6 eV. Man schreibt die \emph{Elektronenkonfiguration} als 1s$^2$, wobei die hochgestellte Zahl die Anzahl der Elektronen in der 1s-Schale angibt. Nach dem Pauli-Prinzip haben die beiden Elektronen entgegengesetzten Spin. Helium ist ein Edelgas, da seine voll besetzte 1s-Schale energetisch relativ weit von der Ionisationsgrenze bei $E=0$ entfernt ist.
   
    \item[Lithium (\ch{Li})] Da die 1s-Schale voll besetzt ist, muss das dritte Elektron in die 2s-Schale gehen. Lithium hat also nur ein Valenzelektron, das nur schwach gebunden ist (5.4 eV). Lithium ist daher chemisch sehr reaktionsfreudig.
    
    \item[Kohlenstoff (\ch{C})]   Kohlenstoff hat die Elektronenkonfiguration 1s$^2$2s$^2$2p$^2$ oder abgekürzt [He]2s$^2$2p$^2$. In der 2p-Schale befinden sich also 2 Elektronen. Während bei zwei Elektronen in einer s-Schale der Spin immer entgegengesetzt sein muss, ist dies in einer p-Schale nicht mehr notwendig, es gibt sogar 3 verschiedene Werte für $m_l$. Die \emph{Hund'sche Regel} bestimmt, in welcher Reihenfolge diese insgesamt 6 Zustände der p-Schale besetzt werden: Zunächst haben alle Elektronen die gleiche Spinausrichtung, erst das 4. Elektron (bei Sauerstoff) hat die entgegengesetzte. Wie wir später sehen werden, wird dadurch der Gesamtspin maximiert. Die Hund'sche Regel ist nur eine Abkürzung für eine Energiebetrachtung. Wenn die Elektronen den gleichen Spin haben, sind sie viel weiter voneinander entfernt (siehe Beispiel am Anfang des Kapitels). Dadurch verringert sich die Energie ihrer gegenseitigen Coulomb-Abstoßung.
    
    \item[Fluor (\ch{F})] Beim Fluor befinden sich 7 Elektronen in der 2p-Schale. Für die Edelgaskonfiguration fehlt noch eines. In chemischen Reaktionen nimmt Fluor daher gerne ein weiteres Elektron auf, um die Schale zu vervollständigen.

    \item[Neon (\ch{Ne})]  Bei Neon ist die 2p-Schale vollständig besetzt. Auch dies ist eine der stabilen Edelgaskonfigurationen.
    
    \item[Natrium (\ch{Na})] Bei Natrium wiederholt sich eigentlich alles analog zu Lithium. Auch hier befindet sich nur ein Valenzelektron in einer s-Schale, hier einer 3s-Schale. Hauptgruppe geben bei chemischen Reaktionen gerne Elektronen ab.
\end{description}

Diese Periodizität findet sich in vielen chemischen Eigenschaften wieder. Abbildung \ref{fig:6_E_ionisation}
zeigt sie für die Ionisierungsenergie, d. h. den Abstand des höchsten besetzten Zustandes vom Vakuum. Eine ähnliche Periodizität findet sich auch für das Atomvolumen.

\begin{marginfigure}
    \inputtikz{\currfiledir E_ionisation}
    \caption{Ionisationsenergie der ersten Elemente (Daten aus dem Julia-Paket Mendeleev.jl). Die gefüllten Schalen der Edelgase sind besonders stabil.}
    \label{fig:6_E_ionisation}
\end{marginfigure}


\section{Viele Drehimpulse}


Für das Wasserstoffatom haben wir die Spin-Bahn-Kopplung besprochen, d.h. den Energiebeitrag des Spins des Elektrons im Magnetfeld, der sich aus der Bahn ergibt. Bei Mehrelektronenatomen hat jedes Elektron einen Spin $\bs_i$ und einen Bahndrehimpuls $\bl_l$. Jede Kombination $\bs_i \cdot \bl_j$ liefert einen Energiebeitrag. Das ist zunächst unübersichtlich. Glücklicherweise lassen sich viele Atome sehr gut durch einen der beiden einfachen Grenzfälle beschreiben. Dazu müssen wir zunächst den Gesamtspin und den Gesamtdrehimpuls einführen und auch die Addition von Drehimpulsen in der Quantenmechanik etwas genauer betrachten.


\section{Addition von Drehimpulsen}

Im letzten Kapitel haben wir den Drehimpuls $\bj$ als $\bj = \bl + \bs$ eingeführt und die Eigenschaften von $\bj$ beschrieben. Hilfreich war dabei, dass der Spin eines einzelnen Elektrons nur in zwei Richtungen zeigen kann, nach oben oder nach unten. Der (bald eingeführte) Gesamtspin $\bS$ kann größer sein und damit verschiedenere Richtungen einnehmen. Deshalb müssen wir die Addition von Drehimpulsen etwas genauer betrachten. Ich verwende hier die Variablen $\bJ = \bL + \bS$, also die Großbuchstaben des bald eingeführten Gesamtspins. Das funktioniert aber auch mit zwei beliebigen Drehimpulsvektoren.

Angenommen, wir kennen von zwei Vektoren $\bS$ und $\bL$ die Länge (Quantenzahl $S$ und $L$) und die z-Komponente (Quantenzahl $m_S$ und $m_L$). Was können wir damit über den Vektor $\bJ$ und seine Quantenzahlen $J$ und $m_J$ aussagen? Der Anhang \ref{chap:anhang_drehimpuls}
gibt die Antwort etwas ausführlicher im Formalismus der Quantenmechanik. Hier die Kurzfassung:

Die Längen aller drei Vektoren $\bS$, $\bL$ und $\bJ$ können gleichzeitig ohne Unschärfe gemessen werden, zusammen aber nur die Orientierungsquantenzahl $m_J$ des Summenvektors. Die Orientierungsquantenzahlen $m_S$ und $m_L$ unterliegen nach der Messung von $J$ einer Unschärfe, ähnlich wie im letzten Kapitel die x- und y-Komponente des Bahndrehimpulses $l_x$ und $l_y$.

Die neue Orientierungs-Quantenzahl $m_J$ ist gerade die Summe der Einzeln-Orientierungs-Quantenzahlen
\begin{equation}
 m_J  = m_L + m_S \quad .
\end{equation}
Wenn $m_L$ und $m_S$ bekannt sind, kann $m_J$ berechnet werden. Kennt man aber nur $m_J$, dann ist die Aufteilung in $m_L$ und $m_S$ unbestimmt.  Für die neue Gesamt-Längenquantenzahl $J$ gilt
\begin{equation}
 | L - S | \le J \le  L + S \quad .
\end{equation}
Mehr lässt sich dazu leider nicht sagen. Es ist etwas unbefriedigend, die Summe von zwei Vektoren nicht nennen zu können, obwohl man beide Summanden kennt. Allerdings kennt man die Ausgangs-Vektoren nicht vollständig. Die unbekannte xy-Komponenten sind gerade der Ursprung dieses Spielraums im Wert von $J$.

\subsection{'Gute' Quantenzahlen}

Eine Quantenzahl wird als 'gute' Quantenzahl bezeichnet, wenn sie eine Konstante der Bewegung ist, also erhalten bleibt. Im Fall der Drehimpulsaddition sind dies $L$, $S$, $J$ und $m_J$, aber nicht mehr $m_S$ und $m_L$. Diese waren früher gute Quantenzahlen, bevor wir einen Energiebeitrag proportional zu $\bS \cdot \bL$ erlaubten. Wenn wir das zulassen, dann ist die Aufteilung zwischen $m_S$ und $m_L$§ nicht mehr zeitlich konstant, und diese Quantenzahlen sind nicht mehr 'gut', sie helfen nicht mehr, das System zu beschreiben. Die Gesamtzahl der guten Quantenzahlen bleibt jedoch erhalten. Vor der Kopplung waren es noch $m_S$ und $m_L$, nachher $J$ und $m_J$, jeweils zusätzlich zu $L$ udn $S$. Ebenso muss durch die Kopplung die Anzahl der Zustände, d.h. die Anzahl der möglichen Kombinationen von Quantenzahlen, erhalten bleiben.



\subsection{Geometrische Interpretation}

Bei der Kopplung von Spin und Bahndrehimpuls gibt es einen Energiebeitrag des Spins im Magnetfeld der Bahnbewegung. Klassisch würde dieser vom Winkel zwischen den beiden abhängen. Dieser Winkel ist aber nicht die Quantenzahl, sondern das sich aus $\bS$, $\bL$ und $\bJ$ bildende Dreieck wird vollständig durch die Längen der Seiten bestimmt. Das beinhaltet den Winkel zwischen $\bS$ und $\bL$, aber auch deren Amplitude. Gleichzeitig ist nur $m_J$ eine gute Quantenzahl. Bei einer Messung wie im Stern-Gerlach-Experiment spielt also nur die Orientierung von $\bJ$ eine Rolle. Die Spitze von $\bJ$ kann wieder auf einem Kreis in der xy-Ebene liegen, solange die Länge von $\bJ$ erhalten bleibt. Bei $\bS$ und $\bL$ ist nun aber \emph{nur} die Länge eine gute Quantenzahl, die z-Komponenten nicht mehr. Die Spitze von $\bS$ kann damit auf einem Kreis liegen, dessen Symmetrieachse durch $\bJ$ gegeben ist. Alles andere ist unbekannt, kann nicht gleichzeitig gemessen werden. %Insbesondere ist die Aufteilung zwischen $m_S$ und $m_L$ nicht fix, nur die Summe, also $m_J$.

%XXX skizze Kegel L+S




\section{Gesamtspin}

Nun soll endlich der Gesamtspin $\bS$ eingeführt werden. Er ist die Summe der Spins aller Elektronen eines Atoms
\begin{equation}
    \bS = \sum_{i=1}^Z \, \bs_i \quad . \label{eq:6_S_sum}
\end{equation}
Analog gilt das auch für den Gesamt-Bahndrehimpuls $\bL$. Betrachten wir hier zunächst einmal nur zwei Elektronen $i =1, 2$. Es ist 
\begin{equation}
s_{1,2} = \frac{1}{2} \quad \text{und} \quad m_{s,1,2} = \pm \frac{1}{2}  \quad .
\end{equation}
Welche Werte können nun die Quantenzahlen  $S$ und $m_S$ der Summe annehmen? Die magnetische Quantenzahl $m_S  = m_{s,1} + m_{s,2}$ ist einfach und in nebenstehender Tabelle skizziert.
%
\begin{marginfigure}
\begin{tabular}{r|rr}
                           & $-\frac{1}{2} $  & $+\frac{1}{2} $ \\
                           \hline
 $+\frac{1}{2} $    &     $0$              & $1$ \\
 $-\frac{1}{2} $    &     $-1$              & $0$ 
\end{tabular}
\vspace*{2mm}
\caption{Die möglichen Kombinationen von $m_{s,1}$ und $m_{s,2}$ zu $m_S  = m_{s,1} + m_{s,2}$.}
\end{marginfigure}

Falls $m_{s,1} = m_{s,2}$, also $|m_S| = 1$, dann muss auch $S = 1$ sein, da $S$ nie kleiner als $m_S$ sein kann. Dies sind die Zustände  $(S, m_S) = (1,1) = \uparrow\uparrow$ und  $(S, m_S) = (1,-1) = \downarrow\downarrow$. Die Pfeile zeigen in dieser Darstellung die  $m_{s,i}$-Quantenzahlen als up oder down an.

Damit verbleiben noch die beiden Fälle $m_{s,1} = - m_{s,2}$, also die Diagonale in der Tabelle. Daraus müssen sich die verbleibenden Kombinationen von $S$ und $m_S$ ergeben, nämlich $(S, m_S) = (0,0)$ und $(1,0)$. Die Gesamtzahl der Zustände passt schon einmal. Wie oft in der Quantenmechanik, wenn die Zuordnung nicht einfach entschieden werden kann, werden hier wieder die symmetrische und antisymmetrische Superposition der Ausgangszustände, also der Einträge in der Matrix, gebildet. Welche davon wird  $(S, m_S) = (1,0)$? Die schon gefundenen Zustände $(S, m_S) = (1,\pm1) $ sind symmetrisch bei Vertauschen $1 \leftrightarrow 2$, also wird auch$(S, m_S) = (1,0)$ symmetrisch sein, also 
\begin{equation}
   \frac{1}{\sqrt{2}} \left(  \uparrow \downarrow  +  \downarrow \uparrow \right) \quad .
\end{equation}
Damit gibt es einen anti-symmetrischen Zustand mit $S = 0$, 
\begin{align}
    (S, m_S) = (0, 0)  = & \frac{1}{\sqrt{2}} \left(  \uparrow \downarrow -  \downarrow \uparrow \right) 
\end{align}
und drei symmetrische mit $S=1$
\begin{align}
    (S, m_S) =  (1, +1) =& \uparrow \uparrow  \\
  (1, 0) = & \frac{1}{\sqrt{2}} \left( \uparrow \downarrow +  \downarrow \uparrow \right) \\
  (1, -1) = &\downarrow \downarrow   \quad .
\end{align}
Da es nur einen anti-symmetrischen Zustand gibt, wird dieser als \emph{Singulett} und die anderen als \emph{Triplett} bezeichnet. Generell gibt es, wie immer bei Drehimpulsen, bei einer Quantenzahl $S$ insgesamt $2S+1$ mögliche Werte von $m_S$, also mögliche Zustände.

\subsection{Geometrische Interpretation}

Wie kann man sich vorstellen, dass die Addition von zwei Vektoren gleicher Länge aber unterschiedlicher Orientierungs-Quantenzahl $m_i$ einmal zu einem Vektor der Länge Null und einmal zu einem Vektor der beinahe doppelten Länge führt? Ein Teil der Wahrheit sind die nicht gleichzeitig messbaren anderen Vektor-Komponenten.\sidenote{Ein anderer Teil ist 'so ist die QM eben'.} Die Spitze beider Vektoren liegt auf eine Kreis. Wenn die Position 'in Phase' ist, dann addieren sie sich zu einem Vektor mit verschwindender z-Komponente und der Länge $\hbar \sqrt{2}$, was in diesem Bild dem Zustand $(S, m_S) = (1,0)$ entspricht. Wenn die beiden Ausgangs-Vektoren 'außer Phase' sind, dann addieren sie sich zu Null, ergeben also  $(S, m_S) = (0,0)$. Bei bekannten, aber unterschiedlichen $m_i$, also beispielsweise $\uparrow \downarrow$ ist also nicht eindeutig, welcher Summenvektor sich ergibt. Die Eigenfunktionen des Summen-Operators $\hat{S}$ sind nur Linearkombinationen aus $\uparrow \downarrow$ und $\downarrow \uparrow$.

\begin{marginfigure}
\inputtikz{\currfiledir vector3d_summe}
\vspace*{2mm}

\caption{Die Addition von zwei Vektoren $s=1/2, m_s = 1/2$ und  $s=1/2, m_s = -1/2$ kann sowohl einen Vektor   $S=1, m_S = 0$ ergeben (links) als auch $S=0, m_S = 0$ (rechts).}
\end{marginfigure}

\subsection{Volle Schalen}

In der Gleichung \ref{eq:6_S_sum} hatte ich für den Gesamtspin die Summe über alle Elektronen des Atoms gebildet. Das ist zwar richtig, aber unnötig. Zum Glück ist der Gesamtspin bei vollständig gefüllten Schalen gleich Null. Der Gesamtdrehimpuls ist ebenfalls Null. Das liegt daran, dass die möglichen Werte von $m_S$ und $m_L$ immer symmetrisch um Null sind. Wenn alle beitragen, heben sich die beiden Vorzeichen gegenseitig auf und die Summe ist Null. Daher müssen zur Berechnung des Gesamtspins $\bS$ und des Gesamtbahndrehimpulses $\bL$ nur die Valenzelektonen, also nur die teilweise gefüllten Schalen, betrachtet werden. Die Edelgase haben also im Grundzustand $S=L=0$.


\section{LS-Kopplung und Hund'sche Regeln}

Nun haben wir alle Bausteine beisammen und können diskutieren, wie die vielen Spin- und Bahndrehimpulse eines Mehrelementkerns miteinander wechselwirken. Bei 'leichten' Atomkernen folgt dies meist der LS-Kopplung. Die Hierarchie der Wechselwirkungsenergien ist dabei so, dass zunächst alle Valenzelektronen einen Gesamtspin $\bS$ und einen Gesamtbahndrehimpuls $\bL$ bilden. Die Wechselwirkung zwischen den Spins ist am stärksten. Das Atom versucht, den Gesamtspin zu maximieren, d.h. möglichst viele Elektronen in den gleichen Spinzustand zu bringen, z.B. $\uparrow$. Erst wenn es keine weiteren Zustände mit $m_s = +1/2$ mehr gibt, werden die Zustände mit $m_s = -1/2$ besetzt, wodurch $S$ wieder kleiner wird. Dies ist die \emph{erste Hundsche Regel}, aber eigentlich eine Folge der Columb-Abstoßung der Elektronen. Wenn die Spins gleich sind, ist die Wellenfunktion so, dass die Elektronen sich an Orten befinden, die weiter voneinander entfernt sind.

Der nächste Schritt in der Energiehierarchie ist die Wechselwirkung der Bahndrehimpulse. Die zweite Hundsche Regel fordert ein maximales $L$. Dies ist wiederum auf die Coulomb-Abstoßung der Elektronen zurückzuführen. Wenn sich die $m_l$ der Wellenfunktionen unterscheiden, dann tendieren die Elektronen dazu, weiter voneinander entfernt zu sein, wie z.B. bei den 3D-Isoflächen im letzten Kapitel.

Erst danach folgt die Spin-Bahn-Kopplung mit dem kleinsten Energiebeitrag. Hier orientiert sich der Gesamtspin $\bS$ relativ zum Gesamtbahndrehimpuls $\bL$. Deshalb wird die LS-Kopplung mit Großbuchstaben geschrieben. Die Spin-Bahn-Kopplung bevorzugt ein kleines $J$, wenn die Schale weniger als halb gefüllt ist. Wenn die Schale mehr als halb gefüllt ist, wird ein möglichst großes $J$ bevorzugt. Dies ist die dritte Hund'sche Regel und stammt aus der Energie  proportional zu $\bS \cdot \bL$.

\section{jj-Kopplung}

Der Vollständigkeit halber sei hier noch die jj-Kopplung erwähnt. Sie tritt bei 'schweren' Atomen auf. Mit zunehmender Kernladungszahl $Z$ nimmt das Magnetfeld am Ort des Elektrons zu, das durch die scheinbare Kernbewegung verursacht wird. Damit gewinnt die Spin-Bahn-Kopplung in der Energiehierarchie an Bedeutung. In diesem Fall ist die Kopplung zwischen dem Spin $\bs$ eines einzelnen Elektrons und seinem Bahndrehimpuls $\bl$ der wichtigste Energiebeitrag. Diese beiden koppeln für jedes Elektron zu einem Einzel-Elektron-$\bj$. Erst danach koppeln die vielen $j$ der Elektronen zu einem Gesamt-$J$, um die Abstoßung der Elektronen zu verringern. Deshalb schreibt man jj-Kopplung klein.


\section{Beispiel: Helium}

Helium hat zwei Elektronen, die sich im Grundzustand beide im Ein-Teilchen-Zustand 1s befinden. Der Gesamtdrehimpuls ist $L=0$, da beide $l=0$ sind.

Wie groß ist hier der Gesamtspin $\bS$? Ein Argument ist, dass volle Schalen nichts beitragen, also ist $S=0$. Das andere ist, dass die Gesamtwellenfunktion wegen des Pauli-Prinzips anti-symmetrisch sein muss. Da der Ortsanteil für beide Elektronen gleich ist, nämlich die 1s-Wellenfunktion, muss der Spinanteil zur Antisymmetrie beitragen. Der antisymmetrische Spinzustand ist also der Singulettzustand mit $S=0$.

Das Gesamt-$J$ kann dann nur noch $J  = L + S = 0$ sein.  Analog zu den Ein-Elektron-Zuständen, die als '1s' usw. geschrieben werden, schreibt man die Mehrelektronenzustände als Termsymbol
\begin{equation}
    n^{2S + 1}L_{J}
\end{equation}
wobei für $L$ die gleichen Buchstaben wie für $L$ verwendet werden, jedoch in Großbuchstaben. Man gibt nicht $S$ an, sondern die Multiplizität $2S+1$. Daher auch die Bezeichnungen Singulett (1) und Triplett (3). Der Grundzustand von Helium ist\sidenote{sprich 1-Singulett-S-0} also $1^1S_0$. Einen Triplett-Grundzustand $1^3S_0$ kann es nicht geben.

In den energetisch niedrigsten angeregten Zuständen ändert nur ein Elektron seine Quantenzahl, was auf die sehr große Energielücke von 1s bis 2s zurückzuführen ist. Um zwei Elektronen nach 2s anzuregen, muss wesentlich mehr Energie aufgewendet werden, als um ein Elektron zu ionisieren.

Betrachten wir den Zustand, in dem ein Elektron in 1s bleibt und das zweite nach 2p angeregt ist. Dann muss $L=1$ sein. Da sich die Elektronen nun in zwei verschiedenen Ortswellenfunktionen befinden, können die Spins sowohl symmetrisch (Triplett) als auch antisymmetrisch (Singulett) angeordnet sein, also $S=0$ oder $S=1$. Die Spin-Bahn-Wechselwirkung führt dann zu einem $J$, das in Einerschritten zwischen $|L - S|$ und $L+S$ liegen muss. Die Tabelle zeugt die sich daraus ergebenden Termsymbole.

\begin{marginfigure}
   \begin{tabular}{llll}
    $L$ & $S$ & $J$ & Symbol \\
    1   & 0  &  1 & $2^1P_1$ \\
    1   & 1 &  0 & $2^3P_0$ \\
    1   & 1  &  1 & $2^3P_1$ \\
    1   & 1  &  2 & $2^3P_2$ \\
\end{tabular} 
\vspace*{2mm}

\caption{Mögliche Zustände ausgehend von Einteilchenzustand 1s2p.}
\end{marginfigure}

Wie wir später sehen werden, können optische Übergänge nur zwischen zwei Zuständen gleicher Multiplizität stattfinden. Der Gesamtspin $S$ darf sich dabei nicht ändern. Da der Grundzustand ein Singulett-Zustand ist, können die Zustände $2^3P_J$ noch in den tieferen $2^3S_1$ übergehen, aber nicht weiter. Dieser untere Triplettzustand $2^3S_1$ ist also metastabil, d.h. über einen längeren, aber nicht unendlich langen Zeitraum stabil.


\section{Weitere Beispiele}

Betrachten wir einige weitere Beispiele für die Anwendung der Hundschen Regel und der zugehörigen Termsymbole. Wie wir im Kapitel über die Wechselwirkung mit Licht sehen werden, sind die Symbole so gewählt, dass sie die spektroskopisch relevanten Informationen enthalten, also gut anwendbar sind.

\begin{description}\setlength{\itemsep}{0pt}

    \item[Lithium] Die Elektronenkonfiguration ist [He]2s$^1$. Es gibt nur ein Valenzelektron, also $L=0$ und $S=J=1/2$. Das Termsymbol des Grundzustandes ist daher $2^2S_{1/2}$.
    
    \item[Beryllium] Das hinzukommende Elektron muss den entgegengesetzten Spin haben, um noch in die 2s-Unterschale zu passen. Insgesamt muss dies ein Singulett-Zustand werden, um die Antisymmetrie zu erhalten. Also $L=0$ und $S=J=0$ und somit $2^1S_0$.
    
    \item[Bor] Mit Bor beginnt die 2p-Unterschale. Dieses neue Elektron hat $S=1/2$ und $L=1$. Da die Schale weniger als halb voll ist, wird das niedrigere $J$ bevorzugt, also $J=L-S = 1/2$. Insgesamt ist das Termsymbol $2^2P_{1/2}$.
    
    \item[Kohlenstoff] Kohlenstoff hat zwei Elektronen in 2p. Beide Elektronen haben die gleiche Spinausrichtung, um $S$ auf $S=1$ zu maximieren. Beide Elektronen haben $l=1$, aber die $m_l$ sind unterschiedlich (sonst würde das Pauli-Prinzip verletzt). Die $m_l$ sind so gewählt, dass $L$ maximal wird, d.h. ein Elektron mit $m_l=1$ und eines mit $m_l=0$. Damit ist der Gesamtbahndrehimpuls $L=1$. (Für $L=2$ müssten sich beide Elektronen im Zustand $m_l = 1$ befinden). $J$ wird minimal mit $J=L-S =0$. Das Termsymbol lautet daher $2^3P_0$.
    
    \item[Stickstoff] Jetzt sind 3 Elektronen in 2p. Alle haben die gleiche Spinausrichtung, um $S$ zu maximieren, also $S=3/2$. Die $m_l$ decken den ganzen Bereich $m_l = -1, 0, 1$ ab, so dass die Summe $L= 0$ ist. Damit ist auch $J = S \pm L = 3/2$ und das Termsymbol $2^4S_{3/2}$.
    
    \item[Sauerstoff] Das vierte Elektron muss einen anderen Spin haben, sonst passt es nicht mehr in die 2p-Unterschale. Damit heben sich zwei Spins auf und es bleibt nur $S=1$. Das vierte Elektron liefert aber einen Beitrag zum Bahndrehimpuls, der dann $L=1$ ist. Die Schale ist nun mehr als halb voll, und $J$ will mit $J=L+S=2$ maximal werden. Das Termsymbol ist $2^3P_2$.
    
    \item[Fluor] Ein weiteres Elektron in 2p. Nur ein Spin bleibt ungepaart, $S=1/2$. Das letzte Elektron hat $m_l = 0$, also bleibt $L=1$. Das maximale $J$ ist $J=L+S = 3/2$, zusammen also $2^2P_{3/2}$.
    
    \item[Neon] Das Edelgas Neon ist in dieser Hinsicht langweilig. Die 2p-Schale ist voll besetzt und $S = L = J = 0$ bzw. $2^1S_0$.   
\end{description}

\newpage


\section{Zusammenfassung}

\textit{Schreiben Sie hier ihre persönliche Zusammenfassung des Kapitels auf. Konzentrieren Sie sich auf die wichtigsten Aspekte.}

\vspace*{10cm}


%--------------------
\printbibliography[segment=\therefsegment,heading=subbibliography]

\renewcommand{\lastmod}{22. November 2024}
\renewcommand{\chapterauthors}{Markus Lippitz}

\chapter{Licht-Materie-Wechselwirkung}





\section{Überblick}





% 6. Lasers
% Sim: Lasers
% • We originally covered Lasers towards the end of the course, but we realized that we didn’t
% actually use anything other than the basics of spectra in our treatment, and the engineers got
% grumpy if we spent too long on fundamentals without any applications, so we moved Lasers
% to so that there was more emphasis on applications early in the course. This worked much
% better.
% • When we ask students why laser beams are so powerful, it’s split 50/50 between more power
% in the beam and more concentrated light.
% • The homework on lasers starts with basic questions about absorption and spontaneous and
% stimulated emission, works through the steps of building a laser and troubleshooting a broken
% laser, and ends with essays on why a population inversion is necessary to build a laser and
% why this requires atoms with three energy levels instead of two. Most students are able to
% give coherent explanations in these es



\section{Dipol-Übergänge}

Bisher haben wir die Wellenfunktion eines Elektrons im Atom immer als $\Psi(\br)$ geschrieben, also nur eine Ortsabhängigkeit, aber keine Zeitabhängigkeit berücksichtigt. Es sind aber beschleunigte Ladungen, die elektromagnetische Wellen aussenden. Wir brauchen also auch die Zeitabhängigkeit der Wellenfunktion. Dies ist einfach für Wellenfunktionen, die zu einem Energieeigenwert $E$ gehören. In diesem Fall ist die Zeitkomponente einfach 
\begin{equation}
    \Psi(\br, t) = \Psi(\br) \, e^{- i \frac{E}{\hbar} \, t}
\end{equation}
Das ist wie bei einer ebenen Welle, die die Form 
\begin{equation}
    u(\br, t) = u_0 \, e^{i ( \bk \cdot \br - \omega t)}
\end{equation}
hat. Der räumliche Teil ist in $\Psi(\br)$ ausgelagert. Der zeitliche Teil ist identisch, da $E = \hbar \omega$.

Soll nun ein Atom von einem Zustand $\Psi_A$ in einen Zustand $\Psi_E$ übergehen, so ist es plausibel, dass es sich dabei zumindest für eine sehr kurze Zeit in einer Überlagerung der Form 
\begin{equation}
    \Psi_\text{Übergang} (\br, t) = \Psi_A(\br, t) + \Psi_E(\br, t) = 
    \Psi_A(\br) \, e^{-i E_A t / \hbar} +  \Psi_E(\br) \, e^{-i E_E t / \hbar}
\end{equation}
befindet. Da es hier nur um das Prinzip geht, werden alle Vorfaktoren und deren zeitliche Entwicklung\sidenote{Die Gewichtung zwischen Anfangs- und Endzustand sollte sich bei einem Übergang ja ändern.} weggelassen. Die Wahrscheinlichkeitsdichte ist dann
\begin{align}
    & \left| \Psi_\text{Übergang} (\br, t)  \right|^2 =  \Psi_\text{Übergang}^\star (\br, t) \Psi_\text{Übergang} (\br, t)  \\
   & =  \left(  \Psi_A^\star(\br) \, e^{+i E_A t / \hbar} +  \Psi_E^\star(\br) \, e^{+i E_E t / \hbar} \right)
    \left(  \Psi_A(\br) \, e^{-i E_A t / \hbar} +  \Psi_E(\br) \, e^{-i E_E t / \hbar} \right) \\
    & = | \Psi_A(\br)|^2 + | \Psi_E(\br)|^2  + 2 \Re \left\{ \Psi_A(\br)\Psi_E^\star(\br)  \, e^{-i (E_A - E_E) t / \hbar}  \right\} 
\end{align}
Wasserstoff-Wellenfunktionen sind reellwertig, so dass in der letzten Zeile auch der Konjugiert-Komplex weggelassen und die Exponentialfunktion durch einen Cosinus ersetzt werden kann.

Was passiert hier? Befindet sich ein Atom in einem Überlagerungszustand, so schwingt die Aufenthaltswahrscheinlichkeit des Elektrons und damit die Ladungsdichte mit der Kreisfrequenz $\omega_{AE} = (E_A - E_E) / \hbar $. Diese oszillierende Ladung sendet dann entweder elektromagnetische Wellen mit der Frequenz $\omega_{AE}$ aus oder wird, wie beim getriebenen Oszillator, von einer einfallenden elektromagnetischen Welle mit dieser Frequenz getrieben. Im ersten Fall wird Licht emittiert, im zweiten Fall absorbiert.

In der klassischen Elektronendynamik besteht ein Dipol aus einer positiven Ladung $q=+e$ am Ursprung und einer negativen Ladung $q=-e$ am Ort $\br$. Diese Ladungsverteilung hat das Dipolmoment $\bp = -e \br$. Für die Verteilung der Elektronen um einen positiven Kern im Ursprung integriert man über den Raum, d.h. 
\begin{equation}
    \bp = - \int_\text{Raum} e\br \,  | \Psi(\br, t) |^2 \, d \br 
\end{equation}
Die Wasserstoffwellenfunktionen sind alle punktsymmetrisch um den Ursprung. Daher tragen die Terme $ | \Psi_{A,E}(\br)|^2$ von $| \Psi_\text{Übergang}(\br,t)|^2$ nichts bei. Es bleibt 
\begin{equation}
    \bp =  - \Re \left \{ e^{-i (E_A - E_E) t / \hbar}  \, \int_\text{Raum} e\br \,  \Psi_A(\br)\Psi_E^\star(\br)  \, d \br \right \}
\end{equation}
Das räumliche Integral bestimmt vollständig, wie gut der Übergang $A \rightarrow E$ mit Licht möglich ist. Man nennt diesen Term \emph{Übergangs-Dipolmoment} oder Dipol-Matrixelement $\bM_{EA}$.
\begin{equation}
    \bM_{EA} = \int_\text{Raum} e\br \,  \Psi_A(\br)\Psi_E^\star(\br)  \, d \br 
    \label{eq:7_Matrix_element}
\end{equation}
Es handelt sich um einen Vektor, da das Integral als gewichtete Summe der Vektoren $\br$ aufgefasst werden kann.


\section{Auswahlregeln für Dipolübergänge}

In den meisten Fällen ist der genaue Wert des Übergangs-Dipolmoments $\bM_{EA}$ nicht von Bedeutung. Von Interesse ist vielmehr, ob 
für eine gegebene Kombination der Zustände $A$ und $E$ sein Betrag $|\bM_{EA}|$ von Null verschieden ist oder nicht. Ist er ungleich Null, so wird dieser Übergang als erlaubt bezeichnet, andernfalls als verboten. 

Wir suchen nun nach Regeln, die diese erlaubten Übergänge identifizieren. Dies sind die Auswahlregeln. Tatsächlich ist $|\bM_{EA}|$ für die meisten Kombinationen gleich Null. Das liegt an der Symmetrie der Wellenfunktionen. Wäre $\br$ nicht im Integral, dann wäre das Integral für alle $A \neq E$ Null, da die Wellenfunktionen orthonormiert sind.

Sehr viel erreicht man schon bei der Betrachtung der  \emph{Parität}. Die Parität einer Funktion $f$ beschreibt, wie sie sich unter Spiegelung aller Koordinaten verhält. Man bezeichnet sie als gerade oder ungerade, je nachdem ob das $n$ in 
\begin{equation}
    f(\br) = (-1)^n f(- \br)
\end{equation}
gerade oder ungerade ist. Bei gerader Parität ist also $f(\br) = + f(- \br) $.
Eine Funktion kann auch keine Parität haben, z. B. $f(x) = 1 + x$. Für Wasserstoffwellenfunktionen ist die Parität $(-1)^l$, d.h. gerade für gerade $l$. Das $\br$ in Gl. \ref{eq:7_Matrix_element} hat eine ungerade Parität und das Integral verschwindet, wenn die Parität insgesamt ungerade ist. Daher muss sich die Parität von $\Psi_A$ von der Parität von $\Psi_B$ unterscheiden, d.h. sie muss sich beim Übergang ändern. Dies ist die erste Auswahlregel, die immer gültig ist.

Der Spin in den Zuständen $A$ und $E$ ist ortsunabhängig. Der Spinanteil der Wellenfunktionen\sidenote{den wir bisher nie explizit geschrieben haben} kann also vor das Integral gezogen werden. Da auch die Spin-Wellenfunktionen orthonormal sind, kann sich der (Gesamt-)Spin bei einem Übergang nicht ändern. Allerdings gibt es die Spin-Bahn-Kopplung. Diese führt zu einem Einfluss des Spins auf den räumlichen Teil der Wellenfunktion, so dass nicht mehr alles vor das Integral gezogen werden kann. Die Spin-Erhaltung gilt also nicht strickt und immer weniger, je mehr die Spin-Bahn-Kopplung mit steigender Kernladung zunimmt.


Hier nun ein Überblick über die Auswahlregeln
\begin{description}
    \item[Es gilt immer] \ \\
\begin{itemize}\setlength{\itemsep}{0pt}
    \item Die Parität muss sich ändern.
    \item $\Delta J = 0, \pm 1$, aber der Übergang $J=0$ nach $J=0$ ist verboten.
    \item $\Delta m_J = 0, \pm 1$, aber der Übergang $m_J=0$ nach $m_J=0$ ist verboten, falls $\Delta J = 0$.
\end{itemize}

\item[Für Einelektron-Atome]  gilt immer  \ \\
\begin{itemize}\setlength{\itemsep}{0pt}
    \item  $\Delta l = \pm 1$ (aber  $\Delta l = 0$ ist verboten)
    \item $\Delta s = 0$, weil alle Elektronen $s=1/2$ besitzen.
\end{itemize}

\item[Für Mehrelektron-Atome] gilt im Bereich der LS-Kopplung   \ \\  
\begin{itemize}\setlength{\itemsep}{0pt}
    \item $\Delta S = 0$, gilt durch die LS-Kopplung aber nur schwach.
    \item $\Delta L = \pm 1$. In Spezialfällen (mehr als ein Elektron verändert seine Wellenfunktion) ist auch $\Delta L = 0$ erlaubt. Dann ist immer noch der Übergang $L=0$ nach $L=0$  verboten.
\end{itemize}

\item[Für schwerere Mehrelektron-Atome]  jenseits  der LS-Kopplung gibt es \emph{zusätzlich} zu den Übergängen der LS-Kopplung noch mit geringerer Wahrscheinlichkeit    
\begin{itemize}\setlength{\itemsep}{0pt}
    \item $\Delta S  = \pm 1$
    \item $\Delta L =  \pm 2$ 
\end{itemize}
\end{description}

Es gibt keine Auswahlregel, die eine Änderung der Hauptquantenzahl $n$ verlangt! Insbesondere bei schweren Atomen sind die Energien zwischen den Zuständen gleicher Hauptquantenzahl so verschieden, dass relevante optische Übergänge auftreten, wie wir unten am Beispiel von Natrium sehen werden.

\subsection{Drehimpuls und Polarisation}

Neben Parität und Spin ist es die Drehimpulserhaltung, die die Auswahlregeln bestimmt.
Die Drehimpulserhaltung wird bei optischen Übergängen nicht verletzt, da auch das Photon einen intrinsischen Drehimpuls, den Spin $\bS_\gamma$, besitzt. Wir haben bereits in Tabelle XXX gesehen, dass dieser Spin 1 ist, also ganzzahlig, und dass Photonen daher Bosonen sind. Es gibt auch eine Orientierungsquantenzahl $m_\gamma$. Die Orientierungsquantenzahl kann für Photonen nur die Werte $m_\gamma = \pm 1$ annehmen. Diese entsprechen links- bzw. rechts zirkular polarisiertem Licht, oft auch als $\sigma^+$ bzw. $\sigma^-$ Licht bezeichnet. Eine elektromagnetische Welle hat eine eingebaute Vorzugsrichtung, die Richtung des Wellenvektors $\bk$. Entlang dieser Richtung wird daher die quantisierte Komponente $m_\gamma$ des Drehimpulses angegeben. Den Fall $m_\gamma = 0$ gibt es nicht, da Licht eine Transversalwelle ist. Zirkulare Polarisationen sind die Eigenzustände von Photonen. Linear polarisiertes Licht ist eine Überlagerung der beiden zirkularen Polarisationen.

Es muss also insgesamt die Drehimpulserhaltung gelten:
\begin{align}
    \bJ_A = \bJ_E + \bS_\gamma  & \quad \text{Emission} \\
    \bJ_A + \bS_\gamma = \bJ_E & \quad \text{Absorption}
\end{align}
Bleiben wir der Einfachheit halber bei den Absorptionsvorgängen. Für die Quantenzahl $J_E$ des Drehimpulses $\bJ_E$ am Ende des Prozesses gilt aus der Drehimpulsaddition
\begin{equation}
    | J_A - S_\gamma | \le J_e \le J_A - S_\gamma
\end{equation}
mit $S_\gamma = 1$. Dies beschreibt also $\Delta J = 0, \pm 1$. Die Orientierungsquantenzahl ist einfach die Summe
\begin{equation}
    m_{J,A} + m_\gamma = m_{J,E} \quad .
\end{equation} 
$\Delta m_J = \pm 1$ entspricht also $ m_\gamma = \pm 1$ bei Absorption (bei Emission ändert sich das Vorzeichen). Diese Übergänge sind also selektiv für die jeweiligen zirkularen Polarisationen. Der Übergang $\Delta m_J = 0$ erfordert linear polarisiertes Licht, also eine Überlagerung der beiden zirkularen Polarisationen.


Für das Ein-Elektronen-Atom lässt die Drehimpulserhaltung eigentlich auch $\Delta l = 0$ zu. Dieser Fall ist jedoch verboten, da sich die Parität bei $\Delta l = 0$ nicht ändert. Übergänge in Mehrelektronenatome sind sehr häufig von der Art, dass nur ein Elektron seine Quantenzahl ändert. In diesem Fall sind die Auswahlregeln im Grunde die gleichen wie für das Ein-Elektronen-Atom. Nur in seltenen Fällen führt die Absorption eines Photons zu einer Änderung bei zwei Elektronen. Diese Übergänge können dann auch $\Delta L = 0$ zeigen.



\subsection{Höhere Ordnungen}


Neben den oben besprochenen elektrischen Dipolübergängen gibt es auch magnetische Dipolübergänge, die also einem oszillierenden magnetischen Dipol entsprechen, und alle anderen Multipole, also Quadrupole, Oktupole usw. Diese Übergänge sind in der Regel schwächer und haben andere Auswahlregeln. Unter bestimmten Bedingungen kann ein Photon auch eine Art Bahndrehimpuls besitzen. Die elektromagnetische Welle ist dann keine ebene Welle mehr. Solche Photonen führen dann zu  anderen Auswahlregeln.



\section{Beispiel: Helium}

XXX FIG He

%Harris fig 8.30

Abbildung XXX zeigt die niedrigsten Zustände von Helium. Diese wurden bereits am Ende des letzten Kapitels besprochen. Die Pfeile zeigen alle zulässigen Übergänge als Emission an. Zunächst fällt auf, dass die Übergänge innerhalb ihrer Multiplizität bleiben. Es gibt keinen erlaubten Übergang vom Singulett zum Triplett und umgekehrt, was eine Folge der Regel $\Delta S = 0$ ist. Dann ändern alle Pfeile die 'Spalte' im Diagramm, d.h. $\Delta L = \pm 1$. Bei Helium wird in diesen Zuständen nur ein Elektron angeregt. Der Sonderfall $\Delta L = 0$ tritt daher nicht auf. Es gibt zwei metastabile Zustände, $2^1S_0$ und $2^3S_1$, die keinen erlaubten Zerfallskanal besitzen. Die Spin-Bahn-Kopplung und Wechselwirkungen jenseits der hier beschriebenen Dipolstrahlung führen aber auch diese Zustände wieder in den Grundzustand zurück.



\section{Beispiel: Natrium}


XXX FIG Natrium

% Demtröder fig 6.22

Natrium ähnelt Wasserstoff in dem Sinne, dass ein einziges Valenzelektron seine Eigenschaften bestimmt. Die Elektronenkonfiguration ist [Ne]3s$^1$. Abbildung XXX zeigt die niedrigsten angeregten Zustände und die Übergänge zwischen ihnen. Der Grundzustand hat das Termsymbol 3$^2$S$_{1/2}$, also ein Doublet. Charakteristisch sind die \emph{Natrium-D-Linien} bei 589,5 nm bzw. 588,9 nm Wellenlänge, die z.B. im gelben Licht einiger Straßenlaternen zu sehen sind.
Dies sind Übergänge aus den Zuständen 3$^2$P$_{1/2}$ und 3$^2$P$_{3/2}$. Dabei ist $\Delta S = 0$, $\Delta L = 1$ und $\Delta J = 0$ bzw. $1$. Die Hauptquantenzahl $n$ ändert sich also nicht. Die Feinstrukturaufspaltung $J = 1/2$ und $J=3/2$ führt zum Doublet der Linien.

\section{Lebensdauer angeregter Zustände}

Nachdem ein Atom durch die Absorption eines Photons oder durch einen Stoß in einen angeregten Zustand versetzt wurde, fällt es früher oder später wieder in den Grundzustand zurück. Dabei gibt es zwei Zeitspannen: Die Lebensdauer des angeregten Zustandes, also die Zeit, die das Atom in diesem Zustand bleibt, bevor es wieder zurückfällt, und die Dauer des Quantensprungs selbst. Der Quantensprung ist instantan. Das Atom verbringt also keine Zeit im Übergang selbst. Die Zeit bis zum Übergang lässt sich aber leicht messen: Man regt ein Gas von Atomen mit einem kurzen Lichtblitz an und misst die Helligkeit des abgestrahlten Lichts als Funktion der Zeit. Man findet einen exponentiellen Zerfall mit einer Zeitkonstante von etwa 10~ns.

Was bedeutet das? Die Wahrscheinlichkeit, dass ein gegebenes angeregtes Atom ein Photon emittiert, ist unabhängig vom zeitlichen Abstand $t$ zwischen Anregungspuls und Detektionszeitfenster. Wie beim Würfeln ist die Wahrscheinlichkeit, eine Sechs zu würfeln, unabhängig von der Anzahl der bisherigen Würfe. Je öfter man würfelt, desto wahrscheinlicher ist es, dass es einmal passiert. Für jedes einzelne geworfene Atom kann man also nicht sagen, wann es das Photon aussenden wird. Für alle Atome zusammen kann man jedoch die Anzahl $N$ der Atome in angeregten Zuständen angeben, und diese Anzahl nimmt mit einem Exponentialgesetz ab: 
\begin{equation}
    \frac{d N}{dt} = - \frac{N}{\tau} \quad \text{und also} \quad N(t) = N_0 \, e^{- t / \tau}
\end{equation} 
Dabei nennt man $\tau$ die Lebensdauer des angeregten Zustands und $k = 1 / \tau$ die Zerfallsrate.


\section{Einstein-Koeffizienten}


\begin{marginfigure}
    \inputtikz{\currfiledir einstein-coeff}
    \caption{Die drei Einstein-Koeffizienten.}
\end{marginfigure}

Im Jahre 1917, also noch vor dem Bohrschen Atommodell und der de Broglie-Wellenlänge, erkannte A. Einstein, dass es neben den oben besprochenen Absorptions- und Emissionsprozessen noch einen weiteren Prozess geben muss, damit ein Atom im thermischen Gleichgewicht mit dem umgebenden (Schwarzkörper-) Licht sein kann. Betrachten wir dazu noch einmal Absorption und Emission.

Wir vereinfachen das Atom auf zwei Niveaus mit den Besetzungen $N_1$ und $N_2$. Die oben beschriebene Emission nennen wir \emph{spontane Emission}, da sie ohne äußere Einwirkung erfolgt. Die Übergangsrate\sidenote{hier $ k_\text{spontan}$ mit $ keinen_\text{spontan} = k = 1/ \tau$ von oben } ist proportional zur Besetzung des angeregten Zustands
\begin{equation}
    k_\text{spontan} = A_{21} \, N_2 \label{eq:7_k_spontan}
\end{equation}
$A_{21}$ ist der \emph{Einsteinkoeffizient der spontanen Emission}.


Damit eine Absorption stattfinden kann, das Atom also vom Zustand 1 in den Zustand 2 übergehen kann, muss ein Lichtfeld vorhanden sein. Dieses Feld habe die spektrale Energiedichte $u(E)$. Die Übergangsrate ist dann
\begin{equation}
    k_\text{absorption} = B_{12} \, N_1 \, u(E_2 - E_1) \label{eq:7_k_abs}
\end{equation}
$B_{12}$ ist der \emph{Einstein-Koeffizient der Absorption}.

Im thermodynamischen Gleichgewicht muss die Besetzung $N_i$ der Zustände über die Zeit konstant sein und das Verhältnis $N_2 / N_1$ muss der Boltzmann-Statistik entsprechen. Dies kann nur erreicht werden, wenn es auch den  Prozess der stimulierten Emission gibt. Dabei stimuliert  (oder induziert) ein einfallendes Photon den Zerfall eines angeregten Atoms, inklusive der Emission eines weiteren Photons. Vorher gibt es ein Photon udn ein Atom im angeregten Zustand, nachher  zwei Photonen und ein Atom im Grundzustand. Das zweite Photon ist dabei eine exakte Kopie des ersten. Die Übergangsrate ist 
\begin{equation}
    k_\text{stimuliert} = B_{21} \, N_2 \, u(E_2 - E_1) \label{eq:7_k_stim}
\end{equation}
mit $B_{21}$ dem  \emph{Einstein-Koeffizienten der stimulierten Emission}.


Im thermischen Gleichgewicht muss die Rate aus Zustand 1 heraus  der Summe der Raten aus Zustand 2 heraus entsprechen, 
oder 
\begin{equation}
    B_{12} \, N_1 \, u(E_2 - E_1) =   A_{21} \, N_2 + B_{21} \, N_2 \, u(E_2 - E_1) 
\end{equation}
Und das Verhältnis der Besetzung muss der Boltzmann-Verteilung gehorchen
\begin{equation}
    \frac{N_2}{N_1} = \frac{g_2}{g_1} \, e^{- h \nu / (k_B T)}
\end{equation}
mit $h \nu = E_2 - E_1$ und $g_i = 2J +1$ der Entartung des Zustands $i$. Mit der spektralen Energiedichte des Schwarzkörpers 
\begin{equation}
    u(\nu) = \frac{8 \pi h \nu^3}{c^3} \,  \frac{1}{e^{h\nu/k_B T} -1}
\end{equation}
findet man folgende Zusammenhänge zwischen den drei Einstein-Koeffizienten
\begin{equation}
   g_1 B_{12}  = g_2 B_{21} \quad \text{und} \quad
   A_{21} = \frac{8 \pi h \nu^3}{c^3}\, B_{21} 
\end{equation}
Es gibt also nur einen freien Parameter, der direkt mit dem Übergangsdipolmoment $|\bM_{21}|^2$ zusammenhängt.


\section{Der Laser}
%\phet{Lasers}

Stimulierte Emission spielt unter normalen Bedingungen praktisch keine Rolle. Ein Atom befindet sich nur für sehr kurze Zeit im angeregten Zustand. Genau in diesem Moment müsste ein geeignetes stimulierendes Photon vorhanden sein. Das ist selten. Oder umgekehrt: In den allermeisten Situationen ist die stimulierte Rate viel kleiner als die spontane, also 
\begin{equation}
    u(h\nu)  B_{21}  \ll A_{21} = \frac{8 \pi h \nu^3}{c^3} \, B_{21} 
\end{equation}
weil 
\begin{equation}
    \frac{1}{e^{h\nu/k_B T} -1} \ll 1
\end{equation}

Niemand zweifelte an der von Einstein eingeführten stimulierten Emission, aber es dauerte bis in die 1950er Jahre, bis sie sinnvoll genutzt wurde. 1960 demonstrierte Theodore Maiman den ersten 'Laser'. Die Abkürzung steht für 'light amplification by stimulated emission of radiation' (Lichtverstärkung durch stimulierte Emission von Strahlung). Das Konzept geht zurück auf den von Charles H. Townes\sidenote{dafür Nobelpreis 1963} 1954 demonstrierten 'Maser' (microwave amplification by...), der von Arthur L. Schawlow\sidenote{Nobelpreis 1981 für Laserspektroskopie} weiterentwickelt wurde.


Die stimulierte Emission sollte also der dominierende Prozess sein, wenn ein Photon mit einem Atom wechselwirkt. Wenn wir aber der Einfachheit halber Entartung gleichsetzen ($g_1 = g_2$), dann muss wegen der Ähnlichkeit von Gl \ref{eq:7_k_abs} und \ref{eq:7_k_stim} die Besetzung im oberen Zustand größer sein als die im unteren Zustand, im Idealfall viel größer: $N_2 \gg N_1$. Andernfalls würde ein einfallendes Photon einfach absorbiert und nicht durch stimulierte Emission kopiert. Der Fall $N_2 > N_1$ wird als \emph{Besetzungsinversion} bezeichnet. Er ist nicht im thermischen Gleichgewicht. Für keine Temperatur $T$ liefert die Boltzmannverteilung eine solche Besetzung.

Wie erreichen wir die Besetzungsinversion? Wir müssen den Atomen Energie zuführen. Wir nehmen hier einmal an, dass dies durch Licht geeigneter Wellenlänge geschieht. Das nennt man \emph{optisches Pumpen}. So hat es Maiman in seinem Rubin-Laser gemacht. Heute kann man die Energie auch anders zuführen, zum Beispiel elektrisch. Es hilft nicht, wenn das Pumplicht den Übergang $1 \rightarrow 2$ pumpt. Zunächst nimmt zwar die Besetzung des Zustands 1 mit steigender Pumpintensität zu. Wenn wir aber die Gleichbesetzung erreichen ($N_1 = N_2$), dann wird ein einfallendes Pumpphoton mit der gleichen Wahrscheinlichkeit absorbiert, wie es stimulierte Emission verursacht. Diese Grenze kann nicht überschritten werden, und $N_2 > N_1$ wird nie erreicht.

xxx sketch 3 level system

Der Ausweg ist ein System, das aus mehr als zwei Niveaus besteht, z.B. drei oder vier (siehe Abbildung XXX). Wir pumpen den Übergang $1 \rightarrow 3$. Der Zustand 3 ist kurzlebig und geht schnell in den Zustand 2 über, möglicherweise unter Aussendung eines Photons, das uns hier nicht interessiert. Damit ist die Besetzung $N_3 \ll N_1$ zu jedem Zeitpunkt, so dass dieser Übergang gut gepumpt werden kann. Gleichzeitig können wir  Besetzung im Zustand 2 akkumulieren und so $N_2 > N_1$ erreichen.


Neben der Energiequelle zum Pumpen und dem atomaren Gas als Verstärkungsmedium benötigt man einen Resonator, d.h. zwei Spiegel, die das Verstärkungsmedium umschließen. Die stimulierte Emission findet nicht bei jedem Durchgang eines Photons durch das Medium statt.  Der Resonator sorgt dafür, dass jedes Photon viele Chancen hat, einmal kopiert zu werden. Ist dies einmal geschehen, entsteht eine Art Lawine identischer Photonen, bis die stimulierte Emission ausreicht, um die Besetzungsinversion abzubauen und den Zustand $N_2 = N_1$ zu erreichen. Der Laser ist also eine Quelle identischer Photonen. Dies entspricht klassischerweise kohärentem Licht.


\section{Beispiel: Rubin-Laser}

Der historisch erste Laser verwendet Rubin als aktives Medium. Rubin ist ein Korund-Mineral, Chromaluminiumoxid, d.h. Chromionen in einer Aluminiumoxidmatrix (\ch{Al2O3}). Aluminiumoxid ist durchsichtig wie Glas. Ein Stab aus diesem Material ist an den Enden poliert und verspiegelt. Chrom ist in sehr geringer Menge (0,05\%) eingelagert. Die \ch{Cr^{3+}}-Ionen werden durch eine Blitzlampe in hoch angeregte Zustände gepumpt. Von dort zerfallen sie in einen tieferliegenden angeregten Zustand. Die überschüssige Energie wird in Form von Wärme an den Stab abgegeben. Der Rubinlaser ist somit ein Festkörperlaser und arbeitet bei einer Wellenlänge von 694.3~nm, an der Grenze zum nahen Infrarot. 

xxx sketchj states ruby

\section{Helium-Neon-Laser}

Der \ch{He:Ne}-Laser verwendet Neon als aktives Medium und Helium zur Anregung in einer Gasentladungsröhre. Die beschleunigten Elektronen regen das Helium durch Stöße vom Zustand 1s$^2$ in den Zustand 1s~2s an. Dieser Zustand kann wegen der Auswahlregel $\Delta l = \pm1$ nicht (schnell) in den Grundzustand zerfallen und lebt daher relativ lange. 

Der \ch{He}-2s-Zustand ist resonant mit dem angeregten 5s-Zustand von Neon (Neon-Grundzustand 1s$^2$ 2s$^22$ 2p$^6$). 'Resonant' bedeutet, dass beide Zustände die gleiche Energie von 20,6~eV haben.  Bei der Kollision eines Neon-Atoms mit einem Helium-Atom kann daher die Anregung sehr effizient übertragen werden. Dieser 5s-Zustand dient dann als oberer Zustand im Laser. Der Übergang zum 3p-Zustand bei 632,8~nm erfolgt durch stimulierte Emission. Der untere 3p-Zustand zerfällt sehr schnell weiter in den Grundzustand. Obwohl die Besetzung in 5s niedrig ist, ist sie höher als in 3p, so dass auch hier eine Besetzungsinversion vorliegt. Es handelt sich also auch hier um einen Drei-Niveau-Laser.

xxx sketchj states He Ne


\section{Linienbreiten}

Bisher haben wir nur die Lage der Linien im Atomspektrum diskutiert, wobei wir immer davon ausgegangen sind, dass sie deltaförmig sind. Das ist nicht der Fall. Wenn man hochauflösende Spektroskopie betreibt, beobachtet man eine Linienbreite, die größer ist als die experimentelle Auflösung. Drei Ursachen werden hier diskutiert.

\subsection{Natürliche Linienbreite} 

Selbst bei einem idealen Atom muss aufgrund der Energie-Zeit-Unschärfe (oder der Fourier-Transformation) ein Zusammenhang zwischen der Lebensdauer des angeregten Zustands und der spektralen Breite des Übergangs bestehen. Je länger ich eine Frequenz messen kann, desto genauer kann ich sie bestimmen. Die resultierende Linienbreite wird als 'natürlich' bezeichnet, da sie die fundamentale Grenze darstellt. Für sie gilt
\begin{equation}
    \delta \nu_\text{nat} = \frac{1}{2\pi \, \tau}
\end{equation}
mit der Lebensdauer $\tau$ des angeregten Zustands. Für die Natrium-D-Linie aus dem Zustand 3D$_{1/2}$ ergibt sich bei einer Lebensdauer von $\tau = 16$~ns eine Linienbreite von $\delta \nu_\text{nat} = 10$~MHz.

\paragraph{Nebenbemerkung} Wenn Sie es merkwürdig finden, dass hier die Lebensdauer des Zustands eingeht, während das Atom nur darauf wartet, dass es endlich emittiert, dann liegt das an all den Dingen, die wir weggelassen haben. Lesen Sie dann über  'Rabi-Oszillationen' und stellen fest, dass Quantensprünge manchmal doch nicht instantan sind.


\subsection{Doppler-Verbreiterung}


Bewegt sich eine Atom mit der Geschwindigkeit  $v$ in Strahlrichtung, so ändert sich seine Absorptions-Frequenz aufgrund der Doppler-Verschiebung um $\Delta \nu$.
\begin{equation}
	\Delta \nu = \frac{v}{c_0} \, \nu_0
\end{equation}
mit der Ruhe-Frequenz $\nu_0$. Die Geschwindigkeit kommt  aus der  Maxwell-Boltzmann-Geschwindigkeitsverteilung $\mathcal{P}(v)$
\begin{equation}
	\mathcal{P}(v) = \sqrt{\frac{m}{2 \pi k_B T}} \, \exp \left (
	- \frac{m v^2}{2  k_B T}	
	\right) \quad .
\end{equation}
mit der Masse $m$ des Atoms.
Insgesamt ergibt sich damit eine Gaußförmige  Linie der Halbwertsbreite 
\begin{equation}
	 \delta \nu_{doppler} = \nu_0 \, \sqrt{\frac{8 k_B T \ln 2} {m c^2}} \quad .
\end{equation}
Für die Natrum-D-Linie ergibt sich  $\delta \nu_{doppler} = 1 700$~MHz. Sie übertrifft damit die natürliche Linienbreite bei weitem.

\subsection{Stoß-Verbreiterung}

Stößt ein Atom mit einem anderen zusammen, während es sich im angeregten Zustand befindet, so wird die Lebensdauer des angeregten Zustandes effektiv verkürzt, $\tau$ wird kürzer und damit wie oben $\delta \nu$ breiter. Die mittlere Zeit $\delta t$ zwischen den Stößen kann aus der mittleren freien Weglänge $l$ eines Gases der mittleren Geschwindigkeit $v$ abgeschätzt werden (\cite{Demtroeder_laser}) % eq. 3.52, demtröder laser spectrsocpy I
\begin{equation}
    \Delta t = \frac{l}{v} = \frac{1}{n \sigma} \, \sqrt{ \frac{\pi \mu}{8 k_B T}}
\end{equation}
Dabei ist $n$ die Teilchendichte (proportional zum Druck $p$) des Stoßpartners, $\sigma$ ein Streuquerschnitt, der die effektive Größe des Atoms beschreibt, und $\mu$ die reduzierte  Masse aus Atom und (ggf. anderem) Stoßpartner. Für die Liniebreite gilt dann wieder
\begin{equation}
    \delta \nu_\text{stoss} = \frac{1}{2\pi \, \Delta t} = p \, \sigma \sqrt{ \frac{8}{\pi \mu k_B T}}
\end{equation}
Bei spektroskopischen Experimenten kann man hier den Druck und damit die Teilchendichte $n$ verringern und diesen Verbreitungsmechanismus ausschalten. 


% XXX Rubidium Experiment ???

\section{Röntgenstrahlung}

In den einleitenden Kapiteln wurde bereits kurz über Röntgenstrahlung gesprochen. Röntgenstrahlung ist wie das sichtbare Licht eine elektromagnetische Welle. Der Unterschied besteht darin, dass die Energie pro Photon wesentlich höher ist. In diesem Kapitel wurde bisher nur ein Valenzelektron angeregt. In Abbildung XXX des letzten Kapitels sieht man jedoch, dass die tiefer gelegenen gebundenen Elektronen deutlich stärker gebunden sind als das Valenzelektron und ihre Übergänge daher mit sehr energiereichen Photonen verbunden sind. Wir betrachten nun also die Spektroskopie der inneren Elektronen.

\subsection{Bremsstrahlung}

Bevor wir zu diesem Punkt kommen, müssen wir kurz auf die Bremsstrahlung eingehen. Darunter versteht man ein breites Kontinuum im Röntgenspektrum,  das nicht vom Kathodenmaterial abhängt. Die durch die Spannung $U$ beschleunigten Elektronen werden an den Kernen des Kathodenmaterials abgelenkt, ohne dass die Kerne oder die Elektronen des Kathodenmaterials Energie aufnehmen. Ablenkung bedeutet Beschleunigung, und die beschleunigten Elektronen sind die Quelle  elektromagnetischer Strahlung.  Das Kontiunium hat eine maximale Frequenz bzw. eine minimale Wellenlänge, die von der Beschleunigungsspannung $U$ abhängt: Maximal kann die gesamte Energie $e \, U$ auf ein Photon übertragen werden.
\begin{equation}
    h \nu_\text{grenz} = e \, U \quad \text{oder} \quad
    \lambda_\text{grenz} = \frac{h c}{e U}
\end{equation}
Beschleunigungsspannungen im Bereich von einigen 10~kV ergeben Wellenlängen von 0.1--1~nm (und natürlich Photonenenergien von einigen 10~keV).

Das Spektrum der Bremsstrahlung ist linear in $\nu$, also 
\begin{equation}
    I(\nu) d\nu \propto \nu_\text{grenz} - \nu \quad \text{für} \quad  \nu < \nu_\text{grenz}
\end{equation}
Nach Umrechnung auf eine Wellenlängenskala ergibt sich
\begin{equation}
    I(\lambda) d\lambda \propto \left( \frac{\lambda}{\lambda_\text{grenz}} - 1 \right) \frac{1}{\lambda^2} 
    \quad \text{für} \quad  \lambda > \lambda_\text{grenz}
\end{equation}

xxx sketchj spektrum


\subsection{Charakteristisches Linienspektrum}
Experimentell findet man neben dem Kontinuum der Bremsstrahlung eine für das Kathodenmaterial charakteristische Abfolge von scharfen Linien. Hier wird ein inneres Elektron des Kathodenmaterials aus dem Atom herausgeschlagen. Eines der verbleibenden, energetisch höheren Elektronen fällt unter Aussendung eines Röntgenphotons in diese Lücke zurück. Die Lage der Linien hängt also von zwei Zuständen ab: dem des herausgeschlagenen Elektrons und dem des auffüllenden Elektrons.

Ein einfaches Modell, das Moseley-Gesetz, ignoriert alle Details auf atomarer Ebene und geht einfach von einem wasserstoffähnlichen System aus. In diesem Fall wird die Energie eines Zustands nur durch seine Hauptquantenzahl $n$ beschrieben. Die Energie des emittierten Röntgenphotons ist dann 
\begin{equation}
    h \nu = (Z - S_n)^2 \, R_H \, hc \, \left(  \frac{1}{n^2} - \frac{1}{m^2} \right)
\end{equation}
Dabei ist $n$ die Quantenzahl des herausgeschlagenen Elektrons und $m$ die des auffüllenden. Der Parameter $S_n$ korrigiert die Kernladung $Z$ um die weiter innen liegenden Elektronen und ist daher von $n$ abhängig. Für $n=1$ ist $S \approx. 1$, für $n=2$ ist $S \approx. 7,8$.
Wie bei den Schalen der Mehrelektronenatome werden die Linien(gruppen) entsprechend der Quantenzahl $n = 1, 2, 3$ mit den Buchstaben K, L, M bezeichnet. Die Quantenzahl $m$ wird in ihrem Abstand von $n$ durch griechische Buchstaben kodiert: $\alpha, \beta, \gamma$ bedeutet also $m=n + 1,2,3$. Die Linie K$_\alpha$ entspricht einem Übergang von $m=2$ nach $n=1$, die Linie L$_\gamma$ einem Übergang von $m=5$ nach $n=2$.

\subsection{Absorption}

Absorptionsspektren können auch mit Röntgenstrahlen gemessen werden. Wie immer nimmt die transmittierte Leistung $P$ exponentiell mit der Dicke des Materials ab.
\begin{eqnarray}
    P(x)  = P_0 e^{- \alpha x} = P_0 \, e^{- n \sigma  x}
\end{eqnarray}
mit dem Absorptionskoeffizienten $\alpha$. Dieser ist das Produkt aus der Teilchenzahldichte $n$ und dem Absorptionskoeffizienten $\sigma$ ab. Der Absorptionskoeffizienten beschreibt dem Effekt jedes einzelnen Atoms auf die Röntehstarhlung.

Drei Effekte tragen zur Absorption von Röntgenphotonen bei
\begin{description}
    \item[Photoeffekt] Das Röntgenphoton wird vom Atom absorbiert. Dabei wird ein stark gebundenes Elektron ionisiert und das Atom erhält noch kinetische Energie.
    \item[Compton-Effekt] Das Röntgenphoton stößt mit einem fast freien Elektron einer der äußeren Schalen des Atoms. Dabei übernimmt das Elektron kinetische Energie und die Wellenlänge des Röntgenphotons ändert sich.
    \item[Paarerzeugung] Bei ausreichend hohen Energien des Röntgenphotons ($E > 1$MeV) kann ein Elektron-Positron-Paar erzeugt werden. Hier wird Energie in Materie umgewandelt! Die restliche Energie geht in die kinetische Energie der neuen Teilchen.
\end{description}

Solange die Energie noch nicht für Paarerzeugung ausreicht, findet man
\begin{equation}
    \sigma \approx C \, Z^4 \, \lambda^3
\end{equation}
Die Absorption nimmt also mit der Kernladungszahl und der Wellenlänge stark zu. Die Konstante $C$ hängt von der Quantenzahl $n$ bzw. der beteiligten Schale ab: Mit zunehmender Wellenlänge reicht die Energie des Röntgenphotons irgendwann nicht mehr aus, um Elektronen einer Schale zu ionisieren. Die K-Kante im Absorptionsspektrum fällt daher mit der Grenze der K-Serie im Emissionsspektrum zusammen.



\section{Zusammenfassung}

\textit{Schreiben Sie hier ihre persönliche Zusammenfassung des Kapitels auf. Konzentrieren Sie sich auf die wichtigsten Aspekte.}

\vspace*{10cm}


%--------------------
\printbibliography[segment=\therefsegment,heading=subbibliography]


\renewcommand{\lastmod}{10. September 2024}
\renewcommand{\chapterauthors}{Markus Lippitz}

\chapter{Atome in externen Feldern}



\goal{By the end of this chapter, you should be able to draw, calculate and align a ray's path through an optical system.}


Ich kann den Einfluss eines externen Magnetfelds auf ein Atom beschreiben und für den Grenzfall schwacher Magnetfelder berechnen.

\section{Overview}


Normaler Zeeman-Effekt

8.1 Landé-g-Faktor 1	*	1 

8.2* Anormaler Zeeman-Effekt 2	***

8.3 Paschen-Back-Effekt 3	*	3 

8.4 Stark-Effekt 4	*	4 

8.5 Kern-Spin-Resonanz (NMR) 5	Anw	5 

Hyperfeinstruktur 


Einstein de Haas: Spin gibt es wirklich

\url{https://phet.colorado.edu/en/simulations/stern-gerlach}




\section{Zusammenfassung}

\textit{Schreiben Sie hier ihre persönliche Zusammenfassung des Kapitels auf. Konzentrieren Sie sich auf die wichtigsten Aspekte.}

\vspace*{10cm}

%--------------------
\printbibliography[segment=\therefsegment,heading=subbibliography]



% \renewcommand{\lastmod}{13. Dezember 2024} 
\renewcommand{\chapterauthors}{Markus Lippitz}

\chapter{Chemische Bindung in Molekülen}






% \begin{itemize}
% \item Sie können die Valenzbindungstheorie benutzen, um die Form von Molekülen vorherzusagen und zu erklären. Ein Beispiel ist das hier abgebildete Pentacen-Molekül.

% \item Sie können die Grundzüge verschiedener Methoden erklären, mit denen Eigenschaften von Molekülen bestimmt werden können.

% \item Sie können die Begriffe Orbital, $\sigma$- oder $\pi$-Bindung und Hybridisierung erklären und korrekt verwenden.

% \end{itemize}



\section{Überblick}

In diesem Kapitel diskutieren wir, wie es dazu kommt, dass sich Atome zu Molekülen verbinden. Dieser Effekt kann auf verschiedene Arten erklärt werden. Ein einfaches Modell sind zwei benachbarte Potentialtöpfe. In diesem Modell sind bereits viele Aspekte enthalten, die wir auch in Molekülen wiederfinden. Anschließend werden die in der Chemie gebräuchlichen Begriffe der $\sigma$- und $\pi$-Bindung eingeführt.
 
Wir diskutieren die Quantenmechanik der Bindung am Beispiel des \ch{H2^+}-Moleküls, das aus zwei Protonen und einem Elektron besteht, also das einfachste aller Moleküle ist. Wir verwenden das Variationsprinzip der Quantenmechanik, um die Wellenfunktionen mit der niedrigsten Energie zu finden. Auf diese Weise können wir das Bindungspotential berechnen. Wir finden, dass das Austauschintegral entscheidend dafür ist, dass die Energie im gebundenen Zustand niedriger ist als im ungebundenen Zustand. Für diese Austauschwechselwirkung gibt es keine klassische Entsprechung. Die Bindung ist somit ein rein quantenmechanischer Effekt.

\section{Modell: zwei Potentialtöpfe}
\begin{figure}
  \inputtikz{\currfiledir doppeltopf}
  \caption{Zwei Potentialtöpfe und die Wellenfunktionen mit den niedrigsten vier Energien. Links: kleiner Abstand, ein Molekül bildet sich. rechts: großer Abstand, die Atome bleiben ungekoppelt.}
\end{figure}

Beginnen wir mit einem einfachen Modell, das aber schon sehr viele Eigenschaften eines Moleküls erklären kann. Wir modellieren ein einzelnes Atom durch einen Potentialtopf von endlicher Tiefe. Wir nehmen also nicht an, dass die Wände unendlich hoch sind. Dadurch kann das Elektron ein wenig in den klassisch verbotenen Bereich eindringen. Bringt man nun ein zweites Atom in Form eines zweiten Potentialtopfs in die Nähe, so findet man Wellenfunktionen, die sich über beide Töpfe erstrecken. Dies ist zunächst eine rein mathematische Lösung. Betrachten wir zunächst den Fall sehr großer Abstände. Immer zwei Wellenfunktionen haben (fast) die gleiche Energie. Diese beiden Wellenfunktionen können linear so kombiniert werden, dass die Auftrittswahrscheinlichkeit in einem der beiden Töpfe gleich Null ist. In diesem Fall sind die Töpfe voneinander unabhängig.

Wenn der Abstand zwischen den Töpfen kleiner wird, passiert zweierlei: Die Entartung der Energien wird aufgehoben. In jedem Paar von Wellenfunktionen gibt es eine, die eine niedrigere Energie als die ungekoppelte atomare Eigenenergie hat. Dies ist die räumlich symmetrische Wellenfunktion, die einen Bauch in der Mitte zwischen den beiden Töpfen hat und die in jedem Topf das gleiche Vorzeichen hat. Die zweite Wellenfunktion des Paares liegt energetisch höher als die atomare Eigenenergie.  Diese anti-symmetrische  Wellenfunktion hat verschiedenes Vorzeichen in beiden Töpfen und einen Knoten in der Mitte. Die neuen Wellenfunktionen sind nun von einer Form, die auch nach der Linienkombination in beiden Töpfen immer eine von Null verschiedene Aufenthaltswahrscheinlichkeit hat. Die beiden Atome sind gekoppelt. Das Elektron gehört zu beiden Atomen. Damit dies geschieht, muss der Abstand zwischen den Atomen im Bereich der Abfalllänge der Wellenfunktion im klassisch verbotenen Bereich liegen. Das Elektron muss zum anderen Atom tunneln können.

Diese Kopplung führt manchmal zu einer Verringerung der Energie. Dies hängt davon ab, wie viele Elektronen in den Topf eingebracht werden müssen. Wenn wir annehmen, dass unser Topf ein Wasserstoffatom mit einem Elektron pro Atom darstellt, dann können wir diese zwei Elektronen in die Ortswellenfunktion mit der niedrigsten Energie einbauen, indem wir den Spin unterschiedlich einstellen, um das Pauli-Prinzip zu erfüllen. Diese Wellenfunktion hat eine geringere Energie als die ungekoppelten Atome. Die Gesamtenergie wird also reduziert, es entsteht eine Bindung, das \ch{H2}-Molekül.

Wenn wir aber beispielsweise von Helium ausgehen, müssen wir insgesamt 4 Elektronen unterbringen. Dazu benötigen wir die beiden untersten Wellenfunktionen. Die eine ist energetisch abgesenkt, die andere angehoben. In Summe gewinnt man durch die Kopplung nichts. Es gibt also kein Heliummolekül  \ch{He2}.


\paragraph{Vorschau:} In einem System mit mehr als zwei Töpfen, z.B. $N$, findet man immer Gruppen von $N$ Wellenfunktionen, die gleichmäßig um die alten atomaren Wellenfunktionen herum aufgespalten sind. Im Festkörper mit $N \approx 10^{23}$ bilden sich so \emph{Bänder} in der Energie (und Lücken dazwischen). Dies ist das Modell der \emph{tight binding model}. 


\section{Orbital oder Wellenfunktion?}

Wir besprechen hier Systeme, die  aus vielen Elektronen bestehen. Die Quantenmechanik und Atomphysik konzentrierte sich jedoch auf das Wasserstoff-Atom mit nur einem Elektron. Wir müssen daher vorsichtig mit der Nomenklatur sein. Die (Gesamt-)Wellenfunktion eines Systems aus $n$ Elektronen ist im allgemeinen Fall $\Psi(\mathbf{r}_1, \mathbf{r}_2, \dots)$, wobei die $\mathbf{r}_i$ die Position des Elektrons $i$ bezeichnen. In dieser Allgemeinheit hängt alles miteinander zusammen und ist viel zu komplex. Wir machen daher immer die Annahme, dass sich die Gesamt-Wellenfunktion als Produkt von Orbitalen $\phi_i$ schreiben lässt
\begin{equation}
\Psi(\mathbf{r}_1, \mathbf{r}_2, \dots) = \phi_1(\mathbf{r}_1) \, \phi_2(\mathbf{r}_2)  \dots \quad .
\end{equation}
Die Orbitale hängen also nur von der Position 'ihres' Elektrons ab, nicht von all den anderen Elektronen. Im Fall des Wasserstoff-Atoms mit nur einem Elektron gehen die beiden Begriffe ineinander über.
Diese Aufteilung funktioniert immer, wenn die einzelnen Elektronen nicht miteinander wechselwirken, aber genau das ist der Fall. Diese Näherung versucht also, durch geschickte Wahl der $\phi_i$ diese Wechselwirkung vorweg zu nehmen. Es geht also darum, 'gute' $\phi_i$ zu finden.



\section{$\sigma$ und $\pi$-Bindung}



In der Valenzbindungstheorie entstehen Bindungen wie im obigen Topf-Modell  durch das 
Paaren von zwei Elektronen. Zwei Kerne teilen sich also zwei Elektronen, die nicht mehr einem einzelnen Kern zugeordnet sind. Der Spin der beteiligten Elektronen muss anti-symmetrisch gegen Vertauschung sein, da die Ortswellenfunktion ja identisch ist, und das Pauli-Prinzip eine insgesamt anti-symmetrische Wellenfunktion  verlangt. Molekülorbitale entstehen aus den ursprünglichen atomaren Wellenfunktionen. Die Form des neuen Orbitals übernimmt Eigenschaften der ursprünglichen Wellenfunktion. Dies ist in Abbildung \ref{fig:10_AO_zu_MO} skizziert. 

\begin{marginfigure}
\inputtikz{\currfiledir gerade_ungerade}
\caption{Molekülorbitale, die hier aus atomaren 2s oder 2p-Orbitalen aufgebaut sind. Die Farbe kodiert das Vorzeichen der Wellenfunktion. Die Symmetrie $g$ oder $u$ ergibt sich aus der Punktspiegelung an der Mitte des Moleküls, hier durch den kleinen Punkt markiert. \label{fig:10_AO_zu_MO}}
\end{marginfigure}




Waren die Elektronen ursprünglich im 1s Zustand, so ergibt sich ein Orbital, das in erster Näherung die Summe der beiden 1s Zustände ist. Schaut man entlang der Bindungsachse des Moleküls, so ist die Projektion identisch mit der Projektion eines  s-Orbitals (genauer: das Elektron in diese Orbital hat den Bahndrehimpuls $l=0$). Diese Art der Bindung wird daher als \emph{$\sigma$-Bindung} bezeichnet. Je nach Vorzeichen der Linearkombination kann das Molekülorbital gerade  (g) oder ungerade (u) sein. Das anti-symmetrische ungerade  Orbital ist (wie im Topf-Modell) energetisch höher als das ungekoppelte Orbital. Diese Orbitale werden \emph{anti-bindend} genannt und mit einem Sternchen gekennzeichnet.

%
\begin{marginfigure}
\includegraphics[width=\textwidth]{\currfiledir p-zu-sigma.png}
\caption{Atomare p-Orbitale können zu $\sigma$- und $\pi$-Bindungen kombinieren. }
\end{marginfigure}


Atomare p-Zustände können auf zwei Arten zu Molekülorbitalen kombiniert werden: Nennen wir die Bindungsachse des Moleküls $z$, dann bilden p$_x$ oder p$_y$ Orbitale eine $\pi$-Bindung: In der Projektion sieht das Molekülorbital wie ein p-Zustand aus (genauer: das Elektron hat $l=1$). Zwei p$_z$-Orbitale dagegen sehen in der Projektion wie eine s-Wellenfunktion aus. Es handelt sich also ebenfalls um eine $\sigma$-Bindung. Man unterscheidet wieder gerade und ungerade sowie bindend und antibindend.



Als Beispiel betrachten wir das Molekül \ch{N2}. Die Elektronenkonfiguration von Stickstoff ist [He]2s$^2$2p$_x^1$2p$_y^1$2p$_z^1$. Wir nehmen die z-Achse als Verbindungsachse zwischen den Kernen.
Durch die Paarung der beiden p$_z$-Elektronen entsteht eine $\sigma$-Bindung. Aus den p$_x$- und p$_y$-Elektronenpaaren entstehen zwei weitere $\pi$-Bindungen.
 
 




% \begin{questions} 
% \item Decken sie die MO-Spalte in Abbildung  \ref{fig:10_AO_zu_MO} ab und vergewissern Sie sich, dass Sie die Art der Bindung und die Bezeichnung der Orbitale angeben können.

% \item Schauen Sie ggf. noch einmal in der Atomphysik nach, was eigentlich [He]2s$^2$2p$_x^1$2p$_y^1$2p$_z^1$ bedeutet und wo man diese Information findet.
% \end{questions}

%\textit{Lesen Sie Kapitel 9.7.1 Das \ch{H2O}-Molekül in \cite{Demtröder_ep3}. Wie kann man den Bindungswinkel von \ch{H2O} verstehen? In erster Näherung ergibt sich 90 Grad, in zweiter Näherung ein Wert, der näher am experimentell gefundenen liegt. \newline Schrieben und skizzieren Sie hier Ihre Erkenntnisse. }

% \begin{questions} 
% \item Wie kommt es zur Form der Moleküle?
% \end{questions}


\section{Bindungstypen: kovalent, polar, ionisch}

Die Bindung im Molekül kann durch die Aufenthaltswahrscheinlichkeit der beteiligten Elektronen beschrieben werden. Hat das Molekül die Form \ch{X2}, besteht also aus zwei identischen Atomen, so ist die Aufenthaltswahrscheinlichkeit erwartungsgemäß symmetrisch. Wir werden diesen Fall der kovalenten Bindung weiter unten am Beispiel von \ch{H2^+} genauer betrachten. Besteht das Molekül aus zwei verschiedenen Atomen, z.B. \ch{HCl}, so sind die Elektronen asymmetrisch zwischen den Atomen verteilt, wobei der Schwerpunkt in unserem Fall deutlich näher beim Chlor liegt. Die Bindung ist also polar. Dies kann so weit gehen, dass ein Elektron vollständig auf das andere Atom übergeht. Ein Beispiel ist \ch{CsF}. Cäsium gibt das Elektron ab, Fluor nimmt es auf. Es handelt sich um eine ionische Bindung zwischen den Ionen \ch{Cs^+} und \ch{F^-}. Im Allgemeinen ist eine Bindung in Molekülen weder rein kovalent noch rein ionisch.


\section{Hybridisierung von Kohlenstoff-Orbitalen}
%
\begin{marginfigure}
%\includegraphics[width=0.9\textwidth]{\currfiledir hybrid.png}
\inputtikz{\currfiledir levels_sp}
\caption{Elektronische Niveaus bei der Hybridisierung von Kohlenstoff. }
\end{marginfigure}
%
Insbesondere in der organischen Chemie der Kohlenwasserstoffe spielt die Hybridisierung der Kohlenstoff-Orbitale eine wichtige Rolle. Im Kohlenstoff-Atom besteht nur ein geringer Energieunterschied zwischen der energetisch niedrigsten Elektronenkonfiguration
[He]2s$^2$2p$_x^1$2p$_y^1$2p$_z^0$ und der nächst höheren [He]2s$^1$2p$_x^1$2p$_y^1$2p$_z^1$. Dies bedeutet, dass der Energieunterschied zwischen dem 2s und dem 2p-Orbital in Kohlenstoff sehr gering ist, und insbesondere 
ist der Energiegewinn durch die Bindung sehr oft größer als dieser Unterschied. Es ist daher oft energetisch günstiger, die Bindung ausgehend von einer Linearkombination von 2s und 2p-Orbitalen zu betrachten. Dies nennt man \emph{Hybridisierung} der Orbitale. Wenn ein s-Orbital und drei p-Orbitale beteiligt sind, dann wird dies als sp$^3$-Hybridisierung bezeichnet. Ohne Hybridisierung könnte Kohlenstoff nur zwei Bindungen eingehen (mit den p$_x$ und p$_y$-Orbitalen), nach sp$^3$-Hybridisierung vier, so dass die Gesamtenergie stärker abgesenkt werden kann.\sidenote{Auch ist die Idee eines s- oder p-Orbitals ein Ein-Elektron-Konzept, das in Mehrelektronen-Atomen durch die anderen Elektronen gestört wird.}

Die neuen Hybrid-Orbitale $h_{1 .. 4}$ sind so gewählt, dass $\braket{h_i | h_j} = \delta_{ij}$, also
\begin{align}
 h_1 = & s + p_x + p_y + p_z \\
 h_2 = & s + p_x - p_y - p_z \\
 h_3 = & s - p_x + p_y - p_z \\
 h_4 = & s - p_x - p_y + p_z  \quad .
\end{align}
Diese Orbitale entstehen also durch Interferenz der ursprünglichen Orbitale und haben eine Ladungsverteilung, deren Keulen einen Tetraeder aufspannen. Der Bindungswinkel ist $\arccos (-\frac{1}{3}) = 109.5^\circ$. Methan (\ch{CH4}) ist daher tetraederförmig.

\begin{marginfigure}
\includegraphics[width=\textwidth]{\currfiledir ch4.png}
\caption{sp$^3$-Hybridisierung in \ch{CH4}. }
\end{marginfigure}


Analog gibt es auch die sp$^2$ und die sp-Hybridisierung.  Die sp$^2$-Hybridisierung findet man beispielsweise in Ethen (\ch{C2H4}). Die drei  sp$^2$-Orbitale jedes Kohlenstoff-Atoms sind an der $\sigma$-Bindung der beiden Wasserstoff-Atome beteiligt und $\sigma$-Bindung zwischen den beiden Kohlenstoff-Atomen. Die zweite C--C Bindung ist eine 'gewöhnliche' $\pi$-Bindung zwischen den verbleibenden, nicht hybridisierten p-Orbitalen, die senkrecht auf die durch die sp$^2$-Orbitale gebildete Ebene stehen. Dadurch ergeben sich die Winkel in der HCH bzw. HCC-Bindung zu circa 120$^\circ$. Ein Beispiel für die sp-Hybridisierung ist Ethin (\ch{C2H2}, \ch{HC+CH} ).

%
\begin{marginfigure}
\includegraphics[width=\textwidth]{\currfiledir c2h4.png}
\caption{sp$^2$-Hybridisierung in C$_2$H$_4$. }
\end{marginfigure}
%


% \begin{questions} 
% \item Wie entscheidet sich, ob die 'normalen' oder die 'hybriden' Orbitale zum Einsatz kommen?
% \item Was ist bei der  Hybridisierung  so besonders an Kohlenstoff?
% \end{questions}


\section{Die kovalente Bindung im Wasserstoff-Molekül-Ion H$_2^+$}


Nun betrachten wir dem allereinfachsten Fall, das Wasserstoff-Molekül-Ion H$_2^+$, etwas genauer im Formalismus der Quantenmechanik.  Es gibt also nur ein Elektron, was das Problem der Elektron-Elektron-Wechselwirkung umgeht.


\subsection{Born-Oppenheimer Näherung}

Atomkerne sind viel schwerer als Elektronen. In der Born-Oppenheimer Näherung betrachten wir die Kerne als stillstehend. Die Elektronen bewegen sich im stationären elektrischen Feld der Kerne. Diese Näherung wird quasi immer gemacht, so dass eigentlich nur erwähnt wird, wenn sie \emph{nicht} eingesetzt wird. Formal bedeutet dies, dass die Wellenfunktion des Moleküls geschrieben werden kann als Produkt der Wellenfunktion aller Elektronen und der Wellenfunktion aller Kerne, also
\begin{equation}
\Psi_{\text{Molekül}}(\mathbf{r}_1, \mathbf{r}_2, \dots, \mathbf{R}_1, \mathbf{R}_2, \dots)
  \approx
  \Psi_{\text{Elektronen}}(\mathbf{r}_1, \mathbf{r}_2, \dots )
\Psi_{\text{Kerne}}( \mathbf{R}_1, \mathbf{R}_2, \dots)
\end{equation}
wobei $\mathbf{r}_i$ Elektronenkoordinaten sind und $\mathbf{R}_i$ Kernkoordinaten.


Wir lösen also die Schrödingergleichung für freie Elektronen-Koordinaten, aber die Kern-Koordinaten werden als fix angenommen. Das \emph{Bindungspotential} stellt die Gesamtenergie des Systems dar, wenn für jeden Punkt der Kurve ein anderer aber jeweils fester Kern--Kern--Abstand angenommen wird. Eine Bindung kommt dann zustande, wenn das Bindungspotential ein Minimum hat. Der Kern--Kern--Abstand ist dann der Bindungsabstand.

% \begin{questions} 
% \item Was ist im Bindungspotential gebunden?
% \end{questions}

\subsection{Schrödinger-Gleichung und Variationsprinzip}

Wir benutzen also die Born-Oppenheimer-Näherung. Die Kerne bewegen sich nicht und tragen somit auch nicht zur kinetischen Energie bei. Der Abstand des einzigen Elektrons zu den beiden Kernen sei $r_1$ und $r_2$. Der Hamilton-Operator des Gesamtsystems ist
\begin{equation}
\hat{H} =  - \frac{\hbar^2}{2 m} \nabla^2 - \frac{e^2}{4 \pi \epsilon_0} \frac{1}{r_{1}} - \frac{e^2}{4 \pi \epsilon_0} \frac{1}{r_{2}}
= \hat{H}_1  - \frac{e^2}{4 \pi \epsilon_0} \frac{1}{r_{2}} \quad ,
\end{equation} 
wobei $\hat{H}_1 $ der Hamilton-Operator des Wasserstoff-\emph{Atoms} ist. Die Coulomb-Energie der beiden Kerne untereinander hängt nur vom Kern--Kern--Anstand ab und ist somit eine Konstante, die später zur Gesamtenergie addiert werden wird.


Die Schrödinger-Gleichung
\begin{equation}
 \hat{H} \ket{\Phi} = E_0 \, \ket{\Phi} 
\end{equation}
ist eine Differentialgleichung und in unserem Fall nicht einfach zu lösen. Hier hilft das Variationsprinzip. Für eine beliebige Wellenfunktion  $\ket{\Psi}$ gilt
\begin{equation}
 E = \frac{\braket{\Psi | H | \Psi}} {\braket{\Psi | \Psi}} \ge E_0 \quad .
 \label{eq:10_variation}
\end{equation}
Die Mathematik sagt, dass $E$ minimal wird, wenn  $\ket{\Psi}$ die Schrödinger-Gleichung löst. Aber auch wenn $\ket{\Psi}$ keine Lösung der Schrödinger-Gleichung  ist, kann man Gl.~\ref{eq:10_variation} einfach ausrechnen. Wir probieren  also verschiedene Test-Funktionen durch und versuchen, die Energie nach Gl.~\ref{eq:10_variation} zu minimieren. Dadurch nähern wir uns der echten Eigenfunktion immer mehr an, die Lösung der Schrödinger-Gleichung ist. Leider wissen wir nicht, ob wir  nicht durch noch bessere Test-Funktionen noch kleinere Werte von $E$ erreichen würde.


\subsection{Linear combination of atomic orbitals}

Wir suchen Molekül-Orbitale $\ket{\Psi}$, die mit $\hat{H}$ die Schrödinger-Gleichung lösen, und kennen bereits die Lösungen für $\hat{H}_1$:
\begin{equation}
\hat{H} \ket{\Psi} = E \ket{\Psi} \quad \text{und} \quad 
\hat{H}_1 \ket{\phi} = E_1 \ket{\phi}  \quad .
\end{equation}
Da die beiden Kerne identisch sind, gibt es solche Lösungen $\ket{\phi_2}$ in der gleichen Form aber zentriert um eine andere Kernposition auch für den zweiten Kern. Linearkombinationen von diesen  $\ket{\phi_{1,2}}$ nehmen wir jetzt als Testfunktion $\ket{\Psi}$. Dies nennt man \emph{linear combination of atomic orbitals} (LCAO).



Sei die Testfunktion
\begin{equation}
 \ket{\Psi} = c_1 \ket{\phi_1} + c_2 \ket{\phi_2} \label{eq:10_psi}
\end{equation}
mit normierten  $\ket{\phi_i}$ und reell-wertigen Koeffizienten $c_i$. Damit erhält man
\begin{eqnarray}
\braket{\Psi | \Psi}  &= & c_1^2 + c_2^2  + 2 c_1 c_2 \underbrace{\braket{\phi_1 | \phi_2}}_{= S}\\
\braket{\Psi |  H | \Psi} &=& c_1^2 \underbrace{\braket{\phi_1 |  H | \phi_1 }}_{= H_{11}} +
										c_2^2 \underbrace{\braket{\phi_2 |  H | \phi_2 }}_{= H_{22}} +
								2 c_1 c_2 \underbrace{\braket{\phi_1 |  H | \phi_2 }}_{= H_{12}}  \quad .
\end{eqnarray}
Dabei bezeichnet $S$ das Überlapp-Integral der beiden Wellenfunktionen, und $H_{ij}$ die Matrix-Elemente des Hamilton-Operators. Die Diagonalelemente $H_{11}$ und $H_{22}$ geben die Coulomb-Energie an, die Außerdiagnoalelemente $H_{12} = H_{21}$ die Austausch-Energie\sidenote{gleich mehr zu den Namen}. Man berechnet die Energie (Gl. \ref{eq:10_variation}), leitet nach den $c_i$ ab und findet die minimale Energie bei (Details in jedem Buch zur Quantenmechanik)
\begin{equation}
E_\pm = \frac{H_{11} \pm H_{12}}{1 \pm S} \quad ,\label{eq:10_e_variation}
\end{equation}
wobei wir im letzten Schritt angenommen haben, dass $H_{11} = H_{22}$.

% In diesem Fall sind die Koeffizienten $c_i$
% \begin{equation}
% c_1 = \pm c_2 = \frac{1}{\sqrt{2 (1 \pm S)}} \quad ,
% \end{equation}
% weil  ja ${\braket{\Psi | \Psi}}  = c_1^2 + c_2^2 + 2 c_1 c_2 S = 1$ sein soll.


\subsection{Drei Integrale}


\paragraph{Überlappintegral $S$} 
\begin{marginfigure}
\inputtikz{\currfiledir integrals_s}
\caption{Skizze   Überlappintegral $S$. }
\end{marginfigure}
%
Das Integral $S$ beschreibt den räumlichen Überlapp der beiden Atom-Wellenfunktionen, wenn die einen um Kern 1, die andere um Kern 2 zentriert ist:
\begin{equation}
 S = \braket{\phi_1 | \phi_2} = \int \phi_1^\star( \mathbf{r} )  \, \phi_2( \mathbf{r})   \, d\mathbf{r} \quad .
\end{equation}
Dabei bezeichnet $\mathbf{r}$ die Position des Elektrons. Die Wellenfunktion $\phi_i$ ist um den Kern an Position $\mathbf{r}_{i}$ zentriert.\sidenote{Wasserstoff-Wellenfunktionen sind reell-wertig.} Da die $\ket{\phi}$ normiert sind, liegt der Wert von $S$ zwischen $0$ und $1$.





\paragraph{Coulomb-Wechselwirkung $H_{11}$}  
\begin{marginfigure}
\inputtikz{\currfiledir integrals_c}
\caption{Skizze Coulomb-Integral $C$ }
\end{marginfigure}
%
Dieser Term beschreibt die Coulomb-Energie des Elektrons in der atomaren Wellenfunktion $\phi_1$, aber in Gegenwart beider Kerne:
\begin{eqnarray}
H_{11} &= &  \braket{\phi_1 | \hat{H} | \phi_1} = \braket{\phi_1 | \hat{H}_1 | \phi_1}  - \braket{\phi_1 |  \frac{e^2}{4 \pi \epsilon_0} \frac{1}{r_{2}} | \phi_1}  \\
 & = & E_1 - \frac{e^2}{4 \pi \epsilon_0} \int \frac{|\phi_1(\mathbf{r})|^2 }{|\mathbf{r} - \mathbf{r}_2  |} \, d\mathbf{r} = E_1 + C \quad .
\end{eqnarray} 
Das Ergebnis ist die Eigen-Energie des Elektrons im Wasserstoff-\emph{Atom}, korrigiert im ein Überlappintegral der Ladungsdichte ${|\phi_1(\mathbf{r})|^2 }$ um den einen Kern im Coulomb-Potential des anderen Kerns. Der Korrekturterm $C$ ist negativ.




\paragraph{Austausch-Wechselwirkung $H_{12}$} 
Die Austausch-Wechselwirkung ist ein rein quantenmechanischer Effekt.
\begin{eqnarray}
H_{12} &= &  \braket{\phi_1 | \hat{H} | \phi_2} = \braket{\phi_1 | \hat{H}_1 | \phi_2}  - \braket{\phi_1 |  \frac{e^2}{4 \pi \epsilon_0} \frac{1}{r_{2}} | \phi_2}  \\
 & = & E_1 \, S - \frac{e^2}{4 \pi \epsilon_0} \int \frac{ \phi_1^\star(\mathbf{r}) \, \phi_2(\mathbf{r})  }{|\mathbf{r} - \mathbf{r}_2  |} \, d\mathbf{r} = E_1 \, S + A \quad .
\end{eqnarray}
Die Austausch-Dichte $\phi_1^\star(\mathbf{r}) \, \phi_2(\mathbf{r})$ ist ähnlich einer Ladungsdichte $|\phi(\mathbf{r})|^2$, nur dass zwei verschiedenen Wellenfunktionen eingehen. Das Elektron wechselt sozusagen zwischen der Zugehörigkeit zu Kern 1 und 2. Der Korrekturterm $A$ ist ebenfalls negativ.

\begin{marginfigure}
  \inputtikz{\currfiledir integrals_a}
  \caption{Skizze Austausch-Integral $A$.}
  \end{marginfigure}
  %


\subsection{Bindungspotential}

Mit diesen Integralen wird die Gesamtenergie
\begin{equation}
E_\pm = \frac{H_{11} \pm H_{12}}{1 \pm S} = E_1 + \frac{C \pm A}{1 \pm S} \quad .
\end{equation}
Die zugehörigen Molekül-Orbitale sind die symmetrische und die anti-symmetrische Kombination der Atom-Orbitale
\begin{equation}
\ket{\Psi_\pm }= \frac{1}{\sqrt{2 (1 \pm  S)}} \, \left( \ket{\phi_1} \pm \ket{\phi_2} \right) \quad .
\end{equation}

Zur Berechnung der Bindungsenergie nehmen wir jetzt die nur vom Kern--Kern--Abstand $R$ abhängende Coulomb-Energie der Kerne wieder hinzu. 
%
\begin{eqnarray}
 E_\text{Bindung} &=&  E_\text{Molekül} -  E_\text{Atom} \\
  &=&   E_1 + \frac{C \pm A}{1 \pm S} + \frac{e^2}{4 \pi \epsilon_0} \frac{1}{R} - E_1 \\
   &=&\frac{C \pm A}{1 \pm S} + \frac{e^2}{4 \pi \epsilon_0} \frac{1}{R}  = \frac{C' \pm A'}{1 \pm S}  \quad , \label{eq:10_E_bindung_h2p}
\end{eqnarray}
mit der Definition  von $C'$ und $A'$ wie in Abbildung \ref{fig:10_H2_integrale_r}.
Numerische Rechnungen zeigen, dass das Überlapp-Integral $S$ keinen entscheidenden Einfluss auf das Ergebnis hat, wir es hier also nicht weiter betrachten müssen.\sidenote{Für H$_2^+$ lassen sich relativ einfache geschlossene Formen für die Integrale angeben, siehe \cite{McQuarrie2008} }.

Das Coulomb-Integral $C$ ist für einen  großen Kern--Kern--Abstand $R$ quasi die Energie einer Punkt-Ladung im Potential des anderen Kerns, da die Ausdehnung der Wellenfunktion $\phi_1$ vernachlässigt werden kann. Da $C$ negativ ist, geht $C'$ gegen Null. Für kleine Kern--Kern--Abstände $R$ bleibt $C$ negativ und endlich, da die potentielle Energie eines Elektrons im Wasserstoff-Atom endlich ist. Der zweite Summand von  $C'$ strebt aber mit $1/R$ gegen positiv unendlich. Die Summe der ersten beiden Terme ist also entweder Null oder positiv, so dass kein lokales Minimum zustande kommt.



\begin{marginfigure}
\inputtikz{\currfiledir integrale_von_r}

\caption{Abhängigkeit der Integrale vom Kern--Kern--Abstand $R$. Dargestellt ist 
$C' = C  + \frac{e^2}{4 \pi \epsilon_0} \frac{1 }{R}$ bzw. $A' = A   + \frac{e^2}{4 \pi \epsilon_0} \frac{ S }{R}$. \label{fig:10_H2_integrale_r}
 }
\end{marginfigure}



Den entschiedenen Beitrag liefert das Austausch-Integral $A$. Für große $R$ ist das Austausch-Integral und auch $A'$ wieder Null. Für kleine Abstände $R$ ist das Austausch-Integral sehr ähnlich dem Coulomb-Integral und endlich negativ. Dazwischen ist es in einem gewissen Bereich von $R$ negativ genug, dass bei positivem Vorzeichen in Gl. \ref{eq:10_E_bindung_h2p} die Bindungsenergie negativ wird, eine Bindung also zustande kommt.

Damit ist also $\Psi_+$ das bindende Orbital. Da es aus Wasserstoff-1s-Orbitalen zusammengesetzt ist, ist es ein $\sigma$-Orbital. $\Psi_-$ ist ein anti-bindendes $\sigma^\star$-Orbital. Die Skizze zeigt die Gesamt-Energie als Funktion des Kern--Kern--Abstands $R$. Dies wird als \emph{Bindungspotential} bezeichnet. Für das bindende Orbital sind sehr kleine $R$ durch das Pauli-Verbot ausgeschlossen.
Der Bindungsabstand $R_0$ ist der Abstand minimaler Energie. Das Potential kann in seiner Umgebung durch eine harmonisches Parabel-Potential genähert werden. Die Energie $E(R_0)$ bestimmt die Stärke de Bindung, also wieviel Energie aufgebracht werden muss, um die beiden Atome zu trennen. Die nächsten beiden Kapitel zur Spektroskopie von Molekülen beschäftigt sich eigentlich nur mit Methoden, wie die verschiedenen Parameter dieses Bindungspotentials experimentell bestimmt werden können.



\begin{marginfigure}
\inputtikz{\currfiledir potentiale}

\caption{Skizze des Bindungspotentials $E_{\text{Bindung}, \pm}$ vom Kern--Kern--Abstand $R$. Das bindende Potential $E_+$ zeigt ein Minimum bei $R_0$, das anti-bindende Potential $E_-$ hat nur ein Minimum im Unendlichen.}
\end{marginfigure}




% \begin{questions} 
% \item Die drei Integrale $S$, $C$ und $A$ sind von zentraler Bedeutung. Sie sollten sie sowohl als Gleichung als auch als Skizze darstellen können.

% \item Warum sagt man 'Die Austausch-Wechselwirkung ist ein rein quantenmechanischer Effekt' ?
% \end{questions}



\subsection{Das Austausch-Integral für verschiedene Atom-Orbitale}

Wir haben bisher nicht diskutiert, welche Form die Atom-Orbitale $\ket{\phi}$ denn eigentlich haben.
Im Wasserstoff-Molekül-Ion \ch{H2+}  werden es sicherlich s-Orbitale sein (was auch bei der Diskussion der Beiträge angenommen wurde). Bei anderen Orbitalen kann es zum Verschwinden des Austausch-Integrals $A$ kommen, und somit keine Bindung geben.

\begin{marginfigure}
\inputtikz{\currfiledir orbitale_s_p}

\caption{Je nach Art und Orientierung der beteiligten Orbitale kann das Austausch-Integral $A$ auch verschwinden. Die Farben kodieren das Vorzeichen der Wellenfunktion. }
\end{marginfigure}



Ein Beispiel ist das Austausch-Integrals zwischen  einem s-Orbital und einem p$_x$-Orbital, wenn $z$ die Kern--Kern--Achse ist.  Die beiden Keulen des  p$_x$-Orbitals tragen mit unterschiedlichem Vorzeichen zum Austausch-Integral bei und kompensieren sich so. In diesem Fall wäre $A$ Null. Wenn hingehen ein p$_z$-Orbital mit einem s-Orbital überlappt, dann verschwindet das  Austausch-Integral $A$ nicht.


\section{Anschauliche Argumente für eine chemische Bindung}

Kann man anschaulich verstehen, warum das Wasserstoff-Molekül-Ion \ch{H2+} existiert, also energetisch günstiger ist als ein Wasserstoff-Atom und ein freies Proton? Aus meiner Sicht gibt es zwei bis drei Wege.

\paragraph{Teilchen im Kasten}  Man kann das Molekül-Orbital $\Psi_+$ als Kasten für das Elektron sehen, auch wenn die Wände nicht senkrecht und unendlich hoch sind. Die Energie des niedrigsten Zustands in einem eindimensionalen  Kasten-Potential ist proportional zu $1/L$, mit der Kastenlänge $L$. Das Molekül bildet einen größeren Kasten als das Atom, darum sinkt die Energie für das Elektron und es kommt zur Bindung. Das zeigt schon die Abbildung ganz am Anfang des Kapitels.


\paragraph{Elektronen-Dichte-Verteilung} Im symmetrischen Molekülorbital $\Psi_+ \propto \phi_1 + \phi_2$ ergibt sich ein deutlich von Null verschiedener Wert der Elektronendichte $|\Psi_+|^2$ in der Mitte zwischen den beiden Kernen. Diese negative Ladungsdichte schirmt den positiven Kern vom anderen positiven Kern ab. Die Coulomb-Abstoßung der Kerne ist also geringer, als wenn das Elektron in einem s-Orbital um einen Kern alleine  wäre. Im $\Psi_-$-Orbital ist dies nicht mehr der Fall. Hier ist die Elektronen-Dichte zwischen den Kernen geringer, in der Mitte der Strecke sogar exakt Null.
%
\begin{marginfigure}[-50mm]
\inputtikz{\currfiledir wf_bonding}
\caption{ Wellenfunktion (dünne Linie) und Ladungsdichte (dicke Linie) der bindenden Wellenfunktion $\Psi_+$ und der  anti-bindenden Wellenfunktion $\Psi_-$.}
\end{marginfigure}




\paragraph{Quantenmechanische Interferenz} Die Ladungsdichte in einem Molekül-Orbital ist $| \phi_1 + \phi_2 |^2$, wenn das Orbital aus den beiden Atom-Orbitalen $\phi_1$ und $\phi_2$ aufgebaut ist. Die Ladungsdichte ist damit \emph{nicht} die Summe der Ladungsdichten der beiden Atom-Orbitale, also nicht $| \phi_1 |^2 +| \phi_2 |^2$. Quantenmechanische Wellenfunktionen interferieren, werden also addiert bevor das Betrags-Quadrat gebildet wird. Dies ermöglicht Auslöschung (im Fall von $\Psi_-$) und konstruktive Interferenz (im Fall von $\Psi_+$), wodurch obiges Elektronendichte-Argument zum Tragen kommt und  die chemische Bindung ermöglicht wird.








\section{Anhang: Mehr als zwei Atom-Kerne: Hückel-Näherung}

Ähnlich wie  \ch{H2^+} kann man auch größere Moleküle behandeln, das verlangt dann aber  numerischen Lösungen. Für konjugierte Moleküle liefert die Hückel-Näherung aber gute Ergebnisse. In konjugierten Molekülen wird das mechanische Gerüst durch $\sigma$-Bindungen zwischen den Kohlenstoff-Atomen gebildet. Eine Kette von Kohlenstoff-Atomen ist darüber hinaus durch alternierende $\sigma$ und $\pi$-Bindungen verbunden. Die an diesen Bindungen beteiligten Elektronen sind dann über die ganze Kette delokalisiert. Die Hückel-Näherung erlaubt es, diese ausgedehnten  $\pi$-Orbitale  zu berechnen.

Wir betrachten also nur eine Teilmenge aller Atom-Orbitale, nur die $\pi$-Orbitale, die auch an der $\pi$-Bindung teilnehmen. Wir nehmen an, dass
\begin{itemize} \setlength{\itemsep}{0pt}
\item die Atom-Orbitale nur mit sich selbst überlappen, also $S_{ij} = \delta_{ij}$
\item alle Atome identisch sind, also $H_{ii} = \alpha$
\item Austausch nur zwischen benachbarten Orbitalen stattfinden, also  $H_{ij} = \beta < 0 $ falls Atome $i$ und $j$ benachbart, sonst $0$ 
\end{itemize}

Analog zu Gleichung \ref{eq:10_e_variation} oben berechnen wir die Eigen-Energie nach dem Variationsprinzip
\begin{equation}
 E = \frac{  \sum_{i,j} c_i \, c_j \, H_{i,j} }{ \sum_{i,j} c_i \, c_j \, S_{i,j} } \quad .
\end{equation}
Die minimale Eigen-Energie $E$ ergibt sich, wenn alle partiellen Ableitungen nach den $c_i$ Null sind, oder wenn
\begin{equation}
 \left| \mathbf{H} - E \, \mathbf{S}\right| = 0 \quad .
\end{equation}
Da wir $S_{ij} = \delta_{ij}$ angenommen haben, vereinfacht sich dies zu 
\begin{equation}
 \left| \mathbf{H} - E \, \mathds{1} \right| = 0 \quad .
\end{equation}
Wir müssen also die Eigenwerte und Eigenvektoren von $H_{i,j}$ bestimmen. Die Eigenwerte geben die Energie des Zustands an, die Eigenvektoren die dazugehörige  Linearkombination der atomare Orbitale.

\begin{marginfigure}[20mm]
\inputtikz{\currfiledir benzol}
\caption{Molekülorbitale von Benzol in der Hückel-Näherung. Die Farben kodieren das Vorzeichen der Wellenfunktion. Die Anordnung entspricht der Eigen-Energie.\label{fig:10_benzol}}
\end{marginfigure}

Als Beispiel betrachten wir Benzol (\ch{C6H6}). Die 6 Kohlenstoff-Atome sind sp$^2$ hybridisiert. $\sigma$-Bindungen verbinden die Kohlenstoff-Atome untereinander und mit den Wasserstoff-Atomen. Je ein nicht hybridisiertes p-Orbital steht senkrecht auf dem Ring. Diese Orbitale werden in der Hückel-Näherung betrachtet. Die Hamilton-Matrix $H_{ij}$ hat dann die Form (Nullen weggelassen)
\begin{equation}
\mathbf{H} = 
 \begin{pmatrix}
  \alpha  & \beta &  &  &  & \beta \\
  \beta & \alpha  & \beta & & & \\
  & \beta & \alpha  & \beta & & \\
 &  & \beta & \alpha & \beta & \\
&  &  & \beta & \alpha & \beta \\
\beta & &  &  & \beta & \alpha 
 \end{pmatrix}  \quad .
\end{equation}
Die $\beta$ in den Ecken schließen den Ring.
Wenn wir $E = \alpha + x \beta$ ansetzen, dann vereinfacht sich die Eigenwert-Gleichung zu 
\begin{equation}
x^6 - 6 x^4 + 9x^2 - 4 = 0 \quad \text{oder} \quad x = \pm 1, \pm 1, \pm 2 \quad .
\end{equation}
Wie man das numerisch macht sehen Sie im   Pluto-Skript\pluto{hueckel}.



Da wir insgesamt 6 Elektronen in diese Orbitale einfüllen müssen, und jedes Orbital mit 2 Elektron (spin up und down) besetzen können, sind das Orbitale mit $E=\alpha + 2 \beta$ und die beiden Orbitale mit $E = \alpha + \beta$ besetzt\sidenote{$\beta < 0$}. Auch diese Orbitale tragen also zur Bindung bei, da sie die Gesamtenergie insgesamt um $8\beta$ reduzieren. Wenn man die Eigenfunktionen betrachtet\footcite{Atkins}, sieht man, dass  das Orbital mit $E=\alpha \pm 2 \beta$  über den ganzen Ring delokalisiert ist, die beiden mit $E = \alpha \pm \beta$  über zwei  Atome.


Die Hückel-Näherung in der Molekülphysik entspricht der \emph{tight binding} Methode zur Berechnung der Bandstruktur von Elektronen  in der Festkörperphysik. In der Festkörperphysik macht man den Übergang von hier $N=6$ Atomen hin zu $N= 6 \cdot 10^{23}$ Atomen, wodurch dann  $6 \cdot 10^{23}$ eng benachbarte Zustände für Elektronen entstehen, die alle durch Wellenfunktionen ähnlich zu Abbildung \ref{fig:10_benzol} beschrieben sind.

%https://en.wikipedia.org/wiki/H%C3%BCckel_method#Delocalization_energy,_%CF%80-bond_orders,_and_%CF%80-electron_populations

% \begin{questions} 
% \item Vergleichen Sie die Elektronen-Eigenfunktionen von Benzol in der Hückel-Näherung mit denen eines (ggf. ringförmigen) Kastens.
% \end{questions}



\newpage

\section{Zusammenfassung}

\textit{Schreiben Sie hier ihre persönliche Zusammenfassung des Kapitels auf. Konzentrieren Sie sich auf die wichtigsten Aspekte.}

\vspace*{10cm}


%--------------------
\printbibliography[segment=\therefsegment,heading=subbibliography]

% \renewcommand{\lastmod}{10. September 2024}
\renewcommand{\chapterauthors}{Markus Lippitz}

\chapter{Elementare Anregungen in Molekülen}



\goal{By the end of this chapter, you should be able to draw, calculate and align a ray's path through an optical system.}



\section{Overview}

s.a. Demtröder 3, Kap. 9


11.1 Rotation: IR Absorption [6] 1	

11.2 Schwingung: IR Absorption [6] 2	 

11.3 Schwingung: Raman Streuung 3	 

11.4 Elektronische Anregung: Absorption 4	 

11.5 Elektronische Anregung: Fluoreszenz 5	 



\section{Zusammenfassung}

\textit{Schreiben Sie hier ihre persönliche Zusammenfassung des Kapitels auf. Konzentrieren Sie sich auf die wichtigsten Aspekte.}

\vspace*{10cm}


%--------------------
\printbibliography[segment=\therefsegment,heading=subbibliography]



% %-------
\renewcommand{\kapitelname}{Anhang\ }

\addcontentsline{toc}{part}{Anhang} 
\appendix
\appendixpage

\renewcommand{\chapterauthors}{Markus Lippitz}
\renewcommand{\lastmod}{5. Oktober 2021}

\chapter{Addition von Drehimpulsen}

\label{chap:anhang_drehimpuls}

\section{Überblick}

Der ausführliche Titel des Kapitels sollte wohl sein 'Eigenwerte und Eigenfunktionen eines Operators, der die Summe von quantenmechanischen Drehimpulsoperatoren ist'. Dieses Thema findet sich in quasi allen Büchern zur Quantenmechanik. Ich folge hier \cite{Nolting-QM}, Kap. 5.4. Mit diesem Formalismus kann man beispielsweise die möglichen Werte der Gesamtspin-Quantenzahl $S$ bestimmen, wenn die Orientierung der Einzel-Spins bekannt ist. Oder man kann in der Atomphysik den Gesamt-Drehimpuls-Quantenzahl $J$ aus der Spin-QZ $S$ und der Bahndrehimpuls-QZ $L$ bestimme. Ebenso erhält man die Eigenfunktionen eines Singulett- oder Triplett-Zustands.

\section{Der Drehimpuls-Operator}

In der Quantenmechanik definiert man einen Drehimpuls-Operator $\hat{L}$, der den Betrag eines Drehimpulses misst, sowie einen Operator $\hat{L}_z$, der eine der drei Vektor-Komponenten misst. Die Eigenwerte sind 
\begin{align}
	\hat{L}^2 \ket{l, m}  = & \hbar^2 \, l (l +1 ) \ket{l, m} \quad 
	\text{mit} \quad l = 0, 1, \dots \\
	\hat{L}_z \ket{l, m} = & \hbar \, m \ket{l, m} \quad 
	\text{mit} \quad m = -l, -l +1, \dots , l
\end{align}
Die Quantenzahl $m$ nennt man auch magnetische Quantenzahl (daher das Symbol), weil die z-Achse in Atomen oft durch die Richtung eines externen Magnetfelds vorgegeben ist.

Die Kommutator-Relationen sind so, dass $\hat{L}$ und $\hat{L}_z$ gleichzeitig messbar sind, aber die  einzelnen Vektor-Komponenten nicht. Die Unschärfe in den verbleibenden Komponenten beträgt dann
\begin{equation}
\Delta L_x \, \Delta L_y \ge \frac{\hbar}{2} \left| \braket{\hat{L}_z} \right|
\end{equation}

Man kann sich einen Drehimpuls-Vektor in der Quantenmechanik also als einen Vektor der Länge $\hbar \sqrt{l (l+1)}$ vorstellen, dessen z-Komponente $\hbar m$ ist. Glücklicherweise ist der Maximalwert von $m$, also $l$, immer kleiner als $\sqrt{l (l+1)}$, und $l+1$ immer größer als das.
Die x- und y-Komponenten ist unbekannt, bis auf dass sie gerade die erforderliche Länge des Vektors liefern müssen. Mögliche Werte dieser beiden Komponenten liegen damit auf einem Kreis in der xy-Ebene.

\begin{marginfigure}
\inputtikz{\currfiledir vector3d}
\caption{Skizze eines Drehimpulsvektors mit unbekannter xy-Komponente.}
\end{marginfigure}

Es gibt nicht nur Vektoren, die einem klassischen Drehimpuls entsprechen, sondern auch anderen Größen, die sich sehr ähnlich einem Drehimpuls verhalten, wie beispielsweise der Spin. Erstere haben immer ganzzahlige Quantenzahlen $l$,$m$, letztere können auch halbzahlig sein. Immer ist der Abstand zwischen benachbarten Quanten zahlen aber eins. Ich benutze das Wort Drehimpuls hier immer als Oberbegriff für beides.

\begin{marginfigure}
\inputtikz{\currfiledir vector2d}

\caption{Mögliche Orientierung von  Drehimpuls-artiger Vektoren mit $l=1/2$ (links) und $l=2$ (rechts). Der Abstand der Hilfslinien beträgt $1/2 \hbar$ bzw. $1\hbar$.}
\end{marginfigure}

\section{Addition von Drehimpulsen}


Jetzt haben wir zwei Sätze von Drehimpuls-artigen Operatoren, und kennen deren Eigenwerte und Eigenfunktionen, also
\begin{align}
	\hat{L}^2_1 \ket{l_1, m_1}  = & \hbar^2 \, l_1 (l_1 +1 ) \ket{l_1, m_1} &
	\hat{L}_{z,1} \ket{l_1, m_1} = & \hbar \, m_1 \ket{l_1, m_1} \\
	%
		\hat{L}^2_2 \ket{l_2, m_2}  = & \hbar^2 \, l_2 (l_2 +1 ) \ket{l_2, m_2} &
	\hat{L}_{z,2} \ket{l_2, m_2} = & \hbar \, m_2 \ket{l_2, m_2} 
\end{align}
Wir können dann Summen-Operatoren bilden
\begin{equation}
\hat{L} = \hat{L}_1 + \hat{L}_2 \quad \text{und} \quad\hat{L}_{z} = \hat{L}_{z,1} + \hat{L}_{z,2}
\end{equation}
Diese neuen Operatoren sind glücklicherweise wieder Drehimpuls-Operatoren, folgen also den üblichen Anforderungen der Quantenmechanik an solche Operatoren in Bezug auf die Kommutator-Relationen und die Form der Eigenwerte. Die Frage ist nun, wie man aus bekannten Eigenwerten $l_i$, $m_i$ und dazugehörigen Eigenfunktionen auf die neuen Eigenwerten $l$, $m$ der Summen-Operatoren schließen kann, und welche Werte eigentlich gleichzeitig messbar sind.

Man findet, dass die Gesamt-Länge zusammen mit den beiden Einzel-Längen, aber nur mit der Orientierung des Gesamt-Drehimpulses gleichzeitig messbar ist. Gute\sidenote{'Gut' ist in diesem Zusammenhang ein Fachbegriff und bedeutet 'Konstante der Bewegung', also unveränderlich.} Quantenzahlen sind also 
\begin{equation}
\ket{l_1, l_2; l , m }
\end{equation} 
Die neue Orientierungs-Quantenzahl $m$ ist gerade die Summe der Einzeln-Orientierungs-Quantenzahlen
\begin{equation}
 m  = m_1 + m_2
\end{equation}
Für die  neue Gesamt-Länge gilt
\begin{equation}
 | l_1 - l_2 | \le l \le l_1 + l_2
\end{equation}
Mehr lässt sich dazu leider nicht sagen. Es ist etwas unbefriedigend, die Summe von zwei Vektoren nicht nennen zu können, obwohl man beide Summanden kennt. Allerdings kennt man die Ausgangs-Vektoren nicht vollständig. Die unbekannte xy-Komponenten sind gerade der Ursprung dieses Spielraums im Wert von $l$.


\section{Beispiel: $\vec{\mathbf{J}} \mathbf{=} \vec{\mathbf{S}} \mathbf{+}  \vec{\mathbf{L}} $}

Was bedeutet es, dass die guten Quantenzahlen $\ket{l_1, l_2; l , m }$ sind? Ich möchte das mit dem  Beispiel der Addition von Bahndrehimpuls $\vec{L}$ und Spin $\vec{S}$ zum Gesamtdrehimpuls $\vec{J}$ diskutieren (und passe dabei die Bezeichnungen leicht an). Gute Quantenzahlen sind also $\ket{L, S; J , m_J }$. Die großen Buchstaben sind die Quantenzahlen, die die Länge der Vektoren in der Form $\hbar \sqrt{l (l+1)}$ angeben, $m_J$ ist die magnetische Quantenzahl zu $J$.


Bei der Kopplung von Spin und Bahndrehimpuls gibt es einen Energiebeitrag des Spins im Magnetfeld der Bahnbewegung. Klassisch würde dieser vom Winkel zwischen den beiden abhängen. Dieser Winkel ist aber nicht die Quantenzahl, sondern das sich aus $\vec{S}$, $\vec{L}$ und $\vec{J}$ bildende Dreieck wird vollständig durch die Längen der Seiten bestimmt. Das beinhaltet den Winkel zwischen $\vec{S}$ und $\vec{L}$, aber auch deren Amplitude. Gleichzeitig ist nur $m_J$ eine gute Quantenzahl. Bei der Wechselwirkung mit einem äußeren Feld spielt also nur die Orientierung von $\vec{J}$ eine Rolle. Die Spitze von $\vec{J}$ kann wieder auf einem Kreis in der xy-Ebene liegen, solange die Länge von $\vec{J}$ erhalten bleibt. Bei $\vec{S}$ und $\vec{L}$ ist nun aber \emph{nur} die Länge eine gute Quantenzahl, die z-Komponenten nicht mehr. Die Spitze von $\vec{S}$ kann damit auf einem Kreis liegen, dessen Symmetrieachse durch $\vec{J}$ gegeben ist. Alles andere ist unbekannt, kann nicht gleichzeitig gemessen werden. Insbesondere ist die Aufteilung zwischen $m_S$ und $m_L$ nicht fix, nur die Summe, also $m_J$.

\section{Beispiel: Addition von zwei Vektoren mit Spin 1/2 }

Als Beispiel wollen wir die beiden kürzesten Drehimpulse addieren, was die Zeichnungen einfacher macht. Dies entspricht der Addition von zwei Elektronen-Spins zu einem Gesamtspin. Der allgemeine Formalismus folgt dann unten.
Es sei
\begin{equation}
l_{1,2} = \frac{1}{2} \quad \text{und} \quad m_{1,2} = \pm \frac{1}{2} 
\end{equation}
Welche Werte können nun die Quantenzahlen  $l$ und $m$ der Summe annehmen? Die magnetische Quantenzahl $m  = m_1 + m_2$ ist einfach und in nebenstehender Tabelle skizziert.
%
\begin{marginfigure}
\begin{tabular}{r|rr}
                           & $-\frac{1}{2} $  & $+\frac{1}{2} $ \\
                           \hline
 $+\frac{1}{2} $    &     $0$              & $1$ \\
 $-\frac{1}{2} $    &     $-1$              & $0$ 
\end{tabular}
\caption{Die möglichen Kombinationen von $m_1$ und $m_2$ zu $m = m_1 + m_2$.}
\end{marginfigure}
%
Falls $m_1 = m_2$, also $|m| = 1$, dann muss auch $l = 1$ sein, da $l$ nie kleiner als $m$ sein kann. Dies sind die Zustände\sidenote{$l_1$ und $l_2$ sind nicht angegeben, weil in diesem Abschnitt immer $1/2$.} $\ket{l, m} = \ket{1, -1} $ und $\ket{1, +1} $.

Damit verbleiben noch die beiden Fälle $m_1 = - m_2$, also die Diagonale in der Tabelle. Diese müssen die Zustände $\ket{1, 0}$ und $\ket{0,0}$ bilden. Die Gesamtzahl der Zustände passt schon einmal. Wie oft in der Quantenmechanik, wenn die Zuordnung nicht einfach entschieden werden kann, werden hier wieder die symmetrische und antisymmetrische Superposition der Ausgangszustände, also der Einträge in der Matrix, gebildet. Welche davon wird $\ket{1, 0}$? Die schon gefundenen Zustände $\ket{1, \pm 1} $ sind symmetrisch bei Vertauschen $1 \leftrightarrow 2$, also wird auch $\ket{1, 0}$ symmetrisch sein, also 
\begin{equation}
\ket{1, 0} = \frac{1}{\sqrt{2}} \left( \ket{\uparrow \downarrow} +  \ket{\downarrow \uparrow} \right)
\end{equation}
wobei der Pfeil an Position $i$ das Vorzeichen von $m_i$ anzeigt.
Damit gibt es einen anti-symmetrischen Zustand mit $l = 0$, und drei symmetrische mit $l=1$
\begin{align}
\ket{0, 0} = & \frac{1}{\sqrt{2}} \left( \ket{\uparrow \downarrow} -  \ket{\downarrow \uparrow} \right) \\
\ket{1, +1} =& \ket{\uparrow \uparrow}  \\
\ket{1, 0} = &\frac{1}{\sqrt{2}} \left( \ket{\uparrow \downarrow} +  \ket{\downarrow \uparrow} \right) \\
\ket{1, -1} = &\ket{\downarrow \downarrow}  
\end{align}
Die Vorfaktoren, mit denen man die Zustände auf der linken Seite in der Basis der Zustände auf der rechten Seite darstellen kann, nennt man \emph{Clebsch-Gordan-Koeffizienten}. 

Wie kann man sich vorstellen, dass die Addition von zwei Vektoren gleicher Länge aber unterschiedlicher Orientierungs-Quantenzahl $m_i$ einmal zu einem Vektor der Länge Null und einmal zu einem Vektor der beinahe doppelten Länge führt? Ein Teil der Wahrheit sind die nicht gleichzeitig messbaren anderen Vektor-Komponenten.\sidenote{Ein anderer Teil ist 'so ist die QM eben'.} Die Spitze beider Vektoren liegt auf eine Kreis. Wenn die Position 'in Phase' ist, dann addieren sie sich zu einem Vektor mit verschwindender z-Komponente und der Länge $\hbar \sqrt{2}$, was in diesem Bild dem Zustand $\ket{1,0}$ entspricht. Wenn die beiden Ausgangs-Vektoren 'außer Phase' sind, dann addieren sie sich zu Null, ergeben also  $\ket{0,0}$. Bei bekannten, aber unterschiedlichen $m_i$, also beispielsweise $\ket{\uparrow \downarrow}$ ist also nicht eindeutig, welcher Summenvektor sich ergibt. Die Eigenfunktionen des Summen-Operators $\hat{L}$ sind nur Linearkombinationen aus $\ket{\uparrow \downarrow}$ und $\ket{\downarrow \uparrow}$.

\begin{marginfigure}
\inputtikz{\currfiledir vector3d_summe}

\caption{Die Addition von zwei Vektoren $\ket{s=1/2, m_s = 1/2}$ und  $\ket{s=1/2, m_s = -1/2}$ kann sowohl einen Vektor   $\ket{S=1, m_S = 0}$ ergeben (links) als auch $\ket{S=0, m_S = 0}$ (rechts).}
\end{marginfigure}




\section{Allgemeiner Fall: Clebsch-Gordan-Koeffizienten}

Im allgemeinen Fall der Addition von zwei Drehimpuls-artigen Vektoren mit den Quantenzahlen $l_i$ und $m_i$ bleiben nur die oben schon genannten Regeln
\begin{align}
   m =  m_1 + m_2 \\
   |l_1 - l_2 | \le l \le l_1 + l_2
\end{align}
Insgesamt sind es $(2 l_1 + 1) (2 l_2 +1)$ Eigenfunktionen. Die Parität ist
\begin{equation}
\mathcal{P} = (-1)^{l - l_1 - l_2}
\end{equation}
Die Clebsch-Gordan-Koeffizienten zur Darstellung der Eigenfunktionen des Summen-Operators in den Eigenfunktionen der beiden Einzel-Drehimpuls-Operatoren kann man sich mit einer Rekursionsregel herleiten. Einfacher ist es aber, diese nachzuschlagen, beispielsweise in \cite{ParticleDataGroup20}, bzw. online \href{ https://pdg.lbl.gov/2020/reviews/rpp2020-rev-clebsch-gordan-coefs.pdf}{hier}.
 Für unsere Zwecke reicht es aber aus, die Faktoren für das obige Spin-$1/2$-System zu kennen.

\begin{marginfigure}
\includegraphics[scale=1]{\currfiledir clebsch-gordan-1x1.pdf}
\caption{Clebsch-Gordan-Koeffizienten für $l_1 = l_2 = 1$. Die QZ des Gesamt-Drehimpulses sind hier als $J$ und $M$ bezeichnet. Aus \cite{ParticleDataGroup20}.}
\end{marginfigure}

Als Beispiel zeigt nebenstehende Abbildung die Koeffizienten für den Fall $l_1 = l_2 = 1$. Die QZ des Gesamt-Drehimpulses sind  als $J$ und $M$ bezeichnet. Die Koeffizienten sind, um Platz zu sparen, ohne die Wurzel geschrieben. $-1/3$ ist also als $-\sqrt{1/3}$ zu verstehen, bzw.
\begin{equation}
\ket{J = 0, M= 0} = \frac{1}{\sqrt{3}} \ket{+1, -1} - \frac{1}{\sqrt{3}} \ket{0,0} + \frac{1}{\sqrt{3}} \ket{-1, +1} 
\end{equation}
Man muss also immer eine Linearkombination aus allen Möglichkeiten bilden, die das gewünschte $  m =  m_1 + m_2 $ ergeben.



\printbibliography[segment=\therefsegment,heading=subbibliography]

\
%-------
%
%%\nocite{*}
 
\printbibliography



\end{document}
