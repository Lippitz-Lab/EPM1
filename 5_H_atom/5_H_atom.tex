\renewcommand{\lastmod}{10. September 2024}
\renewcommand{\chapterauthors}{Markus Lippitz}

\chapter{Quantentheorie des H-Atoms}



\goal{By the end of this chapter, you should be able to draw, calculate and align a ray's path through an optical system.}



\section{Overview}

s.a. Demtröder 3, Kap. 5

41.1 The Hydrogen Atom: Angular Momentum and Energy 1231

41.2 The Hydrogen Atom: Wave Functions and Probabilities 1234

41.3 The Electron's Spin 1237
Inkl. Stern-Gerlach, Zeeman-Effekt

5.3 Spin-Bahn-Kopplung bei Wasserstoff 3	***	

5.5 Pauli-Prinzip5	Wdh	5 

5.6 Addition von Drehimpulsen 6	Wdh	6 

5.7 Helium 7	**	

\phet{Models_of_the_Hydrogen_Atom}

\url{https://phet.colorado.edu/en/simulations/stern-gerlach}


% 19. Hydrogen atom
% Sims: Models of the Hydrogen Atom, Rutherford Scattering (also falstad.com/qmatom)
% • It is extremely important in this section to relate the Schrodinger model of the atom back to
% the discussion of models of the atom earlier in the course (section 8), and show how this is the
% next step in the progression of models. Otherwise, students are likely to view this section as
% just one more example of a solution of the Schrodinger equation and not realize that we are
% actually talking about another model of the atom.
% • We have a homework in which we ask students to work through the simulations and explain
% the reasons for and limitations of each models. It’s amazing how difficult this is for students.

%--------------------
\printbibliography[segment=\therefsegment,heading=subbibliography]
