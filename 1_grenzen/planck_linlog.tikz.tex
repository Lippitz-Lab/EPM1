% \documentclass{standalone}
% \input{../tikz_header}




% \begin{document}

% Author: Izaak Neutelings (March 2019)
% from https://tikz.net/blackbody_plots/
% with modifications by ML


% BLACK BODY - 3000, 4000, 5000K
\begin{tikzpicture}
  \def\N{60}
  \def\xmax{2100}
  \def\ymax{1.43e10}

  \begin{groupplot}
    [
        group style={
            group size=1 by 2,
            % columns=2,
            % rows=2,
          %  xlabels at=edge bottom,
          %  ylabels at=edge left,
            vertical sep= 13mm
        },
    ]

  \nextgroupplot[
    font=\footnotesize,
      every axis plot/.style={
        mark=none,samples=\N,domain=5:\xmax,smooth},
      xmin=(0), xmax=(\xmax),
      ymin=(0), ymax=(\ymax),
      restrict y to domain=0:\ymax,
      %axis lines=middle,
     % axis line style=thick,
     % tick style={black,thick},
     % ticklabel style={scale=0.8},
      xlabel={Wellenlänge $\lambda$ (nm)},
      ylabel={Leistung $P$ (willk.E.) },
      ytick = {0},
      %xlabel style={below=-1pt,font=\small},
      %ylabel style={above=-1pt},
      width=55mm, height=40mm,
      %tick scale binop=\times,
      %every y tick scale label/.style={at={(rel axis cs:0,1)},anchor=south}]
      x tick label style={/pgf/number format/.cd,%
          scaled x ticks = false,
          set thousands separator={\, },
          fixed},    
      y tick label style={/pgf/number format/.cd,%
          scaled y ticks = false,
          set thousands separator={\, },
          fixed},  
          ]
    
    % RAINBOW    
    \begin{scope}
      \clip[variable=\x,domain=200:1000,samples=40]
        plot(\x,{planck(\x,5000)}) |- (200,0) -- cycle;
      \shade[shading=rainbow,shading angle=90,opacity=0.7] (380,0) rectangle (740,\ymax);
    \end{scope}
    
    % PLANCK
    \addplot[ thick,black]    {planck(x,3000)};
    \addplot[ thick,blue] {planck(x,4000)};
    \addplot[ thick,samples=3*\N,red] {planck(x,5000)};
    \addplot[dashed,,red,domain=1000:4000]   {rayleighjeans(x,5000)};
    
    % MAXIMUM (Wien's displacement law)
    \addplot[black, thin,variable=T,domain=2200:4000,samples=40]
      ({lampeak(T)},{planck(lampeak(T),T)});
    \addplot[black, thin,variable=T,domain=4000:5500,samples=100]
      ({lampeak(T)},{planck(lampeak(T),T)});
    \fill[black] ({lampeak(3000)},{planck(lampeak(3000),3000)}) circle(1.5pt);
    \fill[black] ({lampeak(4000)},{planck(lampeak(4000),4000)}) circle(1.5pt);
    \fill[black] ({lampeak(5000)},{planck(lampeak(5000),5000)}) circle(1.5pt);
    

     \node[above right=-1pt,red]
       at (1400,{rayleighjeans(1400,5000)}) {Rayleigh};
    
 

    \nextgroupplot[ font=\footnotesize,
    ymode=log, xmode=log,
        %every axis plot/.style={
        %   thick,mark=none,domain=\xmin:\xmax,samples=\N,smooth},
        xmin=1.5e2, xmax=(1.01*2e5),
        ymin=5e-8, ymax=4,
        %restrict y to domain=5e-8:4,
        domain = 1e2:2e5,
        log basis y=10,
        %axis line style=thick,
        %tick style={black,thick},
        %ticklabel style={scale=0.8},
        %ticks=none,
        max space between ticks=23,
        yminorticks=false,
        xlabel={Wellenlänge $\lambda$ (nm)},
        ylabel={Leistung $P/P_0$ },
       % xlabel style={at={(rel axis cs:0.5,0)},below=8pt},
       % ylabel style={above=-2pt},
        width=55mm, height=40mm,
       % legend style={at={(0.14,0.15)},anchor=south west,draw=none,fill=none,font=\small},
       % legend cell align={left},
       % axis line on top
      ]

      

      % RAINBOW
       \shade[shading=rainbow,shading angle=90,opacity=0.1] (380,5e-8) rectangle (740,4);
        
%       % PLANCK
       \addplot[black,samples=100]      {planck(x ,3000)/1e9};
       \addplot[thick,blue] {rayleighjeans(x ,3000)/1e9};
       \addplot[thick,red]  {wien(x ,3000)/1e9};
    
%     %   % LEGENDS

 \node at (25e3, 0.3) {\footnotesize Rayleigh-Jeans};
\node at (15e3,1e-6) {\footnotesize Wien};
      
     \end{groupplot}


\end{tikzpicture}



  


%\end{document}


