\renewcommand{\lastmod}{10. September 2024}
\renewcommand{\chapterauthors}{Markus Lippitz}

\chapter{Grenzen der klassischen Physik}



\goal{By the end of this chapter, you should be able to draw, calculate and align a ray's path through an optical system.}



\section{Overview}

I assume that you have seen a little bit of geometrical optics in your studies, but we will briefly review it. We will introduce the postulates of ray optics and discuss rays at a mirror and a lens as an example. I will also introduce the matrix method of ray optics, which is a very convenient way of calculating the path of a ray through a system of optical elements. More details on these topics can be found in chapter 1 of \cite{SalehTeich1991}, chapter 2 of \cite{Hering_Martin_Optik},  chapters 5 and 6 of \cite{Hecht_Optics}, chapter 2 of \cite{Konijnenberg_Optics}.


\section{Postulates of ray optics}


%--------------------
\printbibliography[segment=\therefsegment,heading=subbibliography]
