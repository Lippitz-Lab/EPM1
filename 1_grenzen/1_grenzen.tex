\renewcommand{\lastmod}{10. September 2024}
\renewcommand{\chapterauthors}{Markus Lippitz}

\chapter{Grenzen der klassischen Physik}



\goal{By the end of this chapter, you should be able to draw, calculate and align a ray's path through an optical system.}


Ich kann erklären, wie experimentell zwischen den Atom-Modellen von Thomson und Rutherford unterschieden werden kann.

\section{Overview}

s.a. Demtröder 3, Kap. 2


37.1 Matter and Light 1116

37.2 The Emission and Absorption of Light 1116

37.3 Cathode Rays and X Rays 1119

37.4 The Discovery of the Electron 1121

37.5 The Fundamental Unit of Charge 1124

37.6 The Discovery of the Nucleus 1125

37.7 Into the Nucleus 1129

37.8 Classical Physics at the Limit 1131

\phet{Radio_Waves_and_Electromagnetic_Fields}

PLUS UV Katastrophe

2.1 Schwarzkörper und Hohlräume [2]1	



2.4 Wien’sches Verschiebegesetz 4	

2.5 Treibhaus

6.2 Franck-Hertz-Versuch2	hier ??



% 4. Rutherford Scattering
% Sim: Rutherford Scattering
% • In interviews we found that after instruction some students described the plum pudding model
% as cloud of negative charge filled with protons. They’re probably mixing it up with the
% Schrodinger model.


%--------------------
\printbibliography[segment=\therefsegment,heading=subbibliography]
