% \documentclass{standalone}
% \usepackage{tikz,tikz-3dplot}

\DeclareUnicodeCharacter{0393}{$\Gamma$} 
\DeclareUnicodeCharacter{03C3}{$\sigma$} 


\newcommand{\inputtikz}[1]{%

 \tikzexternalenable
  \tikzsetnextfilename{#1}%
  \input{#1.tikz}%
  \tikzexternaldisable

}


\usetikzlibrary{math,matrix,fit,positioning,intersections}

\usetikzlibrary{calc}
\usetikzlibrary{arrows.meta} %needed tikz library

\usepackage{standalone}
\usepackage{pgfplots}
 \pgfplotsset{compat=newest}
\usepgfplotslibrary{groupplots}
\usepgfplotslibrary{fillbetween}

\tikzset{>=latex}

\usepackage{tikzorbital}
 \usepackage{tikzsymbols}
\usetikzlibrary{quotes,angles}

\usepackage{currfile,hyperxmp}


% \pgfplotsset{
% tufte line/.style={
%     axis line style={draw opacity=0},
%     ytick=\empty,
%     axis x line*=bottom,
%     x axis line style={
%       draw opacity=1,
%       gray,
%       thick
% },
%  %   yticklabel=\pgfmathprintnumber{\tick}
%   }
%   }

% \tikzset{
% mymat/.style={
%     matrix of math nodes,
%     left delimiter=|, right delimiter=|,
%     align=center,
%     column sep=-\pgflinewidth,
% }
% %,mymats/.style={
% %    mymat,
% %    nodes={draw,fill=#1}
% %} 
%  }
 
% \newcommand{\myarrow}[5]{\draw[#4](#1.south -| #2)  -- ++(#3 :6mm) node[above,pos=0.55]{$#5$};
% } 

% \newcommand{\interactLp}[3]{\myarrow{#1-#2-1}{#1.west}{-135}{<-}{#3}} 
% \newcommand{\interactLm}[3]{\myarrow{#1-#2-1}{#1.west}{+135}{->}{#3}} 
% \newcommand{\interactRp}[3]{\myarrow{#1-#2-2}{#1.east}{ -45}{<-}{#3}} 
% \newcommand{\interactRm}[3]{\myarrow{#1-#2-2}{#1.east}{ +45}{->}{#3}}  

% \newcommand{\interactout}[2]{\myarrow{#1-1-1}{#1.west}{+135}{->,dashed}{#2}} 


\newcommand{\benzene}[8]{%
\tikzmath{\x1 = #1; \dx1 = 0.5; \dx2 = 0.9; \ps=0.5;}
\tikzmath{\x2 = \x1 + \dx1 ;}
\tikzmath{\x3 = \x2 + \dx2 ;}
\tikzmath{\x4 = \x3 + \dx1 ;}

\tikzmath{\y1 = #2; \dy = 0.5;}
\tikzmath{\y2 = \y1 + \dy ;}
\tikzmath{\y3 = \y2 + \dy ;}

\orbital[pos = {(\x1,\y2)},scale=#3 * \ps]{pz}
\orbital[pos = {(\x2,\y1)},scale=#4 * \ps]{pz}
\orbital[pos = {(\x3,\y1)},scale=#5 * \ps]{pz}
\orbital[pos = {(\x4,\y2)},scale=#6 * \ps]{pz}
\orbital[pos = {(\x3,\y3)},scale=#7 * \ps]{pz}
\orbital[pos = {(\x2,\y3)},scale=#8 * \ps]{pz}

\draw (\x1,\y2) -- (\x2,\y1) -- (\x3,\y1) -- (\x4,\y2) --(\x3,\y3) 
-- (\x2,\y3) -- (\x1,\y2);
}

%from https://tex.stackexchange.com/questions/464580/drawing-a-perspective-ellipse-with-tikz/464588#464588
% and https://tex.stackexchange.com/a/438695/121799

\makeatletter
%along x axis
\define@key{x sphericalkeys}{radius}{\def\myradius{#1}}
\define@key{x sphericalkeys}{theta}{\def\mytheta{#1}}
\define@key{x sphericalkeys}{phi}{\def\myphi{#1}}
\tikzdeclarecoordinatesystem{x spherical}{% %%%rotation around x
    \setkeys{x sphericalkeys}{#1}%
    \pgfpointxyz{\myradius*cos(\mytheta)}{\myradius*sin(\mytheta)*cos(\myphi)}{\myradius*sin(\mytheta)*sin(\myphi)}}

%along y axis
\define@key{y sphericalkeys}{radius}{\def\myradius{#1}}
\define@key{y sphericalkeys}{theta}{\def\mytheta{#1}}
\define@key{y sphericalkeys}{phi}{\def\myphi{#1}}
\tikzdeclarecoordinatesystem{y spherical}{% %%%rotation around x
    \setkeys{y sphericalkeys}{#1}%
    \pgfpointxyz{\myradius*sin(\mytheta)*cos(\myphi)}{\myradius*cos(\mytheta)}{\myradius*sin(\mytheta)*sin(\myphi)}}

%along z axis
\define@key{z sphericalkeys}{radius}{\def\myradius{#1}}
\define@key{z sphericalkeys}{theta}{\def\mytheta{#1}}
\define@key{z sphericalkeys}{phi}{\def\myphi{#1}}
\tikzdeclarecoordinatesystem{z spherical}{% %%%rotation around x
    \setkeys{z sphericalkeys}{#1}%
    \pgfpointxyz{\myradius*sin(\mytheta)*cos(\myphi)}{\myradius*sin(\mytheta)*sin(\myphi)}{\myradius*cos(\mytheta)}}


\makeatother % https://tex.stackexchange.com/a/438695/121799






% \begin{document}

% Author: Izaak Neutelings (March 2019)
%v from https://tikz.net/blackbody_plots/



% BLACK BODY - 3000, 4000, 5000K
\begin{tikzpicture}
  \def\N{60}
  \def\xmax{2100}
  \def\ymax{1.43e10}
 % \def\tick#1#2{\draw[thick] (#1+.01*\ymax) -- (#1-.01*\ymax) node[below=-.5pt,scale=0.75] {#2};}
  \begin{axis}[
    font=\footnotesize,
      every axis plot/.style={
        mark=none,samples=\N,domain=5:\xmax,smooth},
      xmin=(0), xmax=(\xmax),
      ymin=(0), ymax=(\ymax),
      restrict y to domain=0:\ymax,
      %axis lines=middle,
     % axis line style=thick,
     % tick style={black,thick},
     % ticklabel style={scale=0.8},
      xlabel={Wellenlänge $\lambda$ (nm)},
      ylabel={Power $P$ (willk.E.) },
      ytick = {0},
      %xlabel style={below=-1pt,font=\small},
      %ylabel style={above=-1pt},
      width=55mm, height=40mm,
      %tick scale binop=\times,
      %every y tick scale label/.style={at={(rel axis cs:0,1)},anchor=south}]
      x tick label style={/pgf/number format/.cd,%
          scaled x ticks = false,
          set thousands separator={\, },
          fixed},    
      y tick label style={/pgf/number format/.cd,%
          scaled y ticks = false,
          set thousands separator={\, },
          fixed},  
          ]
    
    % RAINBOW
    %\draw[dashed] (380,{planck(380,5000)}) -- (380,\ymax);
    %\draw[dashed] (740,{planck(740,5000)}) -- (740,\ymax);
    
    \begin{scope}
      \clip[variable=\x,domain=200:1000,samples=40]
        plot(\x,{planck(\x,5000)}) |- (200,0) -- cycle;
      \shade[shading=rainbow,shading angle=90,opacity=0.7] (380,0) rectangle (740,\ymax);
    \end{scope}
    
    % PLANCK
    \addplot[ thick,black]    {planck(x,3000)};
    \addplot[ thick,blue] {planck(x,4000)};
    \addplot[ thick,samples=3*\N,red] {planck(x,5000)};
    \addplot[dashed,,red,domain=1000:4000]   {rayleighjeans(x,5000)};
    
    % MAXIMUM (Wien's displacement law)
    \addplot[black, thin,variable=T,domain=2200:4000,samples=40]
      ({lampeak(T)},{planck(lampeak(T),T)});
    \addplot[black, thin,variable=T,domain=4000:5500,samples=100]
      ({lampeak(T)},{planck(lampeak(T),T)});
    \fill[black] ({lampeak(3000)},{planck(lampeak(3000),3000)}) circle(1.5pt);
    \fill[black] ({lampeak(4000)},{planck(lampeak(4000),4000)}) circle(1.5pt);
    \fill[black] ({lampeak(5000)},{planck(lampeak(5000),5000)}) circle(1.5pt);
    
    % LABELS
    % \node[above=0pt,scale=0.75,red]
    %   at (1150,{planck(1150,3000)}) {\SI{3000}{K}};
    % \node[above right=-1pt,scale=0.75,orange!80!black]
    %   at (740,{planck(740,4000)}) {\SI{4000}{K}};
    % \node[above right=-1pt,scale=0.75,blue]
    %   at (800,{planck(800,5000)}) {\SI{5000}{K}};
     \node[above right=-1pt,scale=0.75,red]
       at (1400,{rayleighjeans(1400,5000)}) {Rayleigh-J.};
    
    % LABELS
    % \node[below=2pt,scale=0.8] at (200,\ymax) {\strut UV}; % 10 - 400 nm
    % \node[below=2pt,scale=0.8] at (562,\ymax) {\strut optical}; % 380 - 740 nm
    % \node[below=2pt,scale=0.8] at (920,\ymax) {\strut IR}; % 740 - 1050 nm
    
  \end{axis}
\end{tikzpicture}



  


%\end{document}


